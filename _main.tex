% Options for packages loaded elsewhere
\PassOptionsToPackage{unicode}{hyperref}
\PassOptionsToPackage{hyphens}{url}
%
\documentclass[
  openany]{book}
\usepackage{lmodern}
\usepackage{amssymb,amsmath}
\usepackage{ifxetex,ifluatex}
\ifnum 0\ifxetex 1\fi\ifluatex 1\fi=0 % if pdftex
  \usepackage[T1]{fontenc}
  \usepackage[utf8]{inputenc}
  \usepackage{textcomp} % provide euro and other symbols
\else % if luatex or xetex
  \usepackage{unicode-math}
  \defaultfontfeatures{Scale=MatchLowercase}
  \defaultfontfeatures[\rmfamily]{Ligatures=TeX,Scale=1}
\fi
% Use upquote if available, for straight quotes in verbatim environments
\IfFileExists{upquote.sty}{\usepackage{upquote}}{}
\IfFileExists{microtype.sty}{% use microtype if available
  \usepackage[]{microtype}
  \UseMicrotypeSet[protrusion]{basicmath} % disable protrusion for tt fonts
}{}
\makeatletter
\@ifundefined{KOMAClassName}{% if non-KOMA class
  \IfFileExists{parskip.sty}{%
    \usepackage{parskip}
  }{% else
    \setlength{\parindent}{0pt}
    \setlength{\parskip}{6pt plus 2pt minus 1pt}}
}{% if KOMA class
  \KOMAoptions{parskip=half}}
\makeatother
\usepackage{xcolor}
\IfFileExists{xurl.sty}{\usepackage{xurl}}{} % add URL line breaks if available
\IfFileExists{bookmark.sty}{\usepackage{bookmark}}{\usepackage{hyperref}}
\hypersetup{
  pdftitle={The Photographer's Guide to Los Santos},
  hidelinks,
  pdfcreator={LaTeX via pandoc}}
\urlstyle{same} % disable monospaced font for URLs
\usepackage{color}
\usepackage{fancyvrb}
\newcommand{\VerbBar}{|}
\newcommand{\VERB}{\Verb[commandchars=\\\{\}]}
\DefineVerbatimEnvironment{Highlighting}{Verbatim}{commandchars=\\\{\}}
% Add ',fontsize=\small' for more characters per line
\usepackage{framed}
\definecolor{shadecolor}{RGB}{248,248,248}
\newenvironment{Shaded}{\begin{snugshade}}{\end{snugshade}}
\newcommand{\AlertTok}[1]{\textcolor[rgb]{0.94,0.16,0.16}{#1}}
\newcommand{\AnnotationTok}[1]{\textcolor[rgb]{0.56,0.35,0.01}{\textbf{\textit{#1}}}}
\newcommand{\AttributeTok}[1]{\textcolor[rgb]{0.77,0.63,0.00}{#1}}
\newcommand{\BaseNTok}[1]{\textcolor[rgb]{0.00,0.00,0.81}{#1}}
\newcommand{\BuiltInTok}[1]{#1}
\newcommand{\CharTok}[1]{\textcolor[rgb]{0.31,0.60,0.02}{#1}}
\newcommand{\CommentTok}[1]{\textcolor[rgb]{0.56,0.35,0.01}{\textit{#1}}}
\newcommand{\CommentVarTok}[1]{\textcolor[rgb]{0.56,0.35,0.01}{\textbf{\textit{#1}}}}
\newcommand{\ConstantTok}[1]{\textcolor[rgb]{0.00,0.00,0.00}{#1}}
\newcommand{\ControlFlowTok}[1]{\textcolor[rgb]{0.13,0.29,0.53}{\textbf{#1}}}
\newcommand{\DataTypeTok}[1]{\textcolor[rgb]{0.13,0.29,0.53}{#1}}
\newcommand{\DecValTok}[1]{\textcolor[rgb]{0.00,0.00,0.81}{#1}}
\newcommand{\DocumentationTok}[1]{\textcolor[rgb]{0.56,0.35,0.01}{\textbf{\textit{#1}}}}
\newcommand{\ErrorTok}[1]{\textcolor[rgb]{0.64,0.00,0.00}{\textbf{#1}}}
\newcommand{\ExtensionTok}[1]{#1}
\newcommand{\FloatTok}[1]{\textcolor[rgb]{0.00,0.00,0.81}{#1}}
\newcommand{\FunctionTok}[1]{\textcolor[rgb]{0.00,0.00,0.00}{#1}}
\newcommand{\ImportTok}[1]{#1}
\newcommand{\InformationTok}[1]{\textcolor[rgb]{0.56,0.35,0.01}{\textbf{\textit{#1}}}}
\newcommand{\KeywordTok}[1]{\textcolor[rgb]{0.13,0.29,0.53}{\textbf{#1}}}
\newcommand{\NormalTok}[1]{#1}
\newcommand{\OperatorTok}[1]{\textcolor[rgb]{0.81,0.36,0.00}{\textbf{#1}}}
\newcommand{\OtherTok}[1]{\textcolor[rgb]{0.56,0.35,0.01}{#1}}
\newcommand{\PreprocessorTok}[1]{\textcolor[rgb]{0.56,0.35,0.01}{\textit{#1}}}
\newcommand{\RegionMarkerTok}[1]{#1}
\newcommand{\SpecialCharTok}[1]{\textcolor[rgb]{0.00,0.00,0.00}{#1}}
\newcommand{\SpecialStringTok}[1]{\textcolor[rgb]{0.31,0.60,0.02}{#1}}
\newcommand{\StringTok}[1]{\textcolor[rgb]{0.31,0.60,0.02}{#1}}
\newcommand{\VariableTok}[1]{\textcolor[rgb]{0.00,0.00,0.00}{#1}}
\newcommand{\VerbatimStringTok}[1]{\textcolor[rgb]{0.31,0.60,0.02}{#1}}
\newcommand{\WarningTok}[1]{\textcolor[rgb]{0.56,0.35,0.01}{\textbf{\textit{#1}}}}
\usepackage{longtable,booktabs}
% Correct order of tables after \paragraph or \subparagraph
\usepackage{etoolbox}
\makeatletter
\patchcmd\longtable{\par}{\if@noskipsec\mbox{}\fi\par}{}{}
\makeatother
% Allow footnotes in longtable head/foot
\IfFileExists{footnotehyper.sty}{\usepackage{footnotehyper}}{\usepackage{footnote}}
\makesavenoteenv{longtable}
\usepackage{graphicx,grffile}
\makeatletter
\def\maxwidth{\ifdim\Gin@nat@width>\linewidth\linewidth\else\Gin@nat@width\fi}
\def\maxheight{\ifdim\Gin@nat@height>\textheight\textheight\else\Gin@nat@height\fi}
\makeatother
% Scale images if necessary, so that they will not overflow the page
% margins by default, and it is still possible to overwrite the defaults
% using explicit options in \includegraphics[width, height, ...]{}
\setkeys{Gin}{width=\maxwidth,height=\maxheight,keepaspectratio}
% Set default figure placement to htbp
\makeatletter
\def\fps@figure{htbp}
\makeatother
\setlength{\emergencystretch}{3em} % prevent overfull lines
\providecommand{\tightlist}{%
  \setlength{\itemsep}{0pt}\setlength{\parskip}{0pt}}
\setcounter{secnumdepth}{5}

\title{The Photographer's Guide to Los Santos}
\author{}
\date{\vspace{-2.5em}}

\begin{document}
\maketitle

{
\setcounter{tocdepth}{1}
\tableofcontents
}
\hypertarget{preface}{%
\chapter*{Preface}\label{preface}}
\addcontentsline{toc}{chapter}{Preface}

\begin{quote}
``Los Santos. The city of shitheads. Where else would he be?''

--- Trevor Philips
\end{quote}

\hypertarget{introduction}{%
\chapter{Introduction}\label{introduction}}

\hypertarget{about-the-photographers-guide-to-los-santos}{%
\section{About The Photographer's Guide to Los Santos}\label{about-the-photographers-guide-to-los-santos}}

The Photographer's Guide to Los Santos sits between a touristic guide and a photography manual, and between an exhibition catalogue and a peak behind the scenes of artwork creation.

The Photographer's Guide to Los Santos is an ongoing project that builds on top of a research on artistic practices within spaces of computer games, with a particular focus on in-game photography, machinima and digital visual arts. It follows some themes and ideas previously explored in the exhibition \href{https://www.howtowinat.photography/}{\emph{How to Win at Photography}}, while focusing more specifically on the relationship between computer games and photographic activities inside the world of Grand Theft Auto V.

The idea of a guide refers to in-game photography as a form of `virtual tourism' (\href{https://papers.ssrn.com/sol3/papers.cfm?abstract_id=538182}{Book, 2003}), which was also the premise of an actual tourist guide published by Rough Guides in their 2019 \href{https://www.roughguides.com/articles/introduction-to-the-rough-guide-to-xbox/}{\emph{Rough Guide to XBOX}}. Yet this guide project also understands the game world as a site for image production and artistic creation, turning the game into a destination for a `game art tourist'. The Photographer's Guide to Los Santos presents the game environment of Grand Theft Auto V both as a space to explore and in which to create images, as well as a place to navigate and learn about some of the most important artworks that it has enabled to create.

The project also brings together several experiences from teaching in-game photography as an artistic practice in different educational settings and institutions, compiling materials and tools for students and artists interested in engaging with the field. The tourist guide of the game world doubles as a photography manual for the in-game photography age, featuring tutorials and excercises ranging from game screenshotting to computer programming for creative modding. Through the practical excercises, the project invites to rethink the game object as a space for creative, subversive and critical endeavours, which can be played differently, documented, reclaimed or modified through an artistic approach.

Finally, the project draws inspiration from the works of artists who have explored the `metaplay' of photographing game words instead of following the game rules and attempt to reach the goal of winning. The Photographer's Guide to Los Santos is indebted to all the artists it features, but was particularly inspired by Gareth Damian Martin's live streamed workshop \href{https://www.twitch.tv/videos/591840067}{\emph{Photography Tour of No Man's Sky}} (realized for \href{https://www.somersethouse.org.uk/whats-on/now-play-this-2020}{Now Play This Festival 2020}), Total Refusal \& Ismaël Joffroy Chandoutis's 2021 in-game lecture performance and guided tour \href{https://vimeo.com/506064357}{\emph{Everyday Daylight}} (realized for the \href{https://ccsparis.com/en/events/total-refusal-digital-disarmament-movement-a-la-gaite-lyrique/}{CCS Paris}), and Alan Butler's epic 2020 live endurance performance \href{https://www.youtube.com/watch?v=R4Q2G6tOQ_Q}{\emph{Witness to a Changing West}} (realized for \href{https://screenwalks.com/}{Screen Walks}) and his `Content Replication Assignments'.

\hypertarget{grand-theft-auto-v-studies}{%
\section{Grand Theft Auto V Studies}\label{grand-theft-auto-v-studies}}

Los Santos is the Grand Theft Auto V's fictional, parodic version of real-life Los Angeles. Just like Los Angeles is the global centre of film and commercial media production, Los Santos is the epicentre of in-game photography and machinima creation. While it may seem reductive to only focus on a single game to address the larger phenomenon of in-game photography, GTA V is the biggest source of creative outputs to date, with its extended open world and one of the largest community of active modders. Launched in 2013, the game contains a world map of more than 80 square kilometers of total area, which includes the urban area of the city of Los Santos and the rural area of Blaine County. This incredibly vast environment features a large desert region, dense forest, several mountains, beachside towns, on top of the large metropolis of Los Santos. The game simulates the everyday life of hundreds of individual NPCs (while it allegedly counts a population of over 4 million) as well as counting 28 animal species, and \href{https://grandtheftdata.com/landmarks/\#0,0,2,satellite}{more than 800 buildings in GTA V are based on real-life landmarks}. The size of the photorealistic simulation is only matched by the complexity of the game engine and its code, which - thanks to the effort of GTA V's modding community - allows players to use the game world as a powerful tool to create new scenes, take controls of its algorithmic entities, modify cameras and reshaping the game into a movie set or a photo studio.

\href{https://www.bragitoff.com/2015/11/gta-v-maps-quad-ultra-high-definition-8k-quality/gtav_satellite_2048x2048/}{GTA V Satellite Map by Manas Sharma}

\hypertarget{grand-theft-auto-v-tourism}{%
\section{Grand Theft Auto V Tourism}\label{grand-theft-auto-v-tourism}}

This guide allows players to explore the game environment following some of the most interesting artworks that have been created with(in) it. The guide is divided in thematic chapters that follow different artistic practices, taking place in different locations of the game environment, followed by different tutorials and exercises connected with the works and the space analyzed.
Each selected work is presented by a curatorial statement, introducing the work and its artistic relevance. The work is accompanied by information on the in-game location from which it was produced, inviting the readers to reach the destination in Grand Theft Auto V through maps and indications.

\hypertarget{grand-theft-auto-v-art-education}{%
\section{Grand Theft Auto V Art Education}\label{grand-theft-auto-v-art-education}}

Each thematic chapter features a tutorial section that introduces different techniques and strategies to capture images within Grand Theft Auto V. The chapters are thought to be experienced in order, as the tutorials at times rely on knowledge that is built on top of previous lessons. Each tutorial is accompanied by content replication assignments, in which the readers is invited to use the skills learnd from each chapter to recreate a work presented in that section.

\hypertarget{architecture-photography}{%
\chapter{Architecture Photography}\label{architecture-photography}}

\hypertarget{the-continuous-city-by-gareth-damian-martin}{%
\section*{\texorpdfstring{\emph{The Continuous City}, by Gareth Damian Martin}{The Continuous City, by Gareth Damian Martin}}\label{the-continuous-city-by-gareth-damian-martin}}
\addcontentsline{toc}{section}{\emph{The Continuous City}, by Gareth Damian Martin}

Gareth Damian Martin, \emph{Outskirts}, from \emph{The Continuous City},

Gareth Damian Martin, \emph{Pathways}, from \emph{The Continuous City},

artwork text

\href{https://socks-studio.com/2019/10/13/gareth-damian-martin-postcards-from-the-continuous-city-2018/}{More about \emph{The Continuous City}}

\href{https://www.gamescenes.org/2018/04/interview-gareth-damian-martin-the-aesthetics-of-analogue-game-photography.html}{Interview with Gareth Damian Martin}

\hypertarget{getting-there}{%
\subsection*{Getting there}\label{getting-there}}
\addcontentsline{toc}{subsection}{Getting there}

The intersection of Interstate 4 and Interstate 5.

\hypertarget{readings}{%
\section*{Readings}\label{readings}}
\addcontentsline{toc}{section}{Readings}

\href{https://www.heterotopiaszine.com/}{Heterotopias}

Mark D Teo, The Urban Architecture of Los Angeles and Grand Theft Auto, 2015. \url{https://www.academia.edu/18173221/The_Urban_Architecture_of_Los_Angeles_and_Grand_Theft_Auto}

\hypertarget{tutorial}{%
\section*{Tutorial}\label{tutorial}}
\addcontentsline{toc}{section}{Tutorial}

\hypertarget{photographing-the-game-screen}{%
\subsection*{Photographing the Game Screen}\label{photographing-the-game-screen}}
\addcontentsline{toc}{subsection}{Photographing the Game Screen}

\hypertarget{analogue-game-photography}{%
\subsubsection*{Analogue Game Photography}\label{analogue-game-photography}}
\addcontentsline{toc}{subsubsection}{Analogue Game Photography}

\hypertarget{screenshotting}{%
\subsubsection*{Screenshotting}\label{screenshotting}}
\addcontentsline{toc}{subsubsection}{Screenshotting}

\hypertarget{content-replication-assignment}{%
\section*{Content Replication Assignment}\label{content-replication-assignment}}
\addcontentsline{toc}{section}{Content Replication Assignment}

\hypertarget{social-documentary}{%
\chapter{Social Documentary}\label{social-documentary}}

\hypertarget{down-and-out-in-los-santos-by-alan-butler}{%
\section*{\texorpdfstring{\emph{Down and Out in Los Santos} by Alan Butler}{Down and Out in Los Santos by Alan Butler}}\label{down-and-out-in-los-santos-by-alan-butler}}
\addcontentsline{toc}{section}{\emph{Down and Out in Los Santos} by Alan Butler}

artwork text

\href{http://www.alanbutler.info/down-and-out-in-los-santos-2016}{More about \emph{Down and Out in Los Santos}}

\hypertarget{getting-there-1}{%
\subsection*{Getting There}\label{getting-there-1}}
\addcontentsline{toc}{subsection}{Getting There}

The homeless camp in Los Santos is under the Olympic Freeway in Strawberry.

Dignity Village is a tent city established by homeless people near Procopio Beach, east of Paleto Bay.

\hypertarget{fear-and-loathing-in-gta-v-by-morten-rockford-ravn}{%
\section*{\texorpdfstring{\emph{Fear and Loathing in GTA V} by Morten Rockford Ravn}{Fear and Loathing in GTA V by Morten Rockford Ravn}}\label{fear-and-loathing-in-gta-v-by-morten-rockford-ravn}}
\addcontentsline{toc}{section}{\emph{Fear and Loathing in GTA V} by Morten Rockford Ravn}

artwork text

\href{https://fearandloathingingtav.tumblr.com/}{More about \emph{Fear and Loathing in GTA V}}

\hypertarget{getting-there-2}{%
\subsection*{Getting There}\label{getting-there-2}}
\addcontentsline{toc}{subsection}{Getting There}

\hypertarget{readings-1}{%
\section*{Readings}\label{readings-1}}
\addcontentsline{toc}{section}{Readings}

\hypertarget{tutorial-1}{%
\section*{Tutorial}\label{tutorial-1}}
\addcontentsline{toc}{section}{Tutorial}

\hypertarget{in-game-smartphone-camera}{%
\subsection*{In-game Smartphone Camera}\label{in-game-smartphone-camera}}
\addcontentsline{toc}{subsection}{In-game Smartphone Camera}

\hypertarget{content-replication-assignment-1}{%
\section*{Content Replication Assignment}\label{content-replication-assignment-1}}
\addcontentsline{toc}{section}{Content Replication Assignment}

\hypertarget{re-enactment-photography}{%
\chapter{Re-enactment photography}\label{re-enactment-photography}}

\hypertarget{gasoline-stations-in-gta-v-by-lorna-ruth-galloway}{%
\section*{\texorpdfstring{\emph{26 Gasoline stations in GTA V} by Lorna Ruth Galloway}{26 Gasoline stations in GTA V by Lorna Ruth Galloway}}\label{gasoline-stations-in-gta-v-by-lorna-ruth-galloway}}
\addcontentsline{toc}{section}{\emph{26 Gasoline stations in GTA V} by Lorna Ruth Galloway}

artwork text

\href{https://www.lornaruthgalloway.com/charcoal-halftone}{More about \emph{26 Gasoline stations in GTA V}}

\hypertarget{getting-there-3}{%
\subsection*{Getting There}\label{getting-there-3}}
\addcontentsline{toc}{subsection}{Getting There}

\begin{itemize}
\tightlist
\item
  \href{https://grandtheftdata.com/landmarks/\#513.325,-1374.453,4,hybrid,name=gas_station,Gas_Station,_Capital_Blvd_\&_El_Rancho_Blvd,_East_Los_Santos}{Capital Blvd \& El Rancho Blvd, East Los Santos} (1209.708, -1402.539)
\item
  \href{https://grandtheftdata.com/landmarks/\#513.325,-1374.453,4,hybrid,name=gas_station,Xero_Gas_Station,_Strawberry_Ave_\&_Capital_Blvd,_South_Los_Santos}{Xero Gas Station, Strawberry Ave \& Capital Blvd, South Los Santos}
\item
  \href{https://grandtheftdata.com/landmarks/\#513.325,-1374.453,4,hybrid,name=gas_station,RON_Gas_Station,_Davis_Ave_\&_Macdonald_St,_South_Los_Santos}{RON Gas Station, Davis Ave \& Macdonald St, South Los Santos}
\item
  \href{https://grandtheftdata.com/landmarks/\#513.325,-1374.453,4,hybrid,name=gas_station,Xero_Gas_Station,_Calais_Ave_\&_Innocence_Blvd,_Little_Seoul}{Xero Gas Station, Calais Ave \& Innocence Blvd, Little Seoul}
\item
  \href{https://grandtheftdata.com/landmarks/\#513.325,-1374.453,4,hybrid,name=gas_station,LTD_Gas_Station,_Lindsay_Circus_\&_Ginger_St,_Little_Seoul}{LTD Gas Station, Lindsay Circus \& Ginger St, Little Seoul}
\item
  \href{https://grandtheftdata.com/landmarks/\#-1543.17,-676.107,4,hybrid,name=gas_station,RON_Gas_Station,_N_Rockford_Dr_\&_Perth_St,_Morningwood}{RON Gas Station, N Rockford Dr \& Perth St, Morningwood}
\item
  \href{https://grandtheftdata.com/landmarks/\#-1543.17,-676.107,4,hybrid,name=gas_station,Xero_Gas_Station,_Great_Ocean_Hwy,_Pacific_Bluffs}{Xero Gas Station, Great Ocean Hwy, Pacific Bluffs}
\end{itemize}

\hypertarget{a-study-on-perspective-by-roc-herms}{%
\section*{\texorpdfstring{\emph{A Study on Perspective} by Roc Herms}{A Study on Perspective by Roc Herms}}\label{a-study-on-perspective-by-roc-herms}}
\addcontentsline{toc}{section}{\emph{A Study on Perspective} by Roc Herms}

artwork text

\href{https://www.rocherms.com/projects/study-of-perspective/}{More about \emph{A Study on Perspective}}

\hypertarget{getting-there-4}{%
\subsection*{Getting There}\label{getting-there-4}}
\addcontentsline{toc}{subsection}{Getting There}

\hypertarget{further-references}{%
\section*{Further references}\label{further-references}}
\addcontentsline{toc}{section}{Further references}

\hypertarget{little-books-of-los-santos-by-luke-caspar-pearson}{%
\subsection*{\texorpdfstring{\emph{Little Books of Los Santos} by Luke Caspar Pearson}{Little Books of Los Santos by Luke Caspar Pearson}}\label{little-books-of-los-santos-by-luke-caspar-pearson}}
\addcontentsline{toc}{subsection}{\emph{Little Books of Los Santos} by Luke Caspar Pearson}

\href{https://www.alephograph.com/little-books-of-los-santos}{More about \emph{Little Books of Los Santos}}

\hypertarget{gasoline-stations-in-gta-v-by-m.-earl-williams}{%
\subsection*{\texorpdfstring{\emph{26 Gasoline stations in GTA V} by M. Earl Williams}{26 Gasoline stations in GTA V by M. Earl Williams}}\label{gasoline-stations-in-gta-v-by-m.-earl-williams}}
\addcontentsline{toc}{subsection}{\emph{26 Gasoline stations in GTA V} by M. Earl Williams}

\href{https://www.mearlwilliams.com/gasoline_stations\#1}{More about \emph{26 Gasoline stations in GTA V}}

\hypertarget{readings-2}{%
\section*{Readings}\label{readings-2}}
\addcontentsline{toc}{section}{Readings}

\hypertarget{tutorial-2}{%
\section*{Tutorial}\label{tutorial-2}}
\addcontentsline{toc}{section}{Tutorial}

\hypertarget{scene-director-mode}{%
\subsection*{Scene Director Mode}\label{scene-director-mode}}
\addcontentsline{toc}{subsection}{Scene Director Mode}

\hypertarget{content-replication-assignment-2}{%
\section*{Content Replication Assignment}\label{content-replication-assignment-2}}
\addcontentsline{toc}{section}{Content Replication Assignment}

\hypertarget{nature-documentary}{%
\chapter{Nature Documentary}\label{nature-documentary}}

\hypertarget{san-andreas-streaming-deer-cam-by-brent-watanabe}{%
\section*{\texorpdfstring{\emph{San Andreas Streaming Deer Cam} by Brent Watanabe}{San Andreas Streaming Deer Cam by Brent Watanabe}}\label{san-andreas-streaming-deer-cam-by-brent-watanabe}}
\addcontentsline{toc}{section}{\emph{San Andreas Streaming Deer Cam} by Brent Watanabe}

artwork text

\href{https://bwatanabe.com/GTA_V_WanderingDeer.html}{More about \emph{Deercam}}

\hypertarget{getting-there-5}{%
\subsection*{Getting There}\label{getting-there-5}}
\addcontentsline{toc}{subsection}{Getting There}

Deers can usually be found in the forested areas of Blaine County, but can also be spotted in all rural areas, with the exception being the Grand Senora Desert.

\hypertarget{virtual-botany-cyanotype-by-alan-butler}{%
\section*{\texorpdfstring{\emph{Virtual Botany Cyanotype} by Alan Butler}{Virtual Botany Cyanotype by Alan Butler}}\label{virtual-botany-cyanotype-by-alan-butler}}
\addcontentsline{toc}{section}{\emph{Virtual Botany Cyanotype} by Alan Butler}

artwork text

\href{http://www.alanbutler.info/virtual-botany-cyanotypes}{More about \emph{Virtual Botany Cyanotype}}

\hypertarget{getting-there-6}{%
\subsection*{Getting There}\label{getting-there-6}}
\addcontentsline{toc}{subsection}{Getting There}

\hypertarget{readings-3}{%
\section*{Readings}\label{readings-3}}
\addcontentsline{toc}{section}{Readings}

\hypertarget{tutorial-3}{%
\section*{Tutorial}\label{tutorial-3}}
\addcontentsline{toc}{section}{Tutorial}

\hypertarget{scripting-introduction}{%
\subsection*{Scripting Introduction}\label{scripting-introduction}}
\addcontentsline{toc}{subsection}{Scripting Introduction}

\hypertarget{preparation-and-setup}{%
\subsubsection*{Preparation and Setup}\label{preparation-and-setup}}
\addcontentsline{toc}{subsubsection}{Preparation and Setup}

\begin{itemize}
\item
  Install Windows 11
\item
  Download and install \href{https://store.steampowered.com/about/}{Steam} (with a copy of GTA V or buy the game if you do not have it. GTA V is 100+ GB so it will take a few hours depending on your internet connections)
\item
  Download \href{https://store.steampowered.com/about/}{Script Hook V}, go to the bin folder and copy \texttt{dinput8.dll} and \texttt{ScriptHookV.dll} files into your GTA V directory \texttt{C:\textbackslash{}Program\ Files\ (x86)\textbackslash{}Steam\textbackslash{}steamapps\textbackslash{}common\textbackslash{}Grand\ Theft\ Auto\ V}
\item
  Download \href{https://store.steampowered.com/about/}{Script Hook V dot net}, copy the \texttt{ScriptHookVDotNet.asi} file, \texttt{ScriptHookVDotNet2.dll} and \texttt{ScriptHookVDotNet3.dll} files into your GTA V directory \texttt{C:\textbackslash{}Program\ Files\ (x86)\textbackslash{}Steam\textbackslash{}steamapps\textbackslash{}common\textbackslash{}Grand\ Theft\ Auto\ V}
\item
  Create a new folder in GTA V directory and call it ``scripts''.
\item
  Download and install \href{https://store.steampowered.com/about/}{Visual Studio Community} (free version of VS). Open Visual Studio and check the .NET desktop development package and install it
\item
  Run GTA V and test if Script Hook V is working by pressing \texttt{F4}. This should toggle the console view.
\end{itemize}

\hypertarget{creating-a-mod-file}{%
\subsubsection*{Creating a Mod File}\label{creating-a-mod-file}}
\addcontentsline{toc}{subsubsection}{Creating a Mod File}

\begin{itemize}
\item
  Open Visual Studio
\item
  Select File \textgreater{} New \textgreater{} Project
\item
  Select Visual C\# and Class Library (.NET Framework)
\item
  Give a custom file name (e.g.~moddingTutorial)
\item
  Rename public class Class1 as ``moddingTutorial'' in the right panel Solution Explorer
\item
  In the same panel go to References and click add References\ldots{} \textgreater{} Browse \textgreater{} browse to Downloads
\item
  Select ScriptHookedVDotNet \textgreater{} \texttt{ScriptHookVDotNet2.dll} and \texttt{ScriptHookVDotNet3.dll} and add them
\item
  Also add \texttt{System.Windows.forms}
\item
  Also add \texttt{System.Drawing}
\item
  In your code file add the following lines on top:
\end{itemize}

\begin{Shaded}
\begin{Highlighting}[]
\KeywordTok{using}\NormalTok{ GTA;}
\KeywordTok{using}\NormalTok{ GTA.}\FunctionTok{Math}\NormalTok{;}
\KeywordTok{using}\NormalTok{ System.}\FunctionTok{Windows}\NormalTok{.}\FunctionTok{Forms}\NormalTok{;}
\KeywordTok{using}\NormalTok{ System.}\FunctionTok{Drawing}\NormalTok{;}
\KeywordTok{using}\NormalTok{ GTA.}\FunctionTok{Native}\NormalTok{;}
\end{Highlighting}
\end{Shaded}

\begin{itemize}
\tightlist
\item
  Modify class moddingTutorial to the following:
\end{itemize}

\begin{Shaded}
\begin{Highlighting}[]
\KeywordTok{namespace}\NormalTok{ moddingTutorial}
\NormalTok{\{}
  \KeywordTok{public} \KeywordTok{class}\NormalTok{ moddingTutorial : Script}
\NormalTok{  \{}
      \KeywordTok{public} \FunctionTok{moddingTutorial}\NormalTok{()}
\NormalTok{      \{}
          \KeywordTok{this}\NormalTok{.}\FunctionTok{Tick}\NormalTok{ += onTick;}
      \KeywordTok{this}\NormalTok{.}\FunctionTok{KeyUp}\NormalTok{ += onKeyUp;}
      \KeywordTok{this}\NormalTok{.}\FunctionTok{KeyDown}\NormalTok{ += onKeyDown;}
\NormalTok{      \}}
    
      \KeywordTok{private} \DataTypeTok{void} \FunctionTok{onTick}\NormalTok{(}\DataTypeTok{object}\NormalTok{ sender, EventArgs e)}
\NormalTok{      \{}
\NormalTok{      \}}
    
      \KeywordTok{private} \DataTypeTok{void} \FunctionTok{onKeyUp}\NormalTok{(}\DataTypeTok{object}\NormalTok{ sender, KeyEventArgs e)}
\NormalTok{      \{}
\NormalTok{      \}}
 
      \KeywordTok{private} \DataTypeTok{void} \FunctionTok{onKeyDown}\NormalTok{(}\DataTypeTok{object}\NormalTok{ sender, KeyEventArgs e)}
\NormalTok{      \{}
        \KeywordTok{if}\NormalTok{ (e.}\FunctionTok{KeyCode}\NormalTok{ == Keys.}\FunctionTok{H}\NormalTok{)}
\NormalTok{        \{}
\NormalTok{              Game.}\FunctionTok{Player}\NormalTok{.}\FunctionTok{ChangeModel}\NormalTok{(PedHash.}\FunctionTok{Cat}\NormalTok{); }
\NormalTok{          \}}
\NormalTok{      \} }
\NormalTok{  \}}
\NormalTok{\}}
\end{Highlighting}
\end{Shaded}

\begin{itemize}
\item
  Save file
\item
  Go to Documents \textgreater{} Visual Studio \textgreater{} Project \textgreater{} moddingTutorial \textgreater{} moddingTutorial \textgreater{} \texttt{moddingTutorial.cs}
\item
  Copy the .cs file in the GTA V directory inside the scripts folder
\item
  Open GTA V, run the game in Story Mode (mods are only allowed in single player mode, not in GTA Online) and press `H' to see if the game turns your avatar into a cat
\item
  Note: every time you make changes to your .cs file in the scripts folder you can hit \texttt{F4} to open the console, type \texttt{Reload()} in the console for the program to reload the script and test again the changes.
\end{itemize}

\hypertarget{ontick-onkeyup-and-onkeydown}{%
\subsubsection{onTick, onKeyUp and onKeyDown}\label{ontick-onkeyup-and-onkeydown}}

The main events of Script Hook V Dot Net are onTick, onKeyUp and onKeyDown. Script Hook V Dot Net will invoke your functions whenever an event is called.

The code within the onTick brackets is executed every interval milliseconds (which is by default 0), meaning that the event will be executed at every frame, for as long as the game is running.

\begin{Shaded}
\begin{Highlighting}[]
 \KeywordTok{private} \DataTypeTok{void} \FunctionTok{onTick}\NormalTok{(}\DataTypeTok{object}\NormalTok{ sender, EventArgs e)}
\NormalTok{ \{}
        \CommentTok{//code here will be executed every frame (or per usef defined interval)}
\NormalTok{ \}}
\end{Highlighting}
\end{Shaded}

If your function is written inside onKeyDown (withiin the curly brackets following onKeyUp(object sender, KeyEventArgs e)\{\}), your code will be executed every time a key is pressed. If your function is written inside onKeyUp, your code will be executed every time a key is released.

\begin{Shaded}
\begin{Highlighting}[]
\KeywordTok{private} \DataTypeTok{void} \FunctionTok{onKeyUp}\NormalTok{(}\DataTypeTok{object}\NormalTok{ sender, KeyEventArgs e)}
\NormalTok{\{}
      \CommentTok{//code here will be executed whenever a key is released}
\NormalTok{\}}

\KeywordTok{private} \DataTypeTok{void} \FunctionTok{onKeyDown}\NormalTok{(}\DataTypeTok{object}\NormalTok{ sender, KeyEventArgs e)}
\NormalTok{\{}
      \CommentTok{//code here will be executed whenever a key is pressed}
\NormalTok{\} }
\end{Highlighting}
\end{Shaded}

We can specify which code is executed based on what keys are pressed/released

\begin{Shaded}
\begin{Highlighting}[]
\KeywordTok{private} \DataTypeTok{void} \FunctionTok{onKeyDown}\NormalTok{(}\DataTypeTok{object}\NormalTok{ sender, KeyEventArgs e)}
\NormalTok{\{}
    \KeywordTok{if}\NormalTok{ (e.}\FunctionTok{KeyCode}\NormalTok{ == Keys.}\FunctionTok{H}\NormalTok{)}
\NormalTok{    \{}
        \CommentTok{//code here will be executed whenever the key 'H' is pressed }
\NormalTok{    \}}
\NormalTok{\} }
\end{Highlighting}
\end{Shaded}

\hypertarget{change-player-model}{%
\subsubsection*{Change Player Model}\label{change-player-model}}
\addcontentsline{toc}{subsubsection}{Change Player Model}

The player character is controlled as Game.Player. Game.Player can perform different functions, including changing the avatar model, and performing tasks.

Change the 3D model of your character by using the \texttt{ChangeModel} function.
The function needs a model ID, in order to load the model file of our game character.
You can browse through this list of models to find the one you want to try: \url{https://wiki.gtanet.work/index.php/Peds}

These models are all PedHashes, basically ID numbers within the PedHash group. Copy the name of the model below the image and add it to PedHash.
For example if you choose the model Cat, you'll need to write \texttt{PedHash.Cat}.

To change the model of your player character into a cat you can write the following function:

\begin{Shaded}
\begin{Highlighting}[]
\NormalTok{Game.}\FunctionTok{Player}\NormalTok{.}\FunctionTok{ChangeModel}\NormalTok{(PedHash.}\FunctionTok{Cat}\NormalTok{);}
\end{Highlighting}
\end{Shaded}

add it in your .cs file in the onKeyDown event, triggered by the pressing of the `h' key:

Example code

\begin{Shaded}
\begin{Highlighting}[]
\KeywordTok{using}\NormalTok{ System;}
\KeywordTok{using}\NormalTok{ System.}\FunctionTok{Collections}\NormalTok{.}\FunctionTok{Generic}\NormalTok{;}
\KeywordTok{using}\NormalTok{ System.}\FunctionTok{Linq}\NormalTok{;}
\KeywordTok{using}\NormalTok{ System.}\FunctionTok{Text}\NormalTok{;}
\KeywordTok{using}\NormalTok{ System.}\FunctionTok{Threading}\NormalTok{.}\FunctionTok{Tasks}\NormalTok{;}
 
\KeywordTok{using}\NormalTok{ GTA;}
\KeywordTok{using}\NormalTok{ GTA.}\FunctionTok{Math}\NormalTok{;}
\KeywordTok{using}\NormalTok{ System.}\FunctionTok{Windows}\NormalTok{.}\FunctionTok{Forms}\NormalTok{;}
\KeywordTok{using}\NormalTok{ System.}\FunctionTok{Drawing}\NormalTok{;}
\KeywordTok{using}\NormalTok{ GTA.}\FunctionTok{Native}\NormalTok{;}
 
 
\KeywordTok{namespace}\NormalTok{ moddingTutorial}
\NormalTok{\{}
    \KeywordTok{public} \KeywordTok{class}\NormalTok{ moddingTutorial : Script}
\NormalTok{    \{}
        \KeywordTok{public} \FunctionTok{moddingTutorial}\NormalTok{()}
\NormalTok{        \{}
            \KeywordTok{this}\NormalTok{.}\FunctionTok{Tick}\NormalTok{ += onTick;}
            \KeywordTok{this}\NormalTok{.}\FunctionTok{KeyUp}\NormalTok{ += onKeyUp;}
            \KeywordTok{this}\NormalTok{.}\FunctionTok{KeyDown}\NormalTok{ += onKeyDown;}
\NormalTok{        \}}
 
        \KeywordTok{private} \DataTypeTok{void} \FunctionTok{onTick}\NormalTok{(}\DataTypeTok{object}\NormalTok{ sender, EventArgs e) }\CommentTok{//this function gets executed continuously }
\NormalTok{        \{}
\NormalTok{        \}}
 
        \KeywordTok{private} \DataTypeTok{void} \FunctionTok{onKeyUp}\NormalTok{(}\DataTypeTok{object}\NormalTok{ sender, KeyEventArgs e)}\CommentTok{//everything inside here is executed only when we release a key}
\NormalTok{        \{}
\NormalTok{        \}}
 
        \KeywordTok{private} \DataTypeTok{void} \FunctionTok{onKeyDown}\NormalTok{(}\DataTypeTok{object}\NormalTok{ sender, KeyEventArgs e) }\CommentTok{//everything inside here is executed only when we press a key}
\NormalTok{        \{}
            \CommentTok{//when pressing 'H'}
            \KeywordTok{if}\NormalTok{(e.}\FunctionTok{KeyCode}\NormalTok{ == Keys.}\FunctionTok{H}\NormalTok{)}
\NormalTok{            \{}
                \CommentTok{//change player char into a different model}
\NormalTok{                Game.}\FunctionTok{Player}\NormalTok{.}\FunctionTok{ChangeModel}\NormalTok{(PedHash.}\FunctionTok{Cat}\NormalTok{); }
\NormalTok{            \}}
\NormalTok{        \}}
\NormalTok{    \}}
\NormalTok{\}}
\end{Highlighting}
\end{Shaded}

\hypertarget{tasks}{%
\subsubsection*{Tasks}\label{tasks}}
\addcontentsline{toc}{subsubsection}{Tasks}

Our character can be controlled by our script, and given actions that override manual control of the player. These actions are called \emph{Tasks} and in order to assign tasks to our characters we have to define our \emph{Game.Player} as \emph{Game.Player.Character}. The \emph{Game.Player.Character} code gets the specific model the player is controlling.
Now we can give tasks to the character by adding the \emph{Task} function: \emph{Game.Player.Character.Task}.

\hypertarget{content-replication-assignment-3}{%
\section*{Content Replication Assignment}\label{content-replication-assignment-3}}
\addcontentsline{toc}{section}{Content Replication Assignment}

\hypertarget{deercam-reenactment}{%
\subsection*{Deercam reenactment}\label{deercam-reenactment}}
\addcontentsline{toc}{subsection}{Deercam reenactment}

Write a mod script to change your game character into a deer by pressing a key, and make it autonomously wander around Los Santos by pressing another key.

\hypertarget{surrealist-photography}{%
\chapter{Surrealist Photography}\label{surrealist-photography}}

\hypertarget{alexey-andrienko-aka-happ-v2}{%
\section*{Alexey Andrienko aka HAPP v2}\label{alexey-andrienko-aka-happ-v2}}
\addcontentsline{toc}{section}{Alexey Andrienko aka HAPP v2}

artwork text

\hypertarget{getting-there-7}{%
\subsection*{Getting There}\label{getting-there-7}}
\addcontentsline{toc}{subsection}{Getting There}

\hypertarget{readings-4}{%
\section*{Readings}\label{readings-4}}
\addcontentsline{toc}{section}{Readings}

\hypertarget{tutorial-4}{%
\section*{Tutorial}\label{tutorial-4}}
\addcontentsline{toc}{section}{Tutorial}

\hypertarget{modding-peds}{%
\subsection*{Modding Peds}\label{modding-peds}}
\addcontentsline{toc}{subsection}{Modding Peds}

\hypertarget{npcs}{%
\subsubsection*{NPCs}\label{npcs}}
\addcontentsline{toc}{subsubsection}{NPCs}

NPCs are non playable characters and in GTA V scripting they are called \texttt{Peds}. Peds are an entity like Props or Vehicles and can be created, assigned different model textures, equipped with weapons and controlled through different tasks.

\hypertarget{spawn-a-new-npc}{%
\subsubsection*{Spawn a new NPC}\label{spawn-a-new-npc}}
\addcontentsline{toc}{subsubsection}{Spawn a new NPC}

A GTA V Ped can be created by the \texttt{World.CreatePed} function. This takes two parameters: an ID to assign the 3D model and textures, and the location where the Ped is created.

The model IDs are the same we used in the previous tutorial, when we changed our character's appearance to a cat. A list of all available models can be found \href{https://wiki.gtanet.work/index.php/Peds}{here}. \texttt{PedHash.Cat}, \texttt{PedHash.Deer}, \texttt{PedHash.AviSchwartzman}are all possible IDs we can assign to the NPC we want to create.
We can create a new model variable, which we will name `myPedModel' and assign it a model ID:

\begin{Shaded}
\begin{Highlighting}[]
\NormalTok{Models myPedModel = }\StringTok{"PedHash.AviSchwartzman"}\NormalTok{;}
\end{Highlighting}
\end{Shaded}

The location where the NPC is created through a vector3 data type, which represents a vector in 3D space. This basically means a point that contains X, Y and Z coordinates. We can give absolute coordinates, making the Ped appear at a specific location in the game, but we can also use a location relative to our position in the game. In order not to risk making a Ped appear somewhere completely outside of our view -- on some mountain or in the sea -- let's look at a vector3 that points to a position in front of the player.

We want to establish the player with\texttt{Game.Player.Character}, followed by a function that retireve the player position within the game world. That's called by using \texttt{GetOffsetInWorldCoords}, which takes a vector3. The values of the X, Y and Z of the vector 3 offset the location based on the origin point represented by the player. Therefore, we can move the place where we want the Ped to appear by adding values to the X axis (left or right of player), Y axis (ahead or behind the player), and Z axis (above or below the player).
To make a Ped appear in front of the player we can create a vector3 data type with 0 for X, 5 for Y and 0 for Z: \texttt{new\ Vector3(0,\ 5,\ 0)}. Let's make a vector3 variable, which we will name `myPedSpawnPosition', assign it the values above for X, Y and Z coordinates from the player position.

\begin{Shaded}
\begin{Highlighting}[]
\NormalTok{Vector3 myPedSpawnPosition = Game.}\FunctionTok{Player}\NormalTok{.}\FunctionTok{Character}\NormalTok{.}\FunctionTok{GetOffsetInWorldCoords}\NormalTok{(}\KeywordTok{new} \FunctionTok{Vector3}\NormalTok{(}\DecValTok{0}\NormalTok{, }\DecValTok{5}\NormalTok{, }\DecValTok{0}\NormalTok{));}
\end{Highlighting}
\end{Shaded}

Now we can use the model and the position variables to spawn the NPC in front of the player. We'll create a Ped named `myPed1' and use the \texttt{World.CreatePed} function with the two variables as parameters:

\begin{Shaded}
\begin{Highlighting}[]
\DataTypeTok{var}\NormalTok{ myPed1 = World.}\FunctionTok{CreatePed}\NormalTok{(myPedModel, myPedSpawnPosition); }
\end{Highlighting}
\end{Shaded}

Example code

\begin{Shaded}
\begin{Highlighting}[]
\KeywordTok{using}\NormalTok{ System;}
\KeywordTok{using}\NormalTok{ System.}\FunctionTok{Collections}\NormalTok{.}\FunctionTok{Generic}\NormalTok{;}
\KeywordTok{using}\NormalTok{ System.}\FunctionTok{Linq}\NormalTok{;}
\KeywordTok{using}\NormalTok{ System.}\FunctionTok{Text}\NormalTok{;}
\KeywordTok{using}\NormalTok{ System.}\FunctionTok{Threading}\NormalTok{.}\FunctionTok{Tasks}\NormalTok{;}
 
\KeywordTok{using}\NormalTok{ GTA;}
\KeywordTok{using}\NormalTok{ GTA.}\FunctionTok{Math}\NormalTok{;}
\KeywordTok{using}\NormalTok{ System.}\FunctionTok{Windows}\NormalTok{.}\FunctionTok{Forms}\NormalTok{;}
\KeywordTok{using}\NormalTok{ System.}\FunctionTok{Drawing}\NormalTok{;}
\KeywordTok{using}\NormalTok{ GTA.}\FunctionTok{Native}\NormalTok{;}
 
 
\KeywordTok{namespace}\NormalTok{ moddingTutorial}
\NormalTok{\{}
    \KeywordTok{public} \KeywordTok{class}\NormalTok{ moddingTutorial : Script}
\NormalTok{    \{}
        \KeywordTok{public} \FunctionTok{moddingTutorial}\NormalTok{()}
\NormalTok{        \{}
            \KeywordTok{this}\NormalTok{.}\FunctionTok{Tick}\NormalTok{ += onTick;}
            \KeywordTok{this}\NormalTok{.}\FunctionTok{KeyUp}\NormalTok{ += onKeyUp;}
            \KeywordTok{this}\NormalTok{.}\FunctionTok{KeyDown}\NormalTok{ += onKeyDown;}
\NormalTok{        \}}
 
        \KeywordTok{private} \DataTypeTok{void} \FunctionTok{onTick}\NormalTok{(}\DataTypeTok{object}\NormalTok{ sender, EventArgs e) }\CommentTok{//this function gets executed continuously }
\NormalTok{        \{}
\NormalTok{        \}}
 
        \KeywordTok{private} \DataTypeTok{void} \FunctionTok{onKeyUp}\NormalTok{(}\DataTypeTok{object}\NormalTok{ sender, KeyEventArgs e)}\CommentTok{//everything inside here is executed only when we release a key}
\NormalTok{        \{}
\NormalTok{        \}}
 
        \KeywordTok{private} \DataTypeTok{void} \FunctionTok{onKeyDown}\NormalTok{(}\DataTypeTok{object}\NormalTok{ sender, KeyEventArgs e) }\CommentTok{//everything inside here is executed only when we press a key}
\NormalTok{        \{}
            \CommentTok{//when pressing 'K'}
            \KeywordTok{if}\NormalTok{(e.}\FunctionTok{KeyCode}\NormalTok{ == Keys.}\FunctionTok{K}\NormalTok{)}
\NormalTok{            \{}
                \CommentTok{//select a model and store it in a variable}
\NormalTok{                Models myPedModel = }\StringTok{"PedHash.AviSchwartzman"}\NormalTok{;}
        \CommentTok{//create a position relative to the player}
\NormalTok{        Vector3 myPedSpawnPosition = Game.}\FunctionTok{Player}\NormalTok{.}\FunctionTok{Character}\NormalTok{.}\FunctionTok{GetOffsetInWorldCoords}\NormalTok{(}\KeywordTok{new} \FunctionTok{Vector3}\NormalTok{(}\DecValTok{0}\NormalTok{, }\DecValTok{5}\NormalTok{, }\DecValTok{0}\NormalTok{));}
        \CommentTok{//create a Ped with the chosen model, spawning at the chosen position}
        \DataTypeTok{var}\NormalTok{ myPed1 = World.}\FunctionTok{CreatePed}\NormalTok{(myPedModel, myPedSpawnPosition); }
\NormalTok{            \}}
\NormalTok{        \}}
\NormalTok{    \}}
\NormalTok{\}}
\end{Highlighting}
\end{Shaded}

\hypertarget{give-tasks-to-npcs}{%
\subsubsection*{Give Tasks to NPCs}\label{give-tasks-to-npcs}}
\addcontentsline{toc}{subsubsection}{Give Tasks to NPCs}

A Ped can be given a task using the \texttt{Task} function, just like we did in the previous tutorial for the player character.

\begin{Shaded}
\begin{Highlighting}[]
\NormalTok{myPed1.}\FunctionTok{Task}\NormalTok{.}\FunctionTok{WanderAround}\NormalTok{();}
\end{Highlighting}
\end{Shaded}

Some tasks involve interacting with other characters (Peds or Game.Player.Character) or take different parameters like positions (vector3), duration (in milliseconds), and other data types.
We can give our NPC the task to fight against the player by using the \texttt{FightAgainst} function, which requires a Ped parameter -- which in the case of the player is expressed as \texttt{Game.Player.Character}.

\begin{Shaded}
\begin{Highlighting}[]
\NormalTok{myPed1.}\FunctionTok{Task}\NormalTok{.}\FunctionTok{FightAgainst}\NormalTok{(Game.}\FunctionTok{Player}\NormalTok{.}\FunctionTok{Character}\NormalTok{); }\CommentTok{//give npc task to fight against player}
\end{Highlighting}
\end{Shaded}

Try to replace the task to ``fight against'' with ``flee from (player)'' , ``hands up'', ``jump''\ldots{} or some of the other available tasks

See the \href{https://nitanmarcel.github.io/shvdn-docs.github.io/class_g_t_a_1_1_task_invoker.html}{TaskInvoker list} for possible tasks, or click on the list of available tasks below.

List of Available Tasks

\begin{Shaded}
\begin{Highlighting}[]
\DataTypeTok{void} \FunctionTok{AchieveHeading}\NormalTok{ (}\DataTypeTok{float}\NormalTok{ heading, }\DataTypeTok{int}\NormalTok{ timeout=}\DecValTok{0}\NormalTok{)}
 
\DataTypeTok{void} \FunctionTok{AimAt}\NormalTok{ (Entity target, }\DataTypeTok{int}\NormalTok{ duration)}
 
\DataTypeTok{void} \FunctionTok{AimAt}\NormalTok{ (Vector3 target, }\DataTypeTok{int}\NormalTok{ duration)}
 
\DataTypeTok{void} \FunctionTok{Arrest}\NormalTok{ (Ped ped)}
 
\DataTypeTok{void} \FunctionTok{ChatTo}\NormalTok{ (Ped ped)}
 
\DataTypeTok{void} \FunctionTok{Jump}\NormalTok{ ()}
 
\DataTypeTok{void} \FunctionTok{Climb}\NormalTok{ ()}
 
\DataTypeTok{void} \FunctionTok{ClimbLadder}\NormalTok{ ()}
 
\DataTypeTok{void} \FunctionTok{Cower}\NormalTok{ (}\DataTypeTok{int}\NormalTok{ duration)}
 
\DataTypeTok{void} \FunctionTok{ChaseWithGroundVehicle}\NormalTok{ (Ped target)}
 
\DataTypeTok{void} \FunctionTok{ChaseWithHelicopter}\NormalTok{ (Ped target, Vector3 offset)}
 
\DataTypeTok{void} \FunctionTok{ChaseWithPlane}\NormalTok{ (Ped target, Vector3 offset)}
 
\DataTypeTok{void} \FunctionTok{CruiseWithVehicle}\NormalTok{ (Vehicle vehicle, }\DataTypeTok{float}\NormalTok{ speed, DrivingStyle style=DrivingStyle.}\FunctionTok{Normal}\NormalTok{)}
 
\DataTypeTok{void} \FunctionTok{DriveTo}\NormalTok{ (Vehicle vehicle, Vector3 target, }\DataTypeTok{float}\NormalTok{ radius, }\DataTypeTok{float}\NormalTok{ speed, DrivingStyle style=DrivingStyle.}\FunctionTok{Normal}\NormalTok{)}
 
\DataTypeTok{void} \FunctionTok{EnterAnyVehicle}\NormalTok{ (VehicleSeat seat=VehicleSeat.}\FunctionTok{Any}\NormalTok{, }\DataTypeTok{int}\NormalTok{ timeout=-}\DecValTok{1}\NormalTok{, }\DataTypeTok{float}\NormalTok{ speed=1f, EnterVehicleFlags flag=EnterVehicleFlags.}\FunctionTok{None}\NormalTok{)}
 
\DataTypeTok{void} \FunctionTok{EnterVehicle}\NormalTok{ (Vehicle vehicle, VehicleSeat seat=VehicleSeat.}\FunctionTok{Any}\NormalTok{, }\DataTypeTok{int}\NormalTok{ timeout=-}\DecValTok{1}\NormalTok{, }\DataTypeTok{float}\NormalTok{ speed=1f, EnterVehicleFlags flag=EnterVehicleFlags.}\FunctionTok{None}\NormalTok{)}
 
\DataTypeTok{void} \FunctionTok{FightAgainst}\NormalTok{ (Ped target)}
 
\DataTypeTok{void} \FunctionTok{FightAgainst}\NormalTok{ (Ped target, }\DataTypeTok{int}\NormalTok{ duration)}
 
\DataTypeTok{void} \FunctionTok{FightAgainstHatedTargets}\NormalTok{ (}\DataTypeTok{float}\NormalTok{ radius)}
 
\DataTypeTok{void} \FunctionTok{FightAgainstHatedTargets}\NormalTok{ (}\DataTypeTok{float}\NormalTok{ radius, }\DataTypeTok{int}\NormalTok{ duration)}
 
\DataTypeTok{void} \FunctionTok{FleeFrom}\NormalTok{ (Ped ped, }\DataTypeTok{int}\NormalTok{ duration=-}\DecValTok{1}\NormalTok{)}
 
\DataTypeTok{void} \FunctionTok{FleeFrom}\NormalTok{ (Vector3 position, }\DataTypeTok{int}\NormalTok{ duration=-}\DecValTok{1}\NormalTok{)}
 
\DataTypeTok{void} \FunctionTok{FollowPointRoute}\NormalTok{ (}\KeywordTok{params}\NormalTok{ Vector3[] points)}
 
\DataTypeTok{void} \FunctionTok{FollowPointRoute}\NormalTok{ (}\DataTypeTok{float}\NormalTok{ movementSpeed, }\KeywordTok{params}\NormalTok{ Vector3[] points)}
 
\DataTypeTok{void} \FunctionTok{FollowToOffsetFromEntity}\NormalTok{ (Entity target, Vector3 offset, }\DataTypeTok{float}\NormalTok{ movementSpeed, }\DataTypeTok{int}\NormalTok{ timeout=-}\DecValTok{1}\NormalTok{, }\DataTypeTok{float}\NormalTok{ distanceToFollow=10f, }\DataTypeTok{bool}\NormalTok{ persistFollowing=}\KeywordTok{true}\NormalTok{)}
 
\DataTypeTok{void} \FunctionTok{GoTo}\NormalTok{ (Entity target, Vector3 offset=}\KeywordTok{default}\NormalTok{(Vector3), }\DataTypeTok{int}\NormalTok{ timeout=-}\DecValTok{1}\NormalTok{)}
 
\DataTypeTok{void} \FunctionTok{GoTo}\NormalTok{ (Vector3 position, }\DataTypeTok{int}\NormalTok{ timeout=-}\DecValTok{1}\NormalTok{)}
 
\DataTypeTok{void} \FunctionTok{GoStraightTo}\NormalTok{ (Vector3 position, }\DataTypeTok{int}\NormalTok{ timeout=-}\DecValTok{1}\NormalTok{, }\DataTypeTok{float}\NormalTok{ targetHeading=0f, }\DataTypeTok{float}\NormalTok{ distanceToSlide=0f)}
 
\DataTypeTok{void} \FunctionTok{GuardCurrentPosition}\NormalTok{ ()}
 
\DataTypeTok{void} \FunctionTok{HandsUp}\NormalTok{ (}\DataTypeTok{int}\NormalTok{ duration)}
 
\DataTypeTok{void} \FunctionTok{LandPlane}\NormalTok{ (Vector3 startPosition, Vector3 touchdownPosition, Vehicle plane=}\KeywordTok{null}\NormalTok{)}
 
\DataTypeTok{void} \FunctionTok{LeaveVehicle}\NormalTok{ (LeaveVehicleFlags flags=LeaveVehicleFlags.}\FunctionTok{None}\NormalTok{)}
 
\DataTypeTok{void} \FunctionTok{LeaveVehicle}\NormalTok{ (Vehicle vehicle, }\DataTypeTok{bool}\NormalTok{ closeDoor)}
 
\DataTypeTok{void} \FunctionTok{LeaveVehicle}\NormalTok{ (Vehicle vehicle, LeaveVehicleFlags flags)}
 
\DataTypeTok{void} \FunctionTok{LookAt}\NormalTok{ (Entity target, }\DataTypeTok{int}\NormalTok{ duration=-}\DecValTok{1}\NormalTok{)}

\DataTypeTok{void} \FunctionTok{LookAt}\NormalTok{ (Vector3 position, }\DataTypeTok{int}\NormalTok{ duration=-}\DecValTok{1}\NormalTok{)}
  
\DataTypeTok{void} \FunctionTok{ParachuteTo}\NormalTok{ (Vector3 position)}
 
\DataTypeTok{void} \FunctionTok{ParkVehicle}\NormalTok{ (Vehicle vehicle, Vector3 position, }\DataTypeTok{float}\NormalTok{ heading, }\DataTypeTok{float}\NormalTok{ radius=}\FloatTok{20.0f}\NormalTok{, }\DataTypeTok{bool}\NormalTok{ keepEngineOn=}\KeywordTok{false}\NormalTok{)}
 
\DataTypeTok{void} \FunctionTok{PerformSequence}\NormalTok{ (TaskSequence sequence)}
 
\DataTypeTok{void} \FunctionTok{PlayAnimation}\NormalTok{ (}\DataTypeTok{string}\NormalTok{ animDict, }\DataTypeTok{string}\NormalTok{ animName)}
 
\DataTypeTok{void} \FunctionTok{PlayAnimation}\NormalTok{ (}\DataTypeTok{string}\NormalTok{ animDict, }\DataTypeTok{string}\NormalTok{ animName, }\DataTypeTok{float}\NormalTok{ speed, }\DataTypeTok{int}\NormalTok{ duration, }\DataTypeTok{float}\NormalTok{ playbackRate)}
 
\DataTypeTok{void} \FunctionTok{PlayAnimation}\NormalTok{ (}\DataTypeTok{string}\NormalTok{ animDict, }\DataTypeTok{string}\NormalTok{ animName, }\DataTypeTok{float}\NormalTok{ blendInSpeed, }\DataTypeTok{int}\NormalTok{ duration, AnimationFlags flags)}
 
\DataTypeTok{void} \FunctionTok{PlayAnimation}\NormalTok{ (}\DataTypeTok{string}\NormalTok{ animDict, }\DataTypeTok{string}\NormalTok{ animName, }\DataTypeTok{float}\NormalTok{ blendInSpeed, }\DataTypeTok{float}\NormalTok{ blendOutSpeed, }\DataTypeTok{int}\NormalTok{ duration, AnimationFlags flags, }\DataTypeTok{float}\NormalTok{ playbackRate)}
 
\DataTypeTok{void} \FunctionTok{RappelFromHelicopter}\NormalTok{ ()}
 
\DataTypeTok{void} \FunctionTok{ReactAndFlee}\NormalTok{ (Ped ped)}
 
\DataTypeTok{void} \FunctionTok{ReloadWeapon}\NormalTok{ ()}
 
\DataTypeTok{void} \FunctionTok{RunTo}\NormalTok{ (Vector3 position, }\DataTypeTok{bool}\NormalTok{ ignorePaths=}\KeywordTok{false}\NormalTok{, }\DataTypeTok{int}\NormalTok{ timeout=-}\DecValTok{1}\NormalTok{)}
 
\DataTypeTok{void} \FunctionTok{ShootAt}\NormalTok{ (Ped target, }\DataTypeTok{int}\NormalTok{ duration=-}\DecValTok{1}\NormalTok{, FiringPattern pattern=FiringPattern.}\FunctionTok{Default}\NormalTok{)}
 
\DataTypeTok{void} \FunctionTok{ShootAt}\NormalTok{ (Vector3 position, }\DataTypeTok{int}\NormalTok{ duration=-}\DecValTok{1}\NormalTok{, FiringPattern pattern=FiringPattern.}\FunctionTok{Default}\NormalTok{)}
 
\DataTypeTok{void} \FunctionTok{ShuffleToNextVehicleSeat}\NormalTok{ (Vehicle vehicle=}\KeywordTok{null}\NormalTok{)}
 
\DataTypeTok{void} \FunctionTok{Skydive}\NormalTok{ ()}
 
\DataTypeTok{void} \FunctionTok{SlideTo}\NormalTok{ (Vector3 position, }\DataTypeTok{float}\NormalTok{ heading)}
 
\DataTypeTok{void} \FunctionTok{StandStill}\NormalTok{ (}\DataTypeTok{int}\NormalTok{ duration)}
 
\DataTypeTok{void} \FunctionTok{StartScenario}\NormalTok{ (}\DataTypeTok{string}\NormalTok{ name, }\DataTypeTok{float}\NormalTok{ heading)}
 
\DataTypeTok{void} \FunctionTok{StartScenario}\NormalTok{ (}\DataTypeTok{string}\NormalTok{ name, Vector3 position, }\DataTypeTok{float}\NormalTok{ heading)}
 
\DataTypeTok{void} \FunctionTok{SwapWeapon}\NormalTok{ ()}
 
\DataTypeTok{void} \FunctionTok{TurnTo}\NormalTok{ (Entity target, }\DataTypeTok{int}\NormalTok{ duration=-}\DecValTok{1}\NormalTok{)}
 
\DataTypeTok{void} \FunctionTok{TurnTo}\NormalTok{ (Vector3 position, }\DataTypeTok{int}\NormalTok{ duration=-}\DecValTok{1}\NormalTok{)}
 
\DataTypeTok{void} \FunctionTok{UseParachute}\NormalTok{ ()}
 
\DataTypeTok{void} \FunctionTok{UseMobilePhone}\NormalTok{ ()}
 
\DataTypeTok{void} \FunctionTok{UseMobilePhone}\NormalTok{ (}\DataTypeTok{int}\NormalTok{ duration)}
 
\DataTypeTok{void} \FunctionTok{PutAwayParachute}\NormalTok{ ()}
 
\DataTypeTok{void} \FunctionTok{PutAwayMobilePhone}\NormalTok{ ()}
 
\DataTypeTok{void} \FunctionTok{VehicleChase}\NormalTok{ (Ped target)}
 
\DataTypeTok{void} \FunctionTok{VehicleShootAtPed}\NormalTok{ (Ped target)}
 
\DataTypeTok{void} \FunctionTok{Wait}\NormalTok{ (}\DataTypeTok{int}\NormalTok{ duration)}
 
\DataTypeTok{void} \FunctionTok{WanderAround}\NormalTok{ ()}
 
\DataTypeTok{void} \FunctionTok{WanderAround}\NormalTok{ (Vector3 position, }\DataTypeTok{float}\NormalTok{ radius)}
 
\DataTypeTok{void} \FunctionTok{WarpIntoVehicle}\NormalTok{ (Vehicle vehicle, VehicleSeat seat)}
 
\DataTypeTok{void} \FunctionTok{WarpOutOfVehicle}\NormalTok{ (Vehicle vehicle)}
 
\DataTypeTok{void} \FunctionTok{ClearAll}\NormalTok{ ()}
 
\DataTypeTok{void} \FunctionTok{ClearAllImmediately}\NormalTok{ ()}
 
\DataTypeTok{void} \FunctionTok{ClearLookAt}\NormalTok{ ()}
 
\DataTypeTok{void} \FunctionTok{ClearSecondary}\NormalTok{ ()}
 
\DataTypeTok{void} \FunctionTok{ClearAnimation}\NormalTok{ (}\DataTypeTok{string}\NormalTok{ animSet, }\DataTypeTok{string}\NormalTok{ animName)}
\end{Highlighting}
\end{Shaded}

\hypertarget{teleporting}{%
\subsubsection*{Teleporting}\label{teleporting}}
\addcontentsline{toc}{subsubsection}{Teleporting}

We can change the location of the player character or of any Ped or Vehicle entity by using the native function \texttt{SET\_ENTITY\_COORDS}. This function needs an entity and X, Y and Z coordinate to teleport to.
We need to know the exact coordinates of the locations we want to teleport to, but thankfully the modding community forums provide lists with all available coordinates we can teleport to. Let's take the XYZ coordinates of the top of Mount Chiliad (the highest point in the game) to teleport our player character to.

\begin{verbatim}
LOCATION: Top of the Mt Chilad
COORDINATES: X:450.718 Y:5566.614 Z:806.183
\end{verbatim}

To create a teleport function we will use a native function. Script Hook V Dot Net is a wrapper for the C++ ScriptHook, calling the functions in Scripthook to do things in the game. However, there are some functions that are not in Script Hook V Dot Net and in order to use these, we have to use the native calling from Script Hook.

\href{https://nitanmarcel.github.io/shvdn-docs.github.io/namespace_g_t_a_1_1_native.html\#a84977424e1cb7b6f1c2902770bf9ad2d}{Native functions} are called with \texttt{Function.Call} followed by their corresponding hash name and parameters. They use this structure:

\begin{Shaded}
\begin{Highlighting}[]
\NormalTok{Function.}\FunctionTok{Call}\NormalTok{(Hash.}\FunctionTok{HASH_NAME}\NormalTok{, input_params);}
\end{Highlighting}
\end{Shaded}

The native function for teleporting expects the hash \texttt{SET\_ENTITY\_COORDS}, the \texttt{ped} entity to teleport, and the X, Y and Z coordinates to teleport the character to. \texttt{Function.Call(Hash.SET\_ENTITY\_COORDS,\ Ped\ ped,\ X,\ Y,\ Z,\ 0,\ 0,\ 1);}

The function to teleport the player character to the top of Moutn Chiliad is:

\begin{Shaded}
\begin{Highlighting}[]
\CommentTok{//Teleport to the top of Mount Chiliad}
\NormalTok{Function.}\FunctionTok{Call}\NormalTok{(Hash.}\FunctionTok{SET_ENTITY_COORDS}\NormalTok{, Game.}\FunctionTok{Player}\NormalTok{.}\FunctionTok{Character}\NormalTok{, }\FloatTok{450.718f}\NormalTok{, }\FloatTok{5566.614f}\NormalTok{, }\FloatTok{806.183f}\NormalTok{, }\DecValTok{0}\NormalTok{, }\DecValTok{0}\NormalTok{, }\DecValTok{1}\NormalTok{);}
\end{Highlighting}
\end{Shaded}

See this \href{https://gtaforums.com/topic/792877-list-of-over-100-coordinates-more-comming/}{list of locations} to find their respective coordinates or click on the list below

List of Locations with Coordinates

\begin{verbatim}
INDOOR LOCATIONS
 
Strip Club DJ Booth X:126.135 Y:-1278.583 Z:29.270

Blaine County Savings Bank X:-109.299 Y:6464.035 Z:31.627

Police Station X:436.491 Y: -982.172 Z:30.699

Humane Labs Entrance X:3619.749 Y:2742.740 Z:28.690

Burnt FIB Building X:160.868 Y:-745.831 Z:250.063

10 Car Garage Back Room X:223.193 Y:-967.322 Z:99.000

Humane Labs Tunnel X:3525.495 Y:3705.301 Z:20.992

Ammunation Office X:12.494 Y:-1110.130 Z: 29.797

Ammunation Gun Range X: 22.153 Y:-1072.854 Z:29.797

Trevor's Meth Lab X:1391.773 Y:3608.716 Z:38.942

Pacific Standard Bank Vault X:255.851 Y: 217.030 Z:101.683

Lester's House X:1273.898 Y:-1719.304 Z:54.771

Floyd's Apartment X:-1150.703 Y:-1520.713 Z:10.633

FIB Top Floor X:135.733 Y:-749.216 Z:258.152

IAA Office X:117.220 Y:-620.938 Z:206.047

Pacific Standard Bank X:235.046 Y:216.434 Z:106.287

Fort Zancudo ATC entrance X:-2344.373 Y:3267.498 Z:32.811

Fort Zancudo ATC top floor X:-2358.132 Y:3249.754 Z:101.451

Torture Room X: 147.170 Y:-2201.804 Z:4.688

 
OUTDOOR LOCATIONS
 
Main LS Customs X:-365.425 Y:-131.809 Z:37.873

Very High Up X:-129.964 Y:8130.873 Z:6705.307

IAA Roof X:134.085 Y:-637.859 Z:262.851

FIB Roof X:150.126 Y:-754.591 Z:262.865

Maze Bank Roof X:-75.015 Y:-818.215 Z:326.176

Top of the Mt Chilad X:450.718 Y:5566.614 Z:806.183

Most Northerly Point X:24.775 Y:7644.102 Z:19.055

Vinewood Bowl Stage X:686.245 Y:577.950 Z:130.461

Sisyphus Theater Stage X:205.316 Y:1167.378 Z:227.005

Galileo Observatory Roof X:-438.804 Y:1076.097 Z:352.411

Kortz Center X:-2243.810 Y:264.048 Z:174.615

Chumash Historic Family Pier X:-3426.683 Y:967.738 Z:8.347

Paleto Bay Pier X:-275.522 Y:6635.835 Z:7.425

God's thumb X:-1006.402 Y:6272.383 Z:1.503

Calafia Train Bridge X:-517.869 Y:4425.284 Z:89.795

Altruist Cult Camp X:-1170.841 Y:4926.646 Z:224.295

Maze Bank Arena Roof X:-324.300 Y:-1968.545 Z:67.002

Marlowe Vineyards X:-1868.971 Y:2095.674 Z:139.115

Hippy Camp X:2476.712 Y:3789.645 Z:41.226

Devin Weston's House X:-2639.872 Y:1866.812 Z:160.135

Abandon Mine X:-595.342 Y: 2086.008 Z:131.412

Weed Farm X:2208.777 Y:5578.235 Z:53.735

Stab City X: 126.975 Y:3714.419 Z:46.827

Airplane Graveyard Airplane Tail X:2395.096 Y:3049.616 Z:60.053

Satellite Dish Antenna X:2034.988 Y:2953.105 Z:74.602

Satellite Dishes X: 2062.123 Y:2942.055 Z:47.431

Windmill Top X:2026.677 Y:1842.684 Z:133.313

Sandy Shores Building Site Crane X:1051.209 Y:2280.452 Z:89.727

Rebel Radio X:736.153 Y:2583.143 Z:79.634

Quarry X:2954.196 Y:2783.410 Z:41.004

Palmer-Taylor Power Station Chimney X: 2732.931 Y: 1577.540 Z:83.671

Merryweather Dock X: 486.417 Y:-3339.692 Z:6.070

Cargo Ship X:899.678 Y:-2882.191 Z:19.013

Del Perro Pier X:-1850.127 Y:-1231.751 Z:13.017

Play Boy Mansion X:-1475.234 Y:167.088Z:55.841

Jolene Cranley-Evans Ghost X:3059.620 Y:5564.246 Z:197.091

NOOSE Headquarters X:2535.243 Y:-383.799 Z:92.993

Snowman X: 971.245 Y:-1620.993 Z:30.111

Oriental Theater X:293.089 Y:180.466 Z:104.301

Beach Skatepark X:-1374.881 Y:-1398.835 Z:6.141

Underpass Skatepark X:718.341 Y:-1218.714 Z: 26.014

Casino X:925.329 Y:46.152 Z:80.908

University of San Andreas X:-1696.866 Y:142.747 Z:64.372

La Puerta Freeway Bridge X: -543.932 Y:-2225.543 Z:122.366

Land Act Dam X: 1660.369 Y:-12.013 Z:170.020

Mount Gordo X: 2877.633 Y:5911.078 Z:369.624

Little Seoul X:-889.655 Y:-853.499 Z:20.566

Epsilon Building X:-695.025 Y:82.955 Z:55.855 Z:55.855

The Richman Hotel X:-1330.911 Y:340.871 Z:64.078

Vinewood sign X:711.362 Y:1198.134 Z:348.526

Los Santos Golf Club X:-1336.715 Y:59.051 Z:55.246

Chicken X:-31.010 Y:6316.830 Z:40.083

Little Portola X:-635.463 Y:-242.402 Z:38.175

Pacific Bluffs Country Club X:-3022.222 Y:39.968 Z:13.611

Vinewood Cemetery X:-1659993 Y:-128.399 Z:59.954

Paleto Forest Sawmill Chimney X:-549.467 Y:5308.221 Z:114.146

Mirror Park X:1070.206 Y:-711.958 Z:58.483

Rocket X:1608.698 Y:6438.096 Z:37.637

El Gordo Lighthouse X:3430.155 Y:5174.196 Z:41.280
\end{verbatim}

\hypertarget{content-replication-assignment-4}{%
\section*{Content Replication Assignment}\label{content-replication-assignment-4}}
\addcontentsline{toc}{section}{Content Replication Assignment}

Teleport the player to a beach, spawn ten whales on the shore and generate an NPC wandering aroud them and take a screenshot in the style of HAPP V2.

\end{document}
