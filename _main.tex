% Options for packages loaded elsewhere
\PassOptionsToPackage{unicode}{hyperref}
\PassOptionsToPackage{hyphens}{url}
%
\documentclass[
  openany]{book}
\usepackage{lmodern}
\usepackage{amssymb,amsmath}
\usepackage{ifxetex,ifluatex}
\ifnum 0\ifxetex 1\fi\ifluatex 1\fi=0 % if pdftex
  \usepackage[T1]{fontenc}
  \usepackage[utf8]{inputenc}
  \usepackage{textcomp} % provide euro and other symbols
\else % if luatex or xetex
  \usepackage{unicode-math}
  \defaultfontfeatures{Scale=MatchLowercase}
  \defaultfontfeatures[\rmfamily]{Ligatures=TeX,Scale=1}
\fi
% Use upquote if available, for straight quotes in verbatim environments
\IfFileExists{upquote.sty}{\usepackage{upquote}}{}
\IfFileExists{microtype.sty}{% use microtype if available
  \usepackage[]{microtype}
  \UseMicrotypeSet[protrusion]{basicmath} % disable protrusion for tt fonts
}{}
\makeatletter
\@ifundefined{KOMAClassName}{% if non-KOMA class
  \IfFileExists{parskip.sty}{%
    \usepackage{parskip}
  }{% else
    \setlength{\parindent}{0pt}
    \setlength{\parskip}{6pt plus 2pt minus 1pt}}
}{% if KOMA class
  \KOMAoptions{parskip=half}}
\makeatother
\usepackage{xcolor}
\IfFileExists{xurl.sty}{\usepackage{xurl}}{} % add URL line breaks if available
\IfFileExists{bookmark.sty}{\usepackage{bookmark}}{\usepackage{hyperref}}
\hypersetup{
  pdftitle={The Photographer's Guide to Los Santos},
  hidelinks,
  pdfcreator={LaTeX via pandoc}}
\urlstyle{same} % disable monospaced font for URLs
\usepackage{color}
\usepackage{fancyvrb}
\newcommand{\VerbBar}{|}
\newcommand{\VERB}{\Verb[commandchars=\\\{\}]}
\DefineVerbatimEnvironment{Highlighting}{Verbatim}{commandchars=\\\{\}}
% Add ',fontsize=\small' for more characters per line
\usepackage{framed}
\definecolor{shadecolor}{RGB}{248,248,248}
\newenvironment{Shaded}{\begin{snugshade}}{\end{snugshade}}
\newcommand{\AlertTok}[1]{\textcolor[rgb]{0.94,0.16,0.16}{#1}}
\newcommand{\AnnotationTok}[1]{\textcolor[rgb]{0.56,0.35,0.01}{\textbf{\textit{#1}}}}
\newcommand{\AttributeTok}[1]{\textcolor[rgb]{0.77,0.63,0.00}{#1}}
\newcommand{\BaseNTok}[1]{\textcolor[rgb]{0.00,0.00,0.81}{#1}}
\newcommand{\BuiltInTok}[1]{#1}
\newcommand{\CharTok}[1]{\textcolor[rgb]{0.31,0.60,0.02}{#1}}
\newcommand{\CommentTok}[1]{\textcolor[rgb]{0.56,0.35,0.01}{\textit{#1}}}
\newcommand{\CommentVarTok}[1]{\textcolor[rgb]{0.56,0.35,0.01}{\textbf{\textit{#1}}}}
\newcommand{\ConstantTok}[1]{\textcolor[rgb]{0.00,0.00,0.00}{#1}}
\newcommand{\ControlFlowTok}[1]{\textcolor[rgb]{0.13,0.29,0.53}{\textbf{#1}}}
\newcommand{\DataTypeTok}[1]{\textcolor[rgb]{0.13,0.29,0.53}{#1}}
\newcommand{\DecValTok}[1]{\textcolor[rgb]{0.00,0.00,0.81}{#1}}
\newcommand{\DocumentationTok}[1]{\textcolor[rgb]{0.56,0.35,0.01}{\textbf{\textit{#1}}}}
\newcommand{\ErrorTok}[1]{\textcolor[rgb]{0.64,0.00,0.00}{\textbf{#1}}}
\newcommand{\ExtensionTok}[1]{#1}
\newcommand{\FloatTok}[1]{\textcolor[rgb]{0.00,0.00,0.81}{#1}}
\newcommand{\FunctionTok}[1]{\textcolor[rgb]{0.00,0.00,0.00}{#1}}
\newcommand{\ImportTok}[1]{#1}
\newcommand{\InformationTok}[1]{\textcolor[rgb]{0.56,0.35,0.01}{\textbf{\textit{#1}}}}
\newcommand{\KeywordTok}[1]{\textcolor[rgb]{0.13,0.29,0.53}{\textbf{#1}}}
\newcommand{\NormalTok}[1]{#1}
\newcommand{\OperatorTok}[1]{\textcolor[rgb]{0.81,0.36,0.00}{\textbf{#1}}}
\newcommand{\OtherTok}[1]{\textcolor[rgb]{0.56,0.35,0.01}{#1}}
\newcommand{\PreprocessorTok}[1]{\textcolor[rgb]{0.56,0.35,0.01}{\textit{#1}}}
\newcommand{\RegionMarkerTok}[1]{#1}
\newcommand{\SpecialCharTok}[1]{\textcolor[rgb]{0.00,0.00,0.00}{#1}}
\newcommand{\SpecialStringTok}[1]{\textcolor[rgb]{0.31,0.60,0.02}{#1}}
\newcommand{\StringTok}[1]{\textcolor[rgb]{0.31,0.60,0.02}{#1}}
\newcommand{\VariableTok}[1]{\textcolor[rgb]{0.00,0.00,0.00}{#1}}
\newcommand{\VerbatimStringTok}[1]{\textcolor[rgb]{0.31,0.60,0.02}{#1}}
\newcommand{\WarningTok}[1]{\textcolor[rgb]{0.56,0.35,0.01}{\textbf{\textit{#1}}}}
\usepackage{longtable,booktabs}
% Correct order of tables after \paragraph or \subparagraph
\usepackage{etoolbox}
\makeatletter
\patchcmd\longtable{\par}{\if@noskipsec\mbox{}\fi\par}{}{}
\makeatother
% Allow footnotes in longtable head/foot
\IfFileExists{footnotehyper.sty}{\usepackage{footnotehyper}}{\usepackage{footnote}}
\makesavenoteenv{longtable}
\usepackage{graphicx,grffile}
\makeatletter
\def\maxwidth{\ifdim\Gin@nat@width>\linewidth\linewidth\else\Gin@nat@width\fi}
\def\maxheight{\ifdim\Gin@nat@height>\textheight\textheight\else\Gin@nat@height\fi}
\makeatother
% Scale images if necessary, so that they will not overflow the page
% margins by default, and it is still possible to overwrite the defaults
% using explicit options in \includegraphics[width, height, ...]{}
\setkeys{Gin}{width=\maxwidth,height=\maxheight,keepaspectratio}
% Set default figure placement to htbp
\makeatletter
\def\fps@figure{htbp}
\makeatother
\setlength{\emergencystretch}{3em} % prevent overfull lines
\providecommand{\tightlist}{%
  \setlength{\itemsep}{0pt}\setlength{\parskip}{0pt}}
\setcounter{secnumdepth}{5}

\title{The Photographer's Guide to Los Santos}
\author{}
\date{\vspace{-2.5em}}

\begin{document}
\maketitle

{
\setcounter{tocdepth}{1}
\tableofcontents
}
\hypertarget{preface}{%
\chapter*{Preface}\label{preface}}
\addcontentsline{toc}{chapter}{Preface}

\begin{quote}
``Los Santos. The city of shitheads. Where else would he be?''

--- Trevor Philips
\end{quote}

\hypertarget{introduction}{%
\chapter{Introduction}\label{introduction}}

\hypertarget{about-the-photographers-guide-to-los-santos}{%
\section{About The Photographer's Guide to Los Santos}\label{about-the-photographers-guide-to-los-santos}}

The Photographer's Guide to Los Santos sits between a touristic guide and a photography manual, and between an exhibition catalogue and a peak behind the scenes of artwork creation.

The Photographer's Guide to Los Santos is an ongoing project that builds on top of a research on artistic practices within spaces of computer games, with a particular focus on in-game photography, machinima and digital visual arts. It follows some themes and ideas previously explored in the exhibition \href{https://www.howtowinat.photography/}{\emph{How to Win at Photography}}, while focusing more specifically on the relationship between computer games and photographic activities inside the world of Grand Theft Auto V.

The idea of a guide refers to in-game photography as a form of `virtual tourism' (\href{https://papers.ssrn.com/sol3/papers.cfm?abstract_id=538182}{Book, 2003}), which was also the premise of an actual tourist guide published by Rough Guides in their 2019 \href{https://www.roughguides.com/articles/introduction-to-the-rough-guide-to-xbox/}{\emph{Rough Guide to XBOX}}. Yet this guide project also understands the game world as a site for image production and artistic creation, turning the game into a destination for a `game art tourist'. The Photographer's Guide to Los Santos presents the game environment of Grand Theft Auto V both as a space to explore and in which to create images, as well as a place to navigate and learn about some of the most important artworks that it has enabled to create.

The project also brings together several experiences from teaching in-game photography as an artistic practice in different educational settings and institutions, compiling materials and tools for students and artists interested in engaging with the field. The tourist guide of the game world doubles as a photography manual for the in-game photography age, featuring tutorials and excercises ranging from game screenshotting to computer programming for creative modding. Through the practical excercises, the project invites to rethink the game object as a space for creative, subversive and critical endeavours, which can be played differently, documented, reclaimed or modified through an artistic approach.

Finally, the project draws inspiration from the works of artists who have explored the `metaplay' of photographing game words instead of following the game rules and attempt to reach the goal of winning. The Photographer's Guide to Los Santos is indebted to all the artists it features, but was particularly inspired by Gareth Damian Martin's live streamed workshop \href{https://www.twitch.tv/videos/591840067}{\emph{Photography Tour of No Man's Sky}} (realized for \href{https://www.somersethouse.org.uk/whats-on/now-play-this-2020}{Now Play This Festival 2020}), Total Refusal \& Ismaël Joffroy Chandoutis's 2021 in-game lecture performance and guided tour \href{https://vimeo.com/506064357}{\emph{Everyday Daylight}} (realized for the \href{https://ccsparis.com/en/events/total-refusal-digital-disarmament-movement-a-la-gaite-lyrique/}{CCS Paris}), and Alan Butler's epic 2020 live endurance performance \href{https://www.youtube.com/watch?v=R4Q2G6tOQ_Q}{\emph{Witness to a Changing West}} (realized for \href{https://screenwalks.com/}{Screen Walks}) and his `Content Replication Assignments'.

\hypertarget{grand-theft-auto-v-studies}{%
\section{Grand Theft Auto V Studies}\label{grand-theft-auto-v-studies}}

Los Santos is the Grand Theft Auto V's fictional, parodic version of real-life Los Angeles. Just like Los Angeles is the global centre of film and commercial media production, Los Santos is the epicentre of in-game photography and machinima creation. While it may seem reductive to only focus on a single game to address the larger phenomenon of in-game photography, GTA V is the biggest source of creative outputs to date, with its extended open world and one of the largest community of active modders. Launched in 2013, the game contains a world map of more than 80 square kilometers of total area, which includes the urban area of the city of Los Santos and the rural area of Blaine County. This incredibly vast environment features a large desert region, dense forest, several mountains, beachside towns, on top of the large metropolis of Los Santos. The game simulates the everyday life of hundreds of individual NPCs (while it allegedly counts a population of over 4 million) as well as counting 28 animal species, and \href{https://grandtheftdata.com/landmarks/\#0,0,2,satellite}{more than 800 buildings in GTA V are based on real-life landmarks}. The size of the photorealistic simulation is only matched by the complexity of the game engine and its code, which - thanks to the effort of GTA V's modding community - allows players to use the game world as a powerful tool to create new scenes, take controls of its algorithmic entities, modify cameras and reshaping the game into a movie set or a photo studio.

\href{https://www.bragitoff.com/2015/11/gta-v-maps-quad-ultra-high-definition-8k-quality/gtav_satellite_2048x2048/}{GTA V Satellite Map by Manas Sharma}

\hypertarget{grand-theft-auto-v-tourism}{%
\section{Grand Theft Auto V Tourism}\label{grand-theft-auto-v-tourism}}

This guide allows players to explore the game environment following some of the most interesting artworks that have been created with(in) it. The guide is divided in thematic chapters that follow different artistic practices, taking place in different locations of the game environment, followed by different tutorials and exercises connected with the works and the space analyzed.
Each selected work is presented by a curatorial statement, introducing the work and its artistic relevance. The work is accompanied by information on the in-game location from which it was produced, inviting the readers to reach the destination in Grand Theft Auto V through maps and indications.

\hypertarget{grand-theft-auto-v-art-education}{%
\section{Grand Theft Auto V Art Education}\label{grand-theft-auto-v-art-education}}

Each thematic chapter features a tutorial section that introduces different techniques and strategies to capture images within Grand Theft Auto V. The chapters are thought to be experienced in order, as the tutorials at times rely on knowledge that is built on top of previous lessons. Each tutorial is accompanied by content replication assignments, in which the readers is invited to use the skills learnd from each chapter to recreate a work presented in that section.

\hypertarget{architecture-photography}{%
\chapter{Architecture Photography}\label{architecture-photography}}

\hypertarget{from-the-continuous-city-by-gareth-damian-martin}{%
\section{\texorpdfstring{from \emph{The Continuous City}, by Gareth Damian Martin}{from The Continuous City, by Gareth Damian Martin}}\label{from-the-continuous-city-by-gareth-damian-martin}}

Gareth Damian Martin, \emph{Outskirts}, from \emph{The Continuous City},

Gareth Damian Martin, \emph{Pathways}, from \emph{The Continuous City},

artwork text

\href{https://socks-studio.com/2019/10/13/gareth-damian-martin-postcards-from-the-continuous-city-2018/}{More about \emph{The Continuous City}}
\href{https://www.gamescenes.org/2018/04/interview-gareth-damian-martin-the-aesthetics-of-analogue-game-photography.html}{Interview with Gareth Damian Martin}

\hypertarget{getting-there}{%
\subsection{Getting there}\label{getting-there}}

The intersection of Interstate 4 and Interstate 5.

\hypertarget{readings}{%
\section{Readings}\label{readings}}

\href{https://www.heterotopiaszine.com/}{Heterotopias}

Mark D Teo, The Urban Architecture of Los Angeles and Grand Theft Auto, 2015. \url{https://www.academia.edu/18173221/The_Urban_Architecture_of_Los_Angeles_and_Grand_Theft_Auto}

\hypertarget{tutorial}{%
\section{Tutorial}\label{tutorial}}

\hypertarget{photographing-the-game-screen}{%
\subsection{Photographing the Game Screen}\label{photographing-the-game-screen}}

\hypertarget{analogue-game-photography}{%
\subsubsection{Analogue Game Photography}\label{analogue-game-photography}}

\hypertarget{screenshotting}{%
\subsubsection{Screenshotting}\label{screenshotting}}

\hypertarget{content-replication-assignment}{%
\section{Content Replication Assignment}\label{content-replication-assignment}}

\hypertarget{social-documentary}{%
\chapter{Social Documentary}\label{social-documentary}}

\hypertarget{down-and-out-in-los-santos-by-alan-butler}{%
\section{\texorpdfstring{\emph{Down and Out in Los Santos} by Alan Butler}{Down and Out in Los Santos by Alan Butler}}\label{down-and-out-in-los-santos-by-alan-butler}}

artwork text

\href{http://www.alanbutler.info/down-and-out-in-los-santos-2016}{More about \emph{Down and Out in Los Santos}}

\hypertarget{getting-there-1}{%
\subsection{Getting There}\label{getting-there-1}}

\hypertarget{fear-and-loathing-in-gta-v-by-morten-rockford-ravn}{%
\section{\texorpdfstring{\emph{Fear and Loathing in GTA V} by Morten Rockford Ravn}{Fear and Loathing in GTA V by Morten Rockford Ravn}}\label{fear-and-loathing-in-gta-v-by-morten-rockford-ravn}}

artwork text

\href{https://fearandloathingingtav.tumblr.com/}{More about \emph{Fear and Loathing in GTA V}}

\hypertarget{getting-there-2}{%
\subsection{Getting There}\label{getting-there-2}}

\hypertarget{readings-1}{%
\section{Readings}\label{readings-1}}

\hypertarget{tutorial-1}{%
\section{Tutorial}\label{tutorial-1}}

\hypertarget{in-game-smartphone-camera}{%
\subsection{In-game Smartphone Camera}\label{in-game-smartphone-camera}}

\hypertarget{content-replication-assignment-1}{%
\section{Content Replication Assignment}\label{content-replication-assignment-1}}

\hypertarget{re-enactment-photography}{%
\chapter{Re-enactment photography}\label{re-enactment-photography}}

\hypertarget{a-study-on-perspective-by-roc-herms}{%
\section{\texorpdfstring{\emph{A Study on Perspective} by Roc Herms}{A Study on Perspective by Roc Herms}}\label{a-study-on-perspective-by-roc-herms}}

artwork text

\hypertarget{little-books-of-los-santos-by-luke-caspar-pearson}{%
\section{\texorpdfstring{\emph{Little Books of Los Santos} by Luke Caspar Pearson}{Little Books of Los Santos by Luke Caspar Pearson}}\label{little-books-of-los-santos-by-luke-caspar-pearson}}

artwork text

\hypertarget{nine-swimming-pools-and-a-broken-glass-by-alan-butler}{%
\section{\texorpdfstring{\emph{Nine swimming pools and a broken glass} by Alan Butler}{Nine swimming pools and a broken glass by Alan Butler}}\label{nine-swimming-pools-and-a-broken-glass-by-alan-butler}}

artwork text

\hypertarget{gasoline-stations-in-gta-v-by-lorna-ruth-galloway}{%
\section{\texorpdfstring{\emph{26 Gasoline stations in GTA V} by Lorna Ruth Galloway}{26 Gasoline stations in GTA V by Lorna Ruth Galloway}}\label{gasoline-stations-in-gta-v-by-lorna-ruth-galloway}}

artwork text

\hypertarget{gasoline-stations-in-gta-v-by-m.-earl-williams}{%
\section{\texorpdfstring{\emph{26 Gasoline stations in GTA V} by M. Earl Williams}{26 Gasoline stations in GTA V by M. Earl Williams}}\label{gasoline-stations-in-gta-v-by-m.-earl-williams}}

artwork text

\hypertarget{getting-there-3}{%
\subsection{Getting There}\label{getting-there-3}}

\hypertarget{readings-2}{%
\section{Readings}\label{readings-2}}

\hypertarget{tutorial-2}{%
\section{Tutorial}\label{tutorial-2}}

\hypertarget{scene-director-mode}{%
\subsection{Scene Director Mode}\label{scene-director-mode}}

\hypertarget{content-replication-assignment-2}{%
\section{Content Replication Assignment}\label{content-replication-assignment-2}}

\hypertarget{nature-documentary}{%
\chapter{Nature Documentary}\label{nature-documentary}}

\hypertarget{deercam-by-brent-watanabe}{%
\section{Deercam by Brent Watanabe}\label{deercam-by-brent-watanabe}}

artwork text

\hypertarget{getting-there-4}{%
\subsection{Getting There}\label{getting-there-4}}

\hypertarget{virtual-flora}{%
\section{Virtual Flora}\label{virtual-flora}}

artwork text

\hypertarget{getting-there-5}{%
\subsection{Getting There}\label{getting-there-5}}

\hypertarget{readings-3}{%
\section{Readings}\label{readings-3}}

\hypertarget{tutorial-3}{%
\section{Tutorial}\label{tutorial-3}}

\hypertarget{modding-introduction}{%
\subsection{Modding Introduction}\label{modding-introduction}}

\hypertarget{preparation-and-setup}{%
\subsubsection{Preparation and Setup}\label{preparation-and-setup}}

\begin{itemize}
\item
  Install Windows 11
\item
  Download and install \href{https://store.steampowered.com/about/}{Steam} (with a copy of GTA V or buy the game if you do not have it. GTA V is 100+ GB so it will take a few hours depending on your internet connections)
\item
  Download \href{https://store.steampowered.com/about/}{Script Hook V}, go to the bin folder and copy \texttt{dinput8.dll} and \texttt{ScriptHookV.dll} files into your GTA V directory \texttt{C:\textbackslash{}Program\ Files\ (x86)\textbackslash{}Steam\textbackslash{}steamapps\textbackslash{}common\textbackslash{}Grand\ Theft\ Auto\ V}
\item
  Download \href{https://store.steampowered.com/about/}{Script Hook V dot net}, copy the \texttt{ScriptHookVDotNet.asi} file, \texttt{ScriptHookVDotNet2.dll} and \texttt{ScriptHookVDotNet3.dll} files into your GTA V directory \texttt{C:\textbackslash{}Program\ Files\ (x86)\textbackslash{}Steam\textbackslash{}steamapps\textbackslash{}common\textbackslash{}Grand\ Theft\ Auto\ V}
\item
  Create a new folder in GTA V directory and call it ``scripts''.
\item
  Download and install \href{https://store.steampowered.com/about/}{Visual Studio Community} (free version of VS). Open Visual Studio and check the .NET desktop development package and install it
\item
  Run GTA V and test if Script Hook V is working by pressing \texttt{F4}. This should toggle the console view.
\end{itemize}

\hypertarget{creating-a-mod-file}{%
\subsubsection{Creating a Mod File}\label{creating-a-mod-file}}

\begin{itemize}
\item
  Open Visual Studio
\item
  Select File \textgreater{} New \textgreater{} Project
\item
  Select Visual C\# and Class Library (.NET Framework)
\item
  Give a custom file name (e.g.~moddingTutorial)
\item
  Rename public class Class1 as ``moddingTutorial'' in the right panel Solution Explorer
\item
  In the same panel go to References and click add References\ldots{} \textgreater{} Browse \textgreater{} browse to Downloads
\item
  Select ScriptHookedVDotNet \textgreater{} \texttt{ScriptHookVDotNet2.dll} and \texttt{ScriptHookVDotNet3.dll} and add them
\item
  Also add \texttt{System.Windows.forms}
\item
  Also add \texttt{System.Drawing}
\item
  In your code file add the following lines on top:
\end{itemize}

\begin{Shaded}
\begin{Highlighting}[]
\KeywordTok{using}\NormalTok{ GTA;}
\KeywordTok{using}\NormalTok{ GTA.}\FunctionTok{Math}\NormalTok{;}
\KeywordTok{using}\NormalTok{ System.}\FunctionTok{Windows}\NormalTok{.}\FunctionTok{Forms}\NormalTok{;}
\KeywordTok{using}\NormalTok{ System.}\FunctionTok{Drawing}\NormalTok{;}
\KeywordTok{using}\NormalTok{ GTA.}\FunctionTok{Native}\NormalTok{;}
\end{Highlighting}
\end{Shaded}

\begin{itemize}
\tightlist
\item
  Modify class moddingTutorial to the following:
\end{itemize}

\begin{Shaded}
\begin{Highlighting}[]
\KeywordTok{namespace}\NormalTok{ moddingTutorial}
\NormalTok{\{}
  \KeywordTok{public} \KeywordTok{class}\NormalTok{ moddingTutorial : Script}
\NormalTok{  \{}
      \KeywordTok{public} \FunctionTok{moddingTutorial}\NormalTok{()}
\NormalTok{      \{}
          \KeywordTok{this}\NormalTok{.}\FunctionTok{Tick}\NormalTok{ += onTick;}
      \KeywordTok{this}\NormalTok{.}\FunctionTok{KeyUp}\NormalTok{ += onKeyUp;}
      \KeywordTok{this}\NormalTok{.}\FunctionTok{KeyDown}\NormalTok{ += onKeyDown;}
\NormalTok{      \}}
    
      \KeywordTok{private} \DataTypeTok{void} \FunctionTok{onTick}\NormalTok{(}\DataTypeTok{object}\NormalTok{ sender, EventArgs e)}
\NormalTok{      \{}
\NormalTok{      \}}
    
      \KeywordTok{private} \DataTypeTok{void} \FunctionTok{onKeyUp}\NormalTok{(}\DataTypeTok{object}\NormalTok{ sender, KeyEventArgs e)}
\NormalTok{      \{}
\NormalTok{      \}}
 
      \KeywordTok{private} \DataTypeTok{void} \FunctionTok{onKeyDown}\NormalTok{(}\DataTypeTok{object}\NormalTok{ sender, KeyEventArgs e)}
\NormalTok{      \{}
        \KeywordTok{if}\NormalTok{ (e.}\FunctionTok{KeyCode}\NormalTok{ == Keys.}\FunctionTok{H}\NormalTok{)}
\NormalTok{        \{}
\NormalTok{              Game.}\FunctionTok{Player}\NormalTok{.}\FunctionTok{ChangeModel}\NormalTok{(PedHash.}\FunctionTok{Cat}\NormalTok{); }
\NormalTok{          \}}
\NormalTok{      \} }
\NormalTok{  \}}
\NormalTok{\}}
\end{Highlighting}
\end{Shaded}

\begin{itemize}
\item
  Save file
\item
  Go to Documents \textgreater{} Visual Studio \textgreater{} Project \textgreater{} moddingTutorial \textgreater{} moddingTutorial \textgreater{} \texttt{moddingTutorial.cs}
\item
  Copy the .cs file in the GTA V directory inside the scripts folder
\item
  Open GTA V, run the game in Story Mode (mods are only allowed in single player mode, not in GTA Online) and press `H' to see if the game turns your avatar into a cat
\item
  Note: every time you make changes to your .cs file in the scripts folder you can hit \texttt{F4} to open the console, type \texttt{Reload()} in the console for the program to reload the script and test again the changes.
\end{itemize}

\hypertarget{ontick-onkeyup-and-onkeydown}{%
\subsubsection{onTick, onKeyUp and onKeyDown}\label{ontick-onkeyup-and-onkeydown}}

The main events of Script Hook V Dot Net are onTick, onKeyUp and onKeyDown. Script Hook V Dot Net will invoke your functions whenever an event is called.

The code within the onTick brackets is executed every interval milliseconds (which is by default 0), meaning that the event will be executed at every frame, for as long as the game is running.

\begin{Shaded}
\begin{Highlighting}[]
 \KeywordTok{private} \DataTypeTok{void} \FunctionTok{onTick}\NormalTok{(}\DataTypeTok{object}\NormalTok{ sender, EventArgs e)}
\NormalTok{      \{}
      \CommentTok{//code here will be executed every frame (or per usef defined interval)}
\NormalTok{      \}}
\end{Highlighting}
\end{Shaded}

If your function is written inside onKeyDown (withiin the curly brackets following onKeyUp(object sender, KeyEventArgs e)\{\}), your code will be executed every time a key is pressed. If your function is written inside onKeyUp, your code will be executed every time a key is released.

\begin{Shaded}
\begin{Highlighting}[]
\KeywordTok{private} \DataTypeTok{void} \FunctionTok{onKeyUp}\NormalTok{(}\DataTypeTok{object}\NormalTok{ sender, KeyEventArgs e)}
\NormalTok{      \{}
      \CommentTok{//code here will be executed whenever a key is released}
\NormalTok{      \}}
\KeywordTok{private} \DataTypeTok{void} \FunctionTok{onKeyDown}\NormalTok{(}\DataTypeTok{object}\NormalTok{ sender, KeyEventArgs e)}
\NormalTok{      \{}
      \CommentTok{//code here will be executed whenever a key is pressed}
\NormalTok{      \} }
\end{Highlighting}
\end{Shaded}

We can specify which code is executed based on what keys are pressed/released

\begin{Shaded}
\begin{Highlighting}[]
\KeywordTok{private} \DataTypeTok{void} \FunctionTok{onKeyDown}\NormalTok{(}\DataTypeTok{object}\NormalTok{ sender, KeyEventArgs e)}
\NormalTok{      \{}
        \KeywordTok{if}\NormalTok{ (e.}\FunctionTok{KeyCode}\NormalTok{ == Keys.}\FunctionTok{H}\NormalTok{)}
\NormalTok{        \{}
               \CommentTok{//code here will be executed whenever the key 'H' is pressed }
\NormalTok{          \}}
\NormalTok{      \} }
\end{Highlighting}
\end{Shaded}

\hypertarget{change-player-model}{%
\subsubsection{Change Player Model}\label{change-player-model}}

The player character is controlled as Game.Player. Game.Player can perform different functions, including changing the avatar model, and performing tasks.

Change the 3D model of your character by using the \emph{ChangeModel} function.
The \emph{ChangeModel} function needs a model ID, in order to load the model file of our game character.
You can browse through this list of models to find the one you want to try: \url{https://wiki.gtanet.work/index.php/Peds}

These models are all PedHashes, basically ID numbers within the PedHash group. Copy the name of the model below the image and add it to PedHash.
For example if you choose the model Cat, you'll need to write \emph{PedHash.Cat}.

To change the model of your player character into a cat you can write the following function:

\begin{Shaded}
\begin{Highlighting}[]
\NormalTok{Game.}\FunctionTok{Player}\NormalTok{.}\FunctionTok{ChangeModel}\NormalTok{(PedHash.}\FunctionTok{Cat}\NormalTok{);}
\end{Highlighting}
\end{Shaded}

add it in your .cs file in the onKeyDown event, triggered by the pressing of the `h' key:

Example code

\begin{Shaded}
\begin{Highlighting}[]
\KeywordTok{using}\NormalTok{ System;}
\KeywordTok{using}\NormalTok{ System.}\FunctionTok{Collections}\NormalTok{.}\FunctionTok{Generic}\NormalTok{;}
\KeywordTok{using}\NormalTok{ System.}\FunctionTok{Linq}\NormalTok{;}
\KeywordTok{using}\NormalTok{ System.}\FunctionTok{Text}\NormalTok{;}
\KeywordTok{using}\NormalTok{ System.}\FunctionTok{Threading}\NormalTok{.}\FunctionTok{Tasks}\NormalTok{;}
 
\KeywordTok{using}\NormalTok{ GTA;}
\KeywordTok{using}\NormalTok{ GTA.}\FunctionTok{Math}\NormalTok{;}
\KeywordTok{using}\NormalTok{ System.}\FunctionTok{Windows}\NormalTok{.}\FunctionTok{Forms}\NormalTok{;}
\KeywordTok{using}\NormalTok{ System.}\FunctionTok{Drawing}\NormalTok{;}
\KeywordTok{using}\NormalTok{ GTA.}\FunctionTok{Native}\NormalTok{;}
 
 
\KeywordTok{namespace}\NormalTok{ moddingTutorial}
\NormalTok{\{}
    \KeywordTok{public} \KeywordTok{class}\NormalTok{ moddingTutorial : Script}
\NormalTok{    \{}
        \KeywordTok{public} \FunctionTok{moddingTutorial}\NormalTok{()}
\NormalTok{        \{}
            \KeywordTok{this}\NormalTok{.}\FunctionTok{Tick}\NormalTok{ += onTick;}
            \KeywordTok{this}\NormalTok{.}\FunctionTok{KeyUp}\NormalTok{ += onKeyUp;}
            \KeywordTok{this}\NormalTok{.}\FunctionTok{KeyDown}\NormalTok{ += onKeyDown;}
\NormalTok{        \}}
 
        \KeywordTok{private} \DataTypeTok{void} \FunctionTok{onTick}\NormalTok{(}\DataTypeTok{object}\NormalTok{ sender, EventArgs e) }\CommentTok{//this function gets executed continuously }
\NormalTok{        \{}
\NormalTok{        \}}
 
        \KeywordTok{private} \DataTypeTok{void} \FunctionTok{onKeyUp}\NormalTok{(}\DataTypeTok{object}\NormalTok{ sender, KeyEventArgs e)}\CommentTok{//everything inside here is executed only when we release a key}
\NormalTok{        \{}
\NormalTok{        \}}
 
        \KeywordTok{private} \DataTypeTok{void} \FunctionTok{onKeyDown}\NormalTok{(}\DataTypeTok{object}\NormalTok{ sender, KeyEventArgs e) }\CommentTok{//everything inside here is executed only when we press a key}
\NormalTok{        \{}
            \CommentTok{//when pressing 'H'}
            \KeywordTok{if}\NormalTok{(e.}\FunctionTok{KeyCode}\NormalTok{ == Keys.}\FunctionTok{H}\NormalTok{)}
\NormalTok{            \{}
                \CommentTok{//change player char into a different model}
\NormalTok{                Game.}\FunctionTok{Player}\NormalTok{.}\FunctionTok{ChangeModel}\NormalTok{(PedHash.}\FunctionTok{Cat}\NormalTok{); }
\NormalTok{            \}}
\NormalTok{        \}}
\NormalTok{    \}}
\NormalTok{\}}
\end{Highlighting}
\end{Shaded}

\hypertarget{tasks}{%
\subsubsection{Tasks}\label{tasks}}

Our character can be controlled by our script, and given actions that override manual control of the player. These actions are called \emph{Tasks} and in order to assign tasks to our characters we have to define our \emph{Game.Player} as \emph{Game.Player.Character}. The \emph{Game.Player.Character} code gets the specific model the player is controlling.
Now we can give tasks to the character by adding the \emph{Task} function: \emph{Game.Player.Character.Task}.

\hypertarget{content-replication-assignment-3}{%
\section{Content Replication Assignment}\label{content-replication-assignment-3}}

\hypertarget{deercam-reenactment}{%
\subsection{Deercam reenactment}\label{deercam-reenactment}}

Write a mod script to change your game character into a deer by pressing a key, and make it autonomously wander around Los Santos by pressing another key.

\hypertarget{surrealist-photography}{%
\chapter{Surrealist Photography}\label{surrealist-photography}}

\hypertarget{alexey-andrienko-aka-happ-v2}{%
\section{Alexey Andrienko aka HAPP v2}\label{alexey-andrienko-aka-happ-v2}}

artwork text

\hypertarget{getting-there-6}{%
\subsection{Getting There}\label{getting-there-6}}

\hypertarget{readings-4}{%
\section{Readings}\label{readings-4}}

\hypertarget{tutorial-4}{%
\section{Tutorial}\label{tutorial-4}}

\hypertarget{modding-peds}{%
\subsection{Modding Peds}\label{modding-peds}}

\hypertarget{npcs}{%
\subsubsection{NPCs}\label{npcs}}

\hypertarget{spawn-a-new-npc}{%
\subsubsection{Spawn a new NPC}\label{spawn-a-new-npc}}

\begin{Shaded}
\begin{Highlighting}[]
\CommentTok{// spawn a new model npc in front of player}
\DataTypeTok{var}\NormalTok{ npc = World.}\FunctionTok{CreatePed}\NormalTok{(PedHash.}\FunctionTok{Cat}\NormalTok{, Game.}\FunctionTok{Player}\NormalTok{.}\FunctionTok{Character}\NormalTok{.}\FunctionTok{GetOffsetInWorldCoords}\NormalTok{(}\KeywordTok{new} \FunctionTok{Vector3}\NormalTok{(}\DecValTok{0}\NormalTok{, }\DecValTok{5}\NormalTok{, }\DecValTok{0}\NormalTok{))); }
\end{Highlighting}
\end{Shaded}

\hypertarget{give-tasks-to-npcs}{%
\subsubsection{Give Tasks to NPCs}\label{give-tasks-to-npcs}}

\hypertarget{content-replication-assignment-4}{%
\section{Content Replication Assignment}\label{content-replication-assignment-4}}

\end{document}
