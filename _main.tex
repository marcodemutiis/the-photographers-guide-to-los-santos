% Options for packages loaded elsewhere
\PassOptionsToPackage{unicode}{hyperref}
\PassOptionsToPackage{hyphens}{url}
%
\documentclass[
  openany]{book}
\usepackage{lmodern}
\usepackage{amssymb,amsmath}
\usepackage{ifxetex,ifluatex}
\ifnum 0\ifxetex 1\fi\ifluatex 1\fi=0 % if pdftex
  \usepackage[T1]{fontenc}
  \usepackage[utf8]{inputenc}
  \usepackage{textcomp} % provide euro and other symbols
\else % if luatex or xetex
  \usepackage{unicode-math}
  \defaultfontfeatures{Scale=MatchLowercase}
  \defaultfontfeatures[\rmfamily]{Ligatures=TeX,Scale=1}
\fi
% Use upquote if available, for straight quotes in verbatim environments
\IfFileExists{upquote.sty}{\usepackage{upquote}}{}
\IfFileExists{microtype.sty}{% use microtype if available
  \usepackage[]{microtype}
  \UseMicrotypeSet[protrusion]{basicmath} % disable protrusion for tt fonts
}{}
\makeatletter
\@ifundefined{KOMAClassName}{% if non-KOMA class
  \IfFileExists{parskip.sty}{%
    \usepackage{parskip}
  }{% else
    \setlength{\parindent}{0pt}
    \setlength{\parskip}{6pt plus 2pt minus 1pt}}
}{% if KOMA class
  \KOMAoptions{parskip=half}}
\makeatother
\usepackage{xcolor}
\IfFileExists{xurl.sty}{\usepackage{xurl}}{} % add URL line breaks if available
\IfFileExists{bookmark.sty}{\usepackage{bookmark}}{\usepackage{hyperref}}
\hypersetup{
  pdftitle={The Photographer's Guide to Los Santos},
  hidelinks,
  pdfcreator={LaTeX via pandoc}}
\urlstyle{same} % disable monospaced font for URLs
\usepackage{color}
\usepackage{fancyvrb}
\newcommand{\VerbBar}{|}
\newcommand{\VERB}{\Verb[commandchars=\\\{\}]}
\DefineVerbatimEnvironment{Highlighting}{Verbatim}{commandchars=\\\{\}}
% Add ',fontsize=\small' for more characters per line
\usepackage{framed}
\definecolor{shadecolor}{RGB}{248,248,248}
\newenvironment{Shaded}{\begin{snugshade}}{\end{snugshade}}
\newcommand{\AlertTok}[1]{\textcolor[rgb]{0.94,0.16,0.16}{#1}}
\newcommand{\AnnotationTok}[1]{\textcolor[rgb]{0.56,0.35,0.01}{\textbf{\textit{#1}}}}
\newcommand{\AttributeTok}[1]{\textcolor[rgb]{0.77,0.63,0.00}{#1}}
\newcommand{\BaseNTok}[1]{\textcolor[rgb]{0.00,0.00,0.81}{#1}}
\newcommand{\BuiltInTok}[1]{#1}
\newcommand{\CharTok}[1]{\textcolor[rgb]{0.31,0.60,0.02}{#1}}
\newcommand{\CommentTok}[1]{\textcolor[rgb]{0.56,0.35,0.01}{\textit{#1}}}
\newcommand{\CommentVarTok}[1]{\textcolor[rgb]{0.56,0.35,0.01}{\textbf{\textit{#1}}}}
\newcommand{\ConstantTok}[1]{\textcolor[rgb]{0.00,0.00,0.00}{#1}}
\newcommand{\ControlFlowTok}[1]{\textcolor[rgb]{0.13,0.29,0.53}{\textbf{#1}}}
\newcommand{\DataTypeTok}[1]{\textcolor[rgb]{0.13,0.29,0.53}{#1}}
\newcommand{\DecValTok}[1]{\textcolor[rgb]{0.00,0.00,0.81}{#1}}
\newcommand{\DocumentationTok}[1]{\textcolor[rgb]{0.56,0.35,0.01}{\textbf{\textit{#1}}}}
\newcommand{\ErrorTok}[1]{\textcolor[rgb]{0.64,0.00,0.00}{\textbf{#1}}}
\newcommand{\ExtensionTok}[1]{#1}
\newcommand{\FloatTok}[1]{\textcolor[rgb]{0.00,0.00,0.81}{#1}}
\newcommand{\FunctionTok}[1]{\textcolor[rgb]{0.00,0.00,0.00}{#1}}
\newcommand{\ImportTok}[1]{#1}
\newcommand{\InformationTok}[1]{\textcolor[rgb]{0.56,0.35,0.01}{\textbf{\textit{#1}}}}
\newcommand{\KeywordTok}[1]{\textcolor[rgb]{0.13,0.29,0.53}{\textbf{#1}}}
\newcommand{\NormalTok}[1]{#1}
\newcommand{\OperatorTok}[1]{\textcolor[rgb]{0.81,0.36,0.00}{\textbf{#1}}}
\newcommand{\OtherTok}[1]{\textcolor[rgb]{0.56,0.35,0.01}{#1}}
\newcommand{\PreprocessorTok}[1]{\textcolor[rgb]{0.56,0.35,0.01}{\textit{#1}}}
\newcommand{\RegionMarkerTok}[1]{#1}
\newcommand{\SpecialCharTok}[1]{\textcolor[rgb]{0.00,0.00,0.00}{#1}}
\newcommand{\SpecialStringTok}[1]{\textcolor[rgb]{0.31,0.60,0.02}{#1}}
\newcommand{\StringTok}[1]{\textcolor[rgb]{0.31,0.60,0.02}{#1}}
\newcommand{\VariableTok}[1]{\textcolor[rgb]{0.00,0.00,0.00}{#1}}
\newcommand{\VerbatimStringTok}[1]{\textcolor[rgb]{0.31,0.60,0.02}{#1}}
\newcommand{\WarningTok}[1]{\textcolor[rgb]{0.56,0.35,0.01}{\textbf{\textit{#1}}}}
\usepackage{longtable,booktabs}
% Correct order of tables after \paragraph or \subparagraph
\usepackage{etoolbox}
\makeatletter
\patchcmd\longtable{\par}{\if@noskipsec\mbox{}\fi\par}{}{}
\makeatother
% Allow footnotes in longtable head/foot
\IfFileExists{footnotehyper.sty}{\usepackage{footnotehyper}}{\usepackage{footnote}}
\makesavenoteenv{longtable}
\usepackage{graphicx,grffile}
\makeatletter
\def\maxwidth{\ifdim\Gin@nat@width>\linewidth\linewidth\else\Gin@nat@width\fi}
\def\maxheight{\ifdim\Gin@nat@height>\textheight\textheight\else\Gin@nat@height\fi}
\makeatother
% Scale images if necessary, so that they will not overflow the page
% margins by default, and it is still possible to overwrite the defaults
% using explicit options in \includegraphics[width, height, ...]{}
\setkeys{Gin}{width=\maxwidth,height=\maxheight,keepaspectratio}
% Set default figure placement to htbp
\makeatletter
\def\fps@figure{htbp}
\makeatother
\setlength{\emergencystretch}{3em} % prevent overfull lines
\providecommand{\tightlist}{%
  \setlength{\itemsep}{0pt}\setlength{\parskip}{0pt}}
\setcounter{secnumdepth}{5}

\title{The Photographer's Guide to Los Santos}
\author{}
\date{\vspace{-2.5em}}

\begin{document}
\maketitle

{
\setcounter{tocdepth}{1}
\tableofcontents
}
\hypertarget{preface}{%
\chapter*{Preface}\label{preface}}
\addcontentsline{toc}{chapter}{Preface}

\begin{quote}
``Los Santos. The city of shitheads. Where else would he be?''

--- Trevor Philips
\end{quote}

\hypertarget{introduction}{%
\chapter{Introduction}\label{introduction}}

\hypertarget{about-the-photographers-guide-to-los-santos}{%
\section{About The Photographer's Guide to Los Santos}\label{about-the-photographers-guide-to-los-santos}}

The Photographer's Guide to Los Santos sits between a touristic guide and a photography manual, and between an exhibition catalogue and a peak behind the scenes of artwork creation.

The Photographer's Guide to Los Santos is an ongoing project that builds on top of a research on artistic practices within spaces of computer games, with a particular focus on in-game photography, machinima and digital visual arts. It follows some themes and ideas previously explored in the exhibition \href{https://www.howtowinat.photography/}{\emph{How to Win at Photography}}, while focusing more specifically on the relationship between computer games and photographic activities inside the world of Grand Theft Auto V.

The idea of a guide refers to in-game photography as a form of `virtual tourism' (\href{https://papers.ssrn.com/sol3/papers.cfm?abstract_id=538182}{Book, 2003}), which was also the premise of an actual tourist guide published by Rough Guides in their 2019 \href{https://www.roughguides.com/articles/introduction-to-the-rough-guide-to-xbox/}{\emph{Rough Guide to XBOX}}. Yet this guide project also understands the game world as a site for image production and artistic creation, turning the game into a destination for a `game art tourist'. The Photographer's Guide to Los Santos presents the game environment of Grand Theft Auto V both as a space to explore and in which to create images, as well as a place to navigate and learn about some of the most important artworks that it has enabled to create.

The project also brings together several experiences from teaching in-game photography as an artistic practice in different educational settings and institutions, compiling materials and tools for students and artists interested in engaging with the field. The tourist guide of the game world doubles as a photography manual for the in-game photography age, featuring tutorials and excercises ranging from game screenshotting to computer programming for creative modding. Through the practical excercises, the project invites to rethink the game object as a space for creative, subversive and critical endeavours, which can be played differently, documented, reclaimed or modified through an artistic approach.

Finally, the project draws inspiration from the works of artists who have explored the `metaplay' of photographing game words instead of following the game rules and attempt to reach the goal of winning. The Photographer's Guide to Los Santos is indebted to all the artists it features, but was particularly inspired by Gareth Damian Martin's live streamed workshop \href{https://www.twitch.tv/videos/591840067}{\emph{Photography Tour of No Man's Sky}} (realized for \href{https://www.somersethouse.org.uk/whats-on/now-play-this-2020}{Now Play This Festival 2020}), Total Refusal \& Ismaël Joffroy Chandoutis's 2021 in-game lecture performance and guided tour \href{https://vimeo.com/506064357}{\emph{Everyday Daylight}} (realized for the \href{https://ccsparis.com/en/events/total-refusal-digital-disarmament-movement-a-la-gaite-lyrique/}{CCS Paris}), and Alan Butler's epic 2020 live endurance performance \href{https://www.youtube.com/watch?v=R4Q2G6tOQ_Q}{\emph{Witness to a Changing West}} (realized for \href{https://screenwalks.com/}{Screen Walks}) and his `Content Replication Assignments'.

\hypertarget{grand-theft-auto-v-studies}{%
\section{Grand Theft Auto V Studies}\label{grand-theft-auto-v-studies}}

Los Santos is the Grand Theft Auto V's fictional, parodic version of real-life Los Angeles. Just like Los Angeles is the global centre of film and commercial media production, Los Santos is the epicentre of in-game photography and machinima creation. While it may seem reductive to only focus on a single game to address the larger phenomenon of in-game photography, GTA V is the biggest source of creative outputs to date, with its extended open world and one of the largest community of active modders. Launched in 2013, the game contains a world map of more than 80 square kilometers of total area, which includes the urban area of the city of Los Santos and the rural area of Blaine County. This incredibly vast environment features a large desert region, dense forest, several mountains, beachside towns, on top of the large metropolis of Los Santos. The game simulates the everyday life of hundreds of individual NPCs (while it allegedly counts a population of over 4 million) as well as counting 28 animal species, and \href{https://grandtheftdata.com/landmarks/\#0,0,2,satellite}{more than 800 buildings in GTA V are based on real-life landmarks}. The size of the photorealistic simulation is only matched by the complexity of the game engine and its code, which - thanks to the effort of GTA V's modding community - allows players to use the game world as a powerful tool to create new scenes, take controls of its algorithmic entities, modify cameras and reshaping the game into a movie set or a photo studio.

\href{https://www.bragitoff.com/2015/11/gta-v-maps-quad-ultra-high-definition-8k-quality/gtav_satellite_2048x2048/}{GTA V Satellite Map by Manas Sharma}

\hypertarget{grand-theft-auto-v-tourism}{%
\section{Grand Theft Auto V Tourism}\label{grand-theft-auto-v-tourism}}

This guide allows players to explore the game environment following some of the most interesting artworks that have been created with(in) it. The guide is divided in thematic chapters that follow different artistic practices, taking place in different locations of the game environment, followed by different tutorials and exercises connected with the works and the space analyzed.
Each selected work is presented by a curatorial statement, introducing the work and its artistic relevance. The work is accompanied by information on the in-game location from which it was produced, inviting the readers to reach the destination in Grand Theft Auto V through maps and indications.

\hypertarget{grand-theft-auto-v-art-education}{%
\section{Grand Theft Auto V Art Education}\label{grand-theft-auto-v-art-education}}

Each thematic chapter features a tutorial section that introduces different techniques and strategies to capture images within Grand Theft Auto V. The chapters are thought to be experienced in order, as the tutorials at times rely on knowledge that is built on top of previous lessons. Each tutorial is accompanied by content replication assignments, in which the readers is invited to use the skills learnd from each chapter to recreate a work presented in that section.

\hypertarget{architecture-photography}{%
\chapter{Architecture Photography}\label{architecture-photography}}

\hypertarget{from-the-continuous-city-by-gareth-damian-martin}{%
\section{\texorpdfstring{from \emph{The Continuous City}, by Gareth Damian Martin}{from The Continuous City, by Gareth Damian Martin}}\label{from-the-continuous-city-by-gareth-damian-martin}}

artwork text

\hypertarget{interview-with-gareth-damian-martin}{%
\section{Interview with Gareth Damian Martin}\label{interview-with-gareth-damian-martin}}

\href{https://www.gamescenes.org/2018/04/interview-gareth-damian-martin-the-aesthetics-of-analogue-game-photography.html}{link}

\hypertarget{tutorial-photographing-the-game-screen}{%
\section{Tutorial: photographing the game screen}\label{tutorial-photographing-the-game-screen}}

\hypertarget{analogue-game-photography}{%
\subsection{analogue game photography}\label{analogue-game-photography}}

\hypertarget{screenshotting}{%
\subsection{screenshotting}\label{screenshotting}}

\hypertarget{content-replication-assignment}{%
\section{Content Replication Assignment:}\label{content-replication-assignment}}

\hypertarget{social-documentary}{%
\chapter{Social Documentary}\label{social-documentary}}

\hypertarget{down-and-out-in-los-santos-by-alan-butler}{%
\section{\texorpdfstring{\emph{Down and Out in Los Santos} by Alan Butler}{Down and Out in Los Santos by Alan Butler}}\label{down-and-out-in-los-santos-by-alan-butler}}

text and artwork

\hypertarget{location}{%
\subsection{location:}\label{location}}

\hypertarget{fear-and-loathing-in-gta-v-by-morten-rockford-ravn}{%
\section{\texorpdfstring{\emph{Fear and Loathing in GTA V} by Morten Rockford Ravn}{Fear and Loathing in GTA V by Morten Rockford Ravn}}\label{fear-and-loathing-in-gta-v-by-morten-rockford-ravn}}

text and artwork

\hypertarget{location-1}{%
\subsection{location:}\label{location-1}}

\hypertarget{tutorial-in-game-smartphone-camera}{%
\section{Tutorial: in-game smartphone camera}\label{tutorial-in-game-smartphone-camera}}

\hypertarget{content-replication-assignment-1}{%
\section{Content Replication Assignment:}\label{content-replication-assignment-1}}

\hypertarget{re-enactment-photography}{%
\chapter{Re-enactment photography}\label{re-enactment-photography}}

\hypertarget{a-study-on-perspective-by-roc-herms}{%
\section{\texorpdfstring{\emph{A Study on Perspective} by Roc Herms}{A Study on Perspective by Roc Herms}}\label{a-study-on-perspective-by-roc-herms}}

\hypertarget{little-books-of-los-santos-by-luke-caspar-pearson}{%
\section{\texorpdfstring{\emph{Little Books of Los Santos} by Luke Caspar Pearson}{Little Books of Los Santos by Luke Caspar Pearson}}\label{little-books-of-los-santos-by-luke-caspar-pearson}}

\hypertarget{nine-swimming-pools-and-a-broken-glas_s-by-alan-butler}{%
\section{\_Nine swimming pools and a broken glas\_s by Alan Butler}\label{nine-swimming-pools-and-a-broken-glas_s-by-alan-butler}}

\hypertarget{gasoline-stations-in-gta-v-by-lorna-ruth-galloway}{%
\section{\texorpdfstring{\emph{26 Gasoline stations in GTA V} by Lorna Ruth Galloway}{26 Gasoline stations in GTA V by Lorna Ruth Galloway}}\label{gasoline-stations-in-gta-v-by-lorna-ruth-galloway}}

\hypertarget{gasoline-stations-in-gta-v-by-m.-earl-williams}{%
\section{\texorpdfstring{\emph{26 Gasoline stations in GTA V} by M. Earl Williams}{26 Gasoline stations in GTA V by M. Earl Williams}}\label{gasoline-stations-in-gta-v-by-m.-earl-williams}}

\hypertarget{locations}{%
\subsection{locations:}\label{locations}}

\hypertarget{tutorial-scene-director-mode}{%
\section{Tutorial: scene director mode}\label{tutorial-scene-director-mode}}

\hypertarget{content-replication-assignment-2}{%
\section{Content Replication Assignment}\label{content-replication-assignment-2}}

\hypertarget{nature-documentary}{%
\chapter{Nature Documentary}\label{nature-documentary}}

\hypertarget{deercam-by-brent-watanabe}{%
\section{Deercam by Brent Watanabe}\label{deercam-by-brent-watanabe}}

text and artwork

\hypertarget{locations-1}{%
\subsection{locations:}\label{locations-1}}

\hypertarget{virtual-flora}{%
\section{Virtual Flora}\label{virtual-flora}}

text and artwork

\hypertarget{locations-2}{%
\subsection{locations:}\label{locations-2}}

\hypertarget{tutorial-modding-introduction}{%
\section{Tutorial: modding introduction}\label{tutorial-modding-introduction}}

\hypertarget{preparation-and-setup}{%
\subsection{Preparation and setup}\label{preparation-and-setup}}

\hypertarget{creating-a-mod-file}{%
\subsection{Creating a mod file}\label{creating-a-mod-file}}

\hypertarget{change-player-model}{%
\subsection{Change player model}\label{change-player-model}}

\hypertarget{content-replication-assignment-3}{%
\section{Content Replication Assignment:}\label{content-replication-assignment-3}}

\hypertarget{surrealist-photography}{%
\chapter{Surrealist Photography}\label{surrealist-photography}}

\hypertarget{alexey-andrienko-aka-happ-v2}{%
\section{Alexey Andrienko aka HAPP v2}\label{alexey-andrienko-aka-happ-v2}}

artwork and text

\hypertarget{tutorial-modding}{%
\section{Tutorial: modding}\label{tutorial-modding}}

\hypertarget{npcs}{%
\subsection{NPCs}\label{npcs}}

\hypertarget{spawn-a-new-npc}{%
\subsubsection{Spawn a new NPC}\label{spawn-a-new-npc}}

\begin{Shaded}
\begin{Highlighting}[]
\CommentTok{// spawn a new model npc in front of player}
\DataTypeTok{var}\NormalTok{ npc = World.}\FunctionTok{CreatePed}\NormalTok{(PedHash.}\FunctionTok{Cat}\NormalTok{, Game.}\FunctionTok{Player}\NormalTok{.}\FunctionTok{Character}\NormalTok{.}\FunctionTok{GetOffsetInWorldCoords}\NormalTok{(}\KeywordTok{new} \FunctionTok{Vector3}\NormalTok{(}\DecValTok{0}\NormalTok{, }\DecValTok{5}\NormalTok{, }\DecValTok{0}\NormalTok{))); }
\end{Highlighting}
\end{Shaded}

\hypertarget{content-replication-assignment-4}{%
\section{Content Replication Assignment:}\label{content-replication-assignment-4}}

\end{document}
