% Options for packages loaded elsewhere
\PassOptionsToPackage{unicode}{hyperref}
\PassOptionsToPackage{hyphens}{url}
%
\documentclass[
  openany]{book}
\usepackage{lmodern}
\usepackage{amssymb,amsmath}
\usepackage{ifxetex,ifluatex}
\ifnum 0\ifxetex 1\fi\ifluatex 1\fi=0 % if pdftex
  \usepackage[T1]{fontenc}
  \usepackage[utf8]{inputenc}
  \usepackage{textcomp} % provide euro and other symbols
\else % if luatex or xetex
  \usepackage{unicode-math}
  \defaultfontfeatures{Scale=MatchLowercase}
  \defaultfontfeatures[\rmfamily]{Ligatures=TeX,Scale=1}
\fi
% Use upquote if available, for straight quotes in verbatim environments
\IfFileExists{upquote.sty}{\usepackage{upquote}}{}
\IfFileExists{microtype.sty}{% use microtype if available
  \usepackage[]{microtype}
  \UseMicrotypeSet[protrusion]{basicmath} % disable protrusion for tt fonts
}{}
\makeatletter
\@ifundefined{KOMAClassName}{% if non-KOMA class
  \IfFileExists{parskip.sty}{%
    \usepackage{parskip}
  }{% else
    \setlength{\parindent}{0pt}
    \setlength{\parskip}{6pt plus 2pt minus 1pt}}
}{% if KOMA class
  \KOMAoptions{parskip=half}}
\makeatother
\usepackage{xcolor}
\IfFileExists{xurl.sty}{\usepackage{xurl}}{} % add URL line breaks if available
\IfFileExists{bookmark.sty}{\usepackage{bookmark}}{\usepackage{hyperref}}
\hypersetup{
  pdftitle={The Photographer's Guide to Los Santos},
  hidelinks,
  pdfcreator={LaTeX via pandoc}}
\urlstyle{same} % disable monospaced font for URLs
\usepackage{longtable,booktabs}
% Correct order of tables after \paragraph or \subparagraph
\usepackage{etoolbox}
\makeatletter
\patchcmd\longtable{\par}{\if@noskipsec\mbox{}\fi\par}{}{}
\makeatother
% Allow footnotes in longtable head/foot
\IfFileExists{footnotehyper.sty}{\usepackage{footnotehyper}}{\usepackage{footnote}}
\makesavenoteenv{longtable}
\usepackage{graphicx,grffile}
\makeatletter
\def\maxwidth{\ifdim\Gin@nat@width>\linewidth\linewidth\else\Gin@nat@width\fi}
\def\maxheight{\ifdim\Gin@nat@height>\textheight\textheight\else\Gin@nat@height\fi}
\makeatother
% Scale images if necessary, so that they will not overflow the page
% margins by default, and it is still possible to overwrite the defaults
% using explicit options in \includegraphics[width, height, ...]{}
\setkeys{Gin}{width=\maxwidth,height=\maxheight,keepaspectratio}
% Set default figure placement to htbp
\makeatletter
\def\fps@figure{htbp}
\makeatother
\setlength{\emergencystretch}{3em} % prevent overfull lines
\providecommand{\tightlist}{%
  \setlength{\itemsep}{0pt}\setlength{\parskip}{0pt}}
\setcounter{secnumdepth}{5}

\title{The Photographer's Guide to Los Santos}
\author{}
\date{\vspace{-2.5em}}

\begin{document}
\maketitle

{
\setcounter{tocdepth}{1}
\tableofcontents
}
\hypertarget{preface}{%
\chapter*{Preface}\label{preface}}
\addcontentsline{toc}{chapter}{Preface}

\begin{quote}
``Los Santos. The city of shitheads. Where else would he be?''

--- Trevor Philips
\end{quote}

\hypertarget{introduction}{%
\chapter{Introduction}\label{introduction}}

This publication focuses on artistic practices within the game Grand Theft Auto V. The goal of this is to create a resource to explore the ways in which the game has been used by different artists in the field of digital visual art and game art.
The Artist's Guide to Los Santos is partly a curatorial project to present important works of in-game photography and machinima, partly a virtual tourist guidebook through the game spaces that were explored and used by artists in their creation, and partly an educational resource for anyone interested in learning how to engage artistically with the game.

\hypertarget{los-santos}{%
\section{Los Santos}\label{los-santos}}

Just like Los Angeles -- the city it is modelled upon -- is the global centre of film and commercial media production, Los Santos is the epicentre of in-game photography and machinima creation.

\hypertarget{game-art-tourism}{%
\section{Game art tourism}\label{game-art-tourism}}

This guide allows players to explore the game environment following some of the artworks that have been created in it. Mixing a playful tourist guide approach with curatorial writings about the works, the publication is an invitation to rethink the game object as a space for creative, subversive and critical endeavours, as well as to showcase seminal artworks that have had a profound impact on contemporary visual culture.

\hypertarget{game-art-education}{%
\section{Game art education}\label{game-art-education}}

Finally, The Artist's Guide to Los Santos provides practical tools in the form of a technical manual. While many scattered recources already exists online, the publication aims at bringing together a series of tutorials that can be easily followed and that start from an artistic perspective. From photographic and cinematic tips, to step by step modding lessons, the guide covers the basic knowledge to capture and modify the world of GTA V. The lessons are also accompanied by re-enactment missions, where the reader is tasked to recreate some of the artworks mentioned in the guide.

\end{document}
