% Options for packages loaded elsewhere
\PassOptionsToPackage{unicode}{hyperref}
\PassOptionsToPackage{hyphens}{url}
%
\documentclass[
  openany]{book}
\usepackage{lmodern}
\usepackage{amssymb,amsmath}
\usepackage{ifxetex,ifluatex}
\ifnum 0\ifxetex 1\fi\ifluatex 1\fi=0 % if pdftex
  \usepackage[T1]{fontenc}
  \usepackage[utf8]{inputenc}
  \usepackage{textcomp} % provide euro and other symbols
\else % if luatex or xetex
  \usepackage{unicode-math}
  \defaultfontfeatures{Scale=MatchLowercase}
  \defaultfontfeatures[\rmfamily]{Ligatures=TeX,Scale=1}
\fi
% Use upquote if available, for straight quotes in verbatim environments
\IfFileExists{upquote.sty}{\usepackage{upquote}}{}
\IfFileExists{microtype.sty}{% use microtype if available
  \usepackage[]{microtype}
  \UseMicrotypeSet[protrusion]{basicmath} % disable protrusion for tt fonts
}{}
\makeatletter
\@ifundefined{KOMAClassName}{% if non-KOMA class
  \IfFileExists{parskip.sty}{%
    \usepackage{parskip}
  }{% else
    \setlength{\parindent}{0pt}
    \setlength{\parskip}{6pt plus 2pt minus 1pt}}
}{% if KOMA class
  \KOMAoptions{parskip=half}}
\makeatother
\usepackage{xcolor}
\IfFileExists{xurl.sty}{\usepackage{xurl}}{} % add URL line breaks if available
\IfFileExists{bookmark.sty}{\usepackage{bookmark}}{\usepackage{hyperref}}
\hypersetup{
  pdftitle={The Photographer's Guide to Los Santos},
  hidelinks,
  pdfcreator={LaTeX via pandoc}}
\urlstyle{same} % disable monospaced font for URLs
\usepackage{color}
\usepackage{fancyvrb}
\newcommand{\VerbBar}{|}
\newcommand{\VERB}{\Verb[commandchars=\\\{\}]}
\DefineVerbatimEnvironment{Highlighting}{Verbatim}{commandchars=\\\{\}}
% Add ',fontsize=\small' for more characters per line
\usepackage{framed}
\definecolor{shadecolor}{RGB}{248,248,248}
\newenvironment{Shaded}{\begin{snugshade}}{\end{snugshade}}
\newcommand{\AlertTok}[1]{\textcolor[rgb]{0.94,0.16,0.16}{#1}}
\newcommand{\AnnotationTok}[1]{\textcolor[rgb]{0.56,0.35,0.01}{\textbf{\textit{#1}}}}
\newcommand{\AttributeTok}[1]{\textcolor[rgb]{0.77,0.63,0.00}{#1}}
\newcommand{\BaseNTok}[1]{\textcolor[rgb]{0.00,0.00,0.81}{#1}}
\newcommand{\BuiltInTok}[1]{#1}
\newcommand{\CharTok}[1]{\textcolor[rgb]{0.31,0.60,0.02}{#1}}
\newcommand{\CommentTok}[1]{\textcolor[rgb]{0.56,0.35,0.01}{\textit{#1}}}
\newcommand{\CommentVarTok}[1]{\textcolor[rgb]{0.56,0.35,0.01}{\textbf{\textit{#1}}}}
\newcommand{\ConstantTok}[1]{\textcolor[rgb]{0.00,0.00,0.00}{#1}}
\newcommand{\ControlFlowTok}[1]{\textcolor[rgb]{0.13,0.29,0.53}{\textbf{#1}}}
\newcommand{\DataTypeTok}[1]{\textcolor[rgb]{0.13,0.29,0.53}{#1}}
\newcommand{\DecValTok}[1]{\textcolor[rgb]{0.00,0.00,0.81}{#1}}
\newcommand{\DocumentationTok}[1]{\textcolor[rgb]{0.56,0.35,0.01}{\textbf{\textit{#1}}}}
\newcommand{\ErrorTok}[1]{\textcolor[rgb]{0.64,0.00,0.00}{\textbf{#1}}}
\newcommand{\ExtensionTok}[1]{#1}
\newcommand{\FloatTok}[1]{\textcolor[rgb]{0.00,0.00,0.81}{#1}}
\newcommand{\FunctionTok}[1]{\textcolor[rgb]{0.00,0.00,0.00}{#1}}
\newcommand{\ImportTok}[1]{#1}
\newcommand{\InformationTok}[1]{\textcolor[rgb]{0.56,0.35,0.01}{\textbf{\textit{#1}}}}
\newcommand{\KeywordTok}[1]{\textcolor[rgb]{0.13,0.29,0.53}{\textbf{#1}}}
\newcommand{\NormalTok}[1]{#1}
\newcommand{\OperatorTok}[1]{\textcolor[rgb]{0.81,0.36,0.00}{\textbf{#1}}}
\newcommand{\OtherTok}[1]{\textcolor[rgb]{0.56,0.35,0.01}{#1}}
\newcommand{\PreprocessorTok}[1]{\textcolor[rgb]{0.56,0.35,0.01}{\textit{#1}}}
\newcommand{\RegionMarkerTok}[1]{#1}
\newcommand{\SpecialCharTok}[1]{\textcolor[rgb]{0.00,0.00,0.00}{#1}}
\newcommand{\SpecialStringTok}[1]{\textcolor[rgb]{0.31,0.60,0.02}{#1}}
\newcommand{\StringTok}[1]{\textcolor[rgb]{0.31,0.60,0.02}{#1}}
\newcommand{\VariableTok}[1]{\textcolor[rgb]{0.00,0.00,0.00}{#1}}
\newcommand{\VerbatimStringTok}[1]{\textcolor[rgb]{0.31,0.60,0.02}{#1}}
\newcommand{\WarningTok}[1]{\textcolor[rgb]{0.56,0.35,0.01}{\textbf{\textit{#1}}}}
\usepackage{longtable,booktabs}
% Correct order of tables after \paragraph or \subparagraph
\usepackage{etoolbox}
\makeatletter
\patchcmd\longtable{\par}{\if@noskipsec\mbox{}\fi\par}{}{}
\makeatother
% Allow footnotes in longtable head/foot
\IfFileExists{footnotehyper.sty}{\usepackage{footnotehyper}}{\usepackage{footnote}}
\makesavenoteenv{longtable}
\usepackage{graphicx,grffile}
\makeatletter
\def\maxwidth{\ifdim\Gin@nat@width>\linewidth\linewidth\else\Gin@nat@width\fi}
\def\maxheight{\ifdim\Gin@nat@height>\textheight\textheight\else\Gin@nat@height\fi}
\makeatother
% Scale images if necessary, so that they will not overflow the page
% margins by default, and it is still possible to overwrite the defaults
% using explicit options in \includegraphics[width, height, ...]{}
\setkeys{Gin}{width=\maxwidth,height=\maxheight,keepaspectratio}
% Set default figure placement to htbp
\makeatletter
\def\fps@figure{htbp}
\makeatother
\setlength{\emergencystretch}{3em} % prevent overfull lines
\providecommand{\tightlist}{%
  \setlength{\itemsep}{0pt}\setlength{\parskip}{0pt}}
\setcounter{secnumdepth}{5}

\title{The Photographer's Guide to Los Santos}
\author{}
\date{\vspace{-2.5em}}

\begin{document}
\maketitle

{
\setcounter{tocdepth}{1}
\tableofcontents
}
\hypertarget{preface}{%
\chapter*{Preface}\label{preface}}
\addcontentsline{toc}{chapter}{Preface}

\hypertarget{introduction}{%
\chapter{Introduction}\label{introduction}}

\hypertarget{about-the-photographers-guide-to-los-santos}{%
\section*{About The Photographer's Guide to Los Santos}\label{about-the-photographers-guide-to-los-santos}}
\addcontentsline{toc}{section}{About The Photographer's Guide to Los Santos}

The Photographer's Guide to Los Santos sits between a touristic guide and a photography manual, and between an exhibition catalogue and a peak behind the scenes of artwork creation.

The Photographer's Guide to Los Santos is an ongoing project that builds on top of a research on artistic practices within spaces of computer games, with a particular focus on in-game photography, machinima and digital visual arts. It follows some themes and ideas previously explored in the exhibition \href{https://www.howtowinat.photography/}{\emph{How to Win at Photography}}, while focusing more specifically on the relationship between computer games and photographic activities inside the world of Grand Theft Auto V.

The idea of a guide refers to in-game photography as a form of `virtual tourism' (\href{https://papers.ssrn.com/sol3/papers.cfm?abstract_id=538182}{Book, 2003}), which was also the premise of an actual tourist guide published by Rough Guides in their 2019 \href{https://www.roughguides.com/articles/introduction-to-the-rough-guide-to-xbox/}{\emph{Rough Guide to XBOX}}. Yet this guide project also understands the game world as a site for image production and artistic creation, turning the game into a destination for a `game art tourist'. The Photographer's Guide to Los Santos presents the game environment of Grand Theft Auto V both as a space to explore and in which to create images, as well as a place to navigate and learn about some of the most important artworks that it has enabled to create.

The project also brings together several experiences from teaching in-game photography as an artistic practice in different educational settings and institutions, compiling materials and tools for students and artists interested in engaging with the field. The tourist guide of the game world doubles as a photography manual for the in-game photography age, featuring tutorials ranging from game screenshotting to computer programming for creative modding. Through the practical exercises, the project invites to rethink the game object as a space for creative, subversive and critical endeavours, which can be played differently, documented, reclaimed or modified through an artistic approach.

Finally, the project draws inspiration from the works of artists who have explored the `metaplay' of photographing game words instead of following the game rules and attempt to reach the goal of winning. The Photographer's Guide to Los Santos is indebted to all the artists it features, but was particularly inspired by Gareth Damian Martin's live streamed workshop \href{https://www.twitch.tv/videos/591840067}{\emph{Photography Tour of No Man's Sky}} (realized for \href{https://www.somersethouse.org.uk/whats-on/now-play-this-2020}{Now Play This Festival 2020}), Total Refusal \& Ismaël Joffroy Chandoutis's 2021 in-game lecture performance and guided tour \href{https://vimeo.com/506064357}{\emph{Everyday Daylight}} (realized for the \href{https://ccsparis.com/en/events/total-refusal-digital-disarmament-movement-a-la-gaite-lyrique/}{CCS Paris}), and Alan Butler's epic 2020 live endurance performance \href{https://www.youtube.com/watch?v=R4Q2G6tOQ_Q}{\emph{Witness to a Changing West}} (realized for \href{https://screenwalks.com/}{Screen Walks}) and his `Content Replication Assignments'.

\hypertarget{grand-theft-auto-v-studies}{%
\section*{Grand Theft Auto V Studies}\label{grand-theft-auto-v-studies}}
\addcontentsline{toc}{section}{Grand Theft Auto V Studies}

Los Santos is the Grand Theft Auto V's fictional, parodic version of real-life Los Angeles. Just like Los Angeles is the global centre of film and commercial media production, Los Santos is the epicentre of in-game photography and machinima creation. While it may seem reductive to only focus on a single game to address the larger phenomenon of in-game photography, GTA V is the biggest source of creative outputs to date, with its extended open world and one of the largest community of active modders.

Launched in 2013, the game contains a world map of more than 80 square kilometers of total area, which includes the urban area of the city of Los Santos and the rural area of Blaine County. This incredibly vast environment features a large desert region, dense forest, several mountains, beachside towns, on top of the large metropolis of Los Santos. The game simulates the everyday life of hundreds of individual NPCs (while it allegedly counts a population of over 4 million) as well as counting 28 animal species, and \href{https://grandtheftdata.com/landmarks/\#0,0,2,satellite}{more than 800 buildings in GTA V are based on real-life landmarks}.

The size of the photorealistic simulation is only matched by the complexity of the game engine and its code, which - thanks to the effort of GTA V's modding community - allows players to use the game world as a powerful tool to create new scenes, take controls of its algorithmic entities, modify cameras and reshaping the game into a movie set or a photo studio.

\hypertarget{grand-theft-auto-v-tourism}{%
\section*{Grand Theft Auto V Tourism}\label{grand-theft-auto-v-tourism}}
\addcontentsline{toc}{section}{Grand Theft Auto V Tourism}

The project can be seen as something between a playful travel guide of Los Santos and one of the star maps offered to tourist in Hollywood, pointing to the homes of movie actors and hollywood celebrities. This guide allows players to explore the game environment following some of the most interesting artworks that have been created with(in) it. It's divided in thematic chapters that follow different artistic practices, taking place in different locations of the game environment, followed by different tutorials and exercises connected with the works and the space analyzed.

The themes explore different approaches and practices connected to established artistic and photographic currents, with a general introduction text that gives an overview of the ideas, contexts and issues connected to the specific topic. A selection of artworks for each themes is presented by a curatorial statement, introducing the work and its artistic relevance. The work is accompanied by information on the in-game location in which it was produced, inviting the readers to reach the destination in Grand Theft Auto V through maps and indications.

The game environment thus becomes the space for possible `art tours', getting insights into the artworks made in GTA V. This form of game tourism allows the player to see the behind the scenes, and experience the making of the works in its place of origin. While the complex algorithms of GTA V produce unexpected interactions and scenes, Los Santos is also stuck in the same time forever. Gas stations, shops, palm trees remain in the same state and location forever, allowing the tourist to witness the exact scene that was first encountered by the artists.

\hypertarget{grand-theft-auto-v-art-education}{%
\section*{Grand Theft Auto V Art Education}\label{grand-theft-auto-v-art-education}}
\addcontentsline{toc}{section}{Grand Theft Auto V Art Education}

This project is also an attempt to introduce a video game as a space for artistic intervention, and an invitation to use its mechanics, its code and its environment as a creative tool itself. The game can be played, documented and captured through a form of artistic play, that differs from normative gameplay and does not focus on advancinf and winning but rather engages with the game object critically. Furthermore, the game software can also be manipulated, modified and used as an apparatus to create new images and interactions. The goal of this guide is to combine a curatorial approach that leads the viewer to discover the artworks made in GTA V with a hands on approach that teaches the player the tools for possible artistic interventions in this space.

Games are often seen as producing specific cultures and shaping identities through through forms of play that follow the intentions of the developer. Here we understand games as objects to be reclaimed and tools to be deconstructed and rebuilt, both conceptually and literally. Consequently, players are not just passive actors that push buttons in the sequence that they are taught by the machine and its softwares, but open up the black box of the game and become critical thinkers and makers that actively play with the game, or even against it.

The Photographer's Guide to Los Santos can be employed as a resource to accompany workshops for students and artists approaching computer games and interested in learning how to engage with it. Each thematic chapter features a tutorial section that introduces different techniques and strategies to capture images within Grand Theft Auto V, connected to the examples and locations of each section. The chapters are thought to be experienced in order, as the tutorials at times rely on knowledge that is built on top of previous lessons. Each tutorial is accompanied by content replication assignments, in which the readers is invited to use the skills learnd from each chapter to recreate a work presented in that section. Tutorials are intended for anyone who might be playing GTA V for the first time in their life and do not assume any previous experience, although some basic idea of programming is helpful when dealing with scripting and modding the game.

\hypertarget{architectural-photography}{%
\chapter{Architectural Photography}\label{architectural-photography}}

//general intro on architecture in game spaces, the contruction of virtual cities, architecture photography and how it relates to the game environment, the player as a photographer documenting urban spaces\ldots{}

\begin{quote}
``I saw it as the end of architecture\ldots{} by pushing the concept to its limit and primarily by using the photograph as a point of departure. This is reflected in the idea that the great majority of images are no longer the expression of a subject, or the reality of an object, but almost exclusively the technical fulfilment of all its intrinsic possibilities. It's the photographic medium that does all the work. People think they're photographing a scene, but they're only technical operators of the device's infinite virtuality. The virtual is the device that wants nothing more than to function, that demands to function.''
-- Jean Baudrillard - \emph{The Singular Objects of Architecture and Philosophy}, 2000
\end{quote}

Architecture photography was born with the invention of heliographs, daguerrotypes and large format cameras in the first half of the 19th century. Due to the long exposure times, buildings were the ideal subjects for the early scientific experiments of oseph Nicéphore Niépce, Louis-Jacques-Mandé Daguerre and William Talbot.

\emph{View from the Window at Le Gras} by Joseph Nicéphore Niépce, 1826-27

\emph{Paris' Boulevard du Temple} by Louis-Jacques-Mandé Daguerre, 1839

While the relationship between architecture and photography has been part of the medium from its birth for technical reasons, this form of image making has evolved to visually explore the connection with material spaces and forms, as well as the relation between human perception and architectural bodies. The photographic image is not simply a document of a structure, but ``is, in fact, \emph{part} of its architecture''.\footnote{Lorenzo Rocha, \emph{Photography and Modern Architecture} ,``Concrete - Photography and Architecture'', Scheidegger \& Spiess, 2013} Curator Urs Stahel wrote that ``pictures {[}\ldots{]} offer a discourse that is unlike the physical experience of architecture. They transform volume into surface; distil matter into forms and signs. Photography shapes architecture, enlarging and reducing it, heightening and shortening it, accentuating it, yet largely leaving it to its own devices.''\footnote{Urs Stahel, Foreword to ``Concrete - Photography and Architecture'', Scheidegger \& Spiess, 2013}

The first architectural photographer is considered to be Joseph-Philibert Girault de Prangey (1804 -- 1892), who started to take daguerrotypes of iconic buildings like the Parthenon in Athens and Notre Dame in Paris from 1841. Architectural photography evolved in two distinct approaches, namely Elevation and Perspective. The Elevation Approach focuses on representing a structure as a two-dimensional image, obtaining a viewpoint that is parallel to the building and aimed at showing as many details as possible. The Perspective Approach aims at depicting the structure within the space, focusing on the third dimension and often taken at an angle or from a vantage point from a corner.

\emph{Cathédrale Notre-Dame de Paris} by Joseph-Philibert Girault Prangey, 1841

//Architecture photography and the modernist project
//Modern architecture started to flourish with photographs in about the 1920s when urban photographers like Eugène Atget, Berenice Abbott, Walker Evans, and T. Lux Feiningercame into play.

//Eugène Atget(12 February 1857 -- 4 August 1927), a French photographer known for his documentation of the streets of Paris before it entered Modernization. His works were published by Berenice Abbott after his death.

//Berenice Abbott(July 17, 1898 -- December 9, 1991), an American Photographer is known for New York and urban design photographs in the 1930s and also for her portraits of between-the-wars 20th century cultural figures.

//``Modernist architecture and photography have been ideologically interconnected'' Lorenzo Rocha
//close connection between photographer and architect:
//Armando Salas Portugal for Luis Barragán,
//Bill Engdahl for Mies van der Rohe,
//Julius Shulman for Richard Neutra

//Bernhard Becher and Hilla Becher, a German conceptual artist and photographer also known as Bernd and Hilla Becher. They worked on Duo in projects which include photography of industrial buildings and structures and they were often organized in grids. They wrote several books and their works are for public display in the Art Institute of Chicago, Chicago, Tate Gallery, London, Museum of Modern Art, New York, and several other famous museums.

//Michael Wesely, a German Architectural photographer known for his ultra-long exposure shots //Potsdamer Platz, Berlin (1997-99) by Michael Wesely

//Hélène Binet postmodern architecture

//CGI
//Into the universe of rendered architectural images - Joel McKim
//Rendering the Desert of The Real -- Tobias Revell
//The Entasis of Elon Musk -- Tamar Shafrir

//Game Urbanism
\url{https://www.youandpea.com/atlas}
heterotopias

``We are outnumbered by virtual worlds, overwhelmed by virtual architecture. Videogames and digital art have furnished us with a hundred thousand matterless forms--landscapes where no rock or earth has ever been present, cities founded on depthless skins of image and texture, expanses that will never see the light of a true sun.
And yet, somehow there is material here, a new kind of matter. Some of it is borrowed---photographs, texture references, photogrammetry. Other parts are inherent properties of digital worlds--- their obsession with surface, the logic of their light, their base particles; pixels, voxels, polygons.''\footnote{Gareth Damian Martin, \emph{Represented , Contested, Inverted}, in ``Heterotopias 001'', 2017}

\hypertarget{the-continuous-city-by-gareth-damian-martin}{%
\section*{\texorpdfstring{\emph{The Continuous City}, by Gareth Damian Martin}{The Continuous City, by Gareth Damian Martin}}\label{the-continuous-city-by-gareth-damian-martin}}
\addcontentsline{toc}{section}{\emph{The Continuous City}, by Gareth Damian Martin}

Gareth Damian Martin, \emph{Outskirts}, from \emph{The Continuous City},

Gareth Damian Martin, \emph{Pathways}, from \emph{The Continuous City},

artwork text

\href{https://socks-studio.com/2019/10/13/gareth-damian-martin-postcards-from-the-continuous-city-2018/}{More about \emph{The Continuous City}}

\href{https://www.gamescenes.org/2018/04/interview-gareth-damian-martin-the-aesthetics-of-analogue-game-photography.html}{Interview with Gareth Damian Martin}

\hypertarget{getting-there}{%
\subsection*{Getting there}\label{getting-there}}
\addcontentsline{toc}{subsection}{Getting there}

\begin{itemize}
\tightlist
\item
  \href{https://grandtheftdata.com/landmarks/\#951.507,-1144.265,4,atlas,name=trainyard_warehouse,Trainyard_Warehouse,_East_Los_Santos}{The intersection of Interstate 4 and Interstate 5} manifests the architecture of traffic of the megalopolis.
\end{itemize}

\hypertarget{readings}{%
\section*{Readings}\label{readings}}
\addcontentsline{toc}{section}{Readings}

\href{https://www.heterotopiaszine.com/}{Heterotopias}

Mark D Teo, The Urban Architecture of Los Angeles and Grand Theft Auto, 2015. \url{https://www.academia.edu/18173221/The_Urban_Architecture_of_Los_Angeles_and_Grand_Theft_Auto}

\hypertarget{tutorial}{%
\section*{Tutorial}\label{tutorial}}
\addcontentsline{toc}{section}{Tutorial}

\hypertarget{photographing-the-game-screen}{%
\subsection*{Photographing the Game Screen}\label{photographing-the-game-screen}}
\addcontentsline{toc}{subsection}{Photographing the Game Screen}

\hypertarget{analogue-game-photography}{%
\subsubsection*{Analogue Game Photography}\label{analogue-game-photography}}
\addcontentsline{toc}{subsubsection}{Analogue Game Photography}

//Photographs of a TV screen taken with a digital camera often exhibit moiré patterns. Since both the TV screen and the digital camera use a scanning technique to produce or to capture pictures with horizontal scan lines, the conflicting sets of lines cause the moiré patterns. To avoid the effect, the digital camera can be aimed at an angle of 30 degrees to the TV screen.

\hypertarget{screenshotting}{%
\subsubsection*{Screenshotting}\label{screenshotting}}
\addcontentsline{toc}{subsubsection}{Screenshotting}

On windows there are several ways to take a screenshot. To capture your entire screen and automatically save the screenshot, press the \texttt{Windows\ logo\ key} + \texttt{PrtScn\ key}. The screenshot will be saved to the \texttt{Pictures\ \textgreater{}\ Screenshots\ folder}.

On windows 10 and 11 you can use the Game bar to take game screenshots and start/stop game screen recordings. Press the \texttt{Windows\ logo\ key} + \texttt{G} on your keyboard to open Game Bar.

\begin{itemize}
\item
  Press the camera icon to take a screenshot of the game screen.
\item
  Press the circle icon to start a clip, then the square icon to stop recording the game screen.
\item
  Click on ``See my captures'' to access the image and video files.
\end{itemize}

\hypertarget{content-replication-assignment}{%
\section*{Content Replication Assignment}\label{content-replication-assignment}}
\addcontentsline{toc}{section}{Content Replication Assignment}

\hypertarget{social-documentary}{%
\chapter{Social Documentary}\label{social-documentary}}

//general intro on simulating society, the creation of NPCs, documentary and street photography traditions connected to politics of visibility and representation, and how they relate to the politics of simulation, how the player-photographer documents the creation of complex social spaces and reveals the process of simulating people and issues of class, gender, race in the game space\ldots{}

\hypertarget{down-and-out-in-los-santos-by-alan-butler}{%
\section*{\texorpdfstring{\emph{Down and Out in Los Santos} by Alan Butler}{Down and Out in Los Santos by Alan Butler}}\label{down-and-out-in-los-santos-by-alan-butler}}
\addcontentsline{toc}{section}{\emph{Down and Out in Los Santos} by Alan Butler}

artwork text

\href{http://www.alanbutler.info/down-and-out-in-los-santos-2016}{More about \emph{Down and Out in Los Santos}}

\hypertarget{getting-there-1}{%
\subsection*{Getting There}\label{getting-there-1}}
\addcontentsline{toc}{subsection}{Getting There}

The homeless camp in Los Santos is under the \href{https://grandtheftdata.com/landmarks/\#252.086,-1208.975,5,hybrid,name,Strawberry_Subway_Station,_Downtown}{Olympic Freeway in Strawberry}.

Dignity Village is a tent city established by homeless people near Procopio Beach, east of \href{https://grandtheftdata.com/landmarks/\#-171.667,6208.676,5,hybrid,name=paleto_bay,Belinda_May's_Beauty_Salon,_Paleto_Blvd,_Paleto_Bay}{Paleto Bay}.

\hypertarget{fear-and-loathing-in-gta-v-by-morten-rockford-ravn}{%
\section*{\texorpdfstring{\emph{Fear and Loathing in GTA V} by Morten Rockford Ravn}{Fear and Loathing in GTA V by Morten Rockford Ravn}}\label{fear-and-loathing-in-gta-v-by-morten-rockford-ravn}}
\addcontentsline{toc}{section}{\emph{Fear and Loathing in GTA V} by Morten Rockford Ravn}

artwork text

\href{https://fearandloathingingtav.tumblr.com/}{More about \emph{Fear and Loathing in GTA V}}

\hypertarget{getting-there-2}{%
\subsection*{Getting There}\label{getting-there-2}}
\addcontentsline{toc}{subsection}{Getting There}

\hypertarget{readings-1}{%
\section*{Readings}\label{readings-1}}
\addcontentsline{toc}{section}{Readings}

\hypertarget{tutorial-1}{%
\section*{Tutorial}\label{tutorial-1}}
\addcontentsline{toc}{section}{Tutorial}

\hypertarget{in-game-smartphone-camera}{%
\subsection*{In-game Smartphone Camera}\label{in-game-smartphone-camera}}
\addcontentsline{toc}{subsection}{In-game Smartphone Camera}

Snapmatic is the photo app on your simulated mobile phone in GTA V.

\begin{itemize}
\item
  Press \texttt{UP} on the on the keyboard (PC) or d-pad (Playstation) to bring up your phone.
\item
  Select the Snapmatic app - it's on the bottom left of the homescreen.
\item
  Move the camera with the \texttt{Mouse} on PC, or with the \texttt{RIGHT\ STICK} on Playstation.
\item
  Zoom in and out with the \texttt{Mouse\ Wheel} on PC, or \texttt{LEFT\ STICK} on Playstation.
\item
  You can shuffle through filters with \texttt{DOWN} or borders with \texttt{UP}.
\item
  To take selfie press the \texttt{Mouse\ Wheel\ Button} on PC or \texttt{R3\ STICK} on Playstation to turn the camera on yourself.
\item
  Once you're happy, take the photo with ``Ènter\texttt{on\ pc\ or}X``` on the Playstation. Press it again to save it to the Gallery.
\item
  You can upload your picture to your Rockstar Game sSocial Club profile by going to the gallery and pressing the ``Left Ctrl'' key on PC.
\item
  Your photos will be published on socialclub.rockstargames.com/member/USERNAME/photos, where USERNAME is replaced by your actual username.
\end{itemize}

\hypertarget{content-replication-assignment-1}{%
\section*{Content Replication Assignment}\label{content-replication-assignment-1}}
\addcontentsline{toc}{section}{Content Replication Assignment}

\hypertarget{re-enactment-photography}{%
\chapter{Re-enactment Photography}\label{re-enactment-photography}}

//general introduction on the development of photorealism in games, the relationship between photography and CGI, the remediation of photographic images and the analog apparatus, the player as photographer situated in the tradition of conceptual photographers like Sherrie Levine and Sturtevant, the copy as a conceptual approach that create new meaning through a similar image but a different context\ldots{}

\hypertarget{gasoline-stations-in-gta-v-by-lorna-ruth-galloway}{%
\section*{\texorpdfstring{\emph{26 Gasoline stations in GTA V} by Lorna Ruth Galloway}{26 Gasoline stations in GTA V by Lorna Ruth Galloway}}\label{gasoline-stations-in-gta-v-by-lorna-ruth-galloway}}
\addcontentsline{toc}{section}{\emph{26 Gasoline stations in GTA V} by Lorna Ruth Galloway}

artwork text

\href{https://www.lornaruthgalloway.com/charcoal-halftone}{More about \emph{26 Gasoline stations in GTA V}}

\hypertarget{getting-there-3}{%
\subsection*{Getting There}\label{getting-there-3}}
\addcontentsline{toc}{subsection}{Getting There}

\begin{itemize}
\tightlist
\item
  \href{https://grandtheftdata.com/landmarks/\#513.325,-1374.453,4,atlas,name=gas_station,Globe_Oil_Gas_Station,_Innocence_Blvd_\&_Alta_St,_South_Los_Santos}{Globe Oil Gas Station, Innocence Blvd \& Alta St, South Los Santos}
\item
  \href{https://grandtheftdata.com/landmarks/\#513.325,-1374.453,4,atlas,name=gas_station,LTD_Gas_Station,_Davis_Ave_\&_Grove_St,_South_Los_Santos}{LTD Gas Station, Davis Ave \& Grove St, South Los Santos}
\item
  \href{https://grandtheftdata.com/landmarks/\#378.149,-934.191,4,atlas,name=gas_station,LTD_Gas_Station,_Mirror_Park_Blvd_\&_W_Mirror_Dr,_Mirror_Park}{LTD Gas Station, Mirror Park Blvd \& W Mirror Dr, Mirror Park}
\item
  \href{https://grandtheftdata.com/landmarks/\#613.567,192.271,6,atlas,name=gas_station,Globe_Oil_Gas_Station,_Clinton_Ave_\&_Fenwell_Pl,_Vinewood_Hills}{Globe Oil Gas Station, Clinton Ave \& Fenwell Pl, Vinewood Hills}
\item
  \href{https://grandtheftdata.com/landmarks/\#513.325,-1374.453,4,atlas,name=gas_station,Xero_Gas_Station,_Strawberry_Ave_\&_Capital_Blvd,_South_Los_Santos}{Xero Gas Station, Strawberry Ave \& Capital Blvd, South Los Santos}
\item
  \href{https://grandtheftdata.com/landmarks/\#513.325,-1374.453,4,atlas,name=gas_station,RON_Gas_Station,_Davis_Ave_\&_Macdonald_St,_South_Los_Santos}{RON Gas Station, Davis Ave \& Macdonald St, South Los Santos}
\item
  \href{https://grandtheftdata.com/landmarks/\#513.325,-1374.453,4,atlas,name=gas_station,Xero_Gas_Station,_Calais_Ave_\&_Innocence_Blvd,_Little_Seoul}{Xero Gas Station, Calais Ave \& Innocence Blvd, Little Seoul}
\item
  \href{https://grandtheftdata.com/landmarks/\#513.325,-1374.453,4,atlas,name=gas_station,LTD_Gas_Station,_Lindsay_Circus_\&_Ginger_St,_Little_Seoul}{LTD Gas Station, Lindsay Circus \& Ginger St, Little Seoul}
\item
  \href{https://grandtheftdata.com/landmarks/\#-1543.17,-676.107,4,atlas,name=gas_station,RON_Gas_Station,_N_Rockford_Dr_\&_Perth_St,_Morningwood}{RON Gas Station, N Rockford Dr \& Perth St, Morningwood}
\item
  \href{https://grandtheftdata.com/landmarks/\#-1543.17,-676.107,4,atlas,name=gas_station,Xero_Gas_Station,_Great_Ocean_Hwy,_Pacific_Bluffs}{Xero Gas Station, Great Ocean Hwy, Pacific Bluffs}
\end{itemize}

\hypertarget{a-study-on-perspective-by-roc-herms}{%
\section*{\texorpdfstring{\emph{A Study on Perspective} by Roc Herms}{A Study on Perspective by Roc Herms}}\label{a-study-on-perspective-by-roc-herms}}
\addcontentsline{toc}{section}{\emph{A Study on Perspective} by Roc Herms}

artwork text

\href{https://www.rocherms.com/projects/study-of-perspective/}{More about \emph{A Study on Perspective}}

\hypertarget{getting-there-4}{%
\subsection*{Getting There}\label{getting-there-4}}
\addcontentsline{toc}{subsection}{Getting There}

\href{https://grandtheftdata.com/landmarks/\#1016.057,312.205,4,atlas,name=vinewood,Vinewood_Sign,_Vinewood_Hills}{Vinewood Sign, Vinewood Hills}

\hypertarget{further-references}{%
\section*{Further references}\label{further-references}}
\addcontentsline{toc}{section}{Further references}

\hypertarget{little-books-of-los-santos-by-luke-caspar-pearson}{%
\subsection*{\texorpdfstring{\emph{Little Books of Los Santos} by Luke Caspar Pearson}{Little Books of Los Santos by Luke Caspar Pearson}}\label{little-books-of-los-santos-by-luke-caspar-pearson}}
\addcontentsline{toc}{subsection}{\emph{Little Books of Los Santos} by Luke Caspar Pearson}

artwork text

\href{https://www.alephograph.com/little-books-of-los-santos}{More about \emph{Little Books of Los Santos}}

\hypertarget{gasoline-stations-in-gta-v-by-m.-earl-williams}{%
\subsection*{\texorpdfstring{\emph{26 Gasoline stations in GTA V} by M. Earl Williams}{26 Gasoline stations in GTA V by M. Earl Williams}}\label{gasoline-stations-in-gta-v-by-m.-earl-williams}}
\addcontentsline{toc}{subsection}{\emph{26 Gasoline stations in GTA V} by M. Earl Williams}

artwork text

\href{https://www.mearlwilliams.com/gasoline_stations\#1}{More about \emph{26 Gasoline stations in GTA V}}

\hypertarget{readings-2}{%
\section*{Readings}\label{readings-2}}
\addcontentsline{toc}{section}{Readings}

\hypertarget{tutorial-2}{%
\section*{Tutorial}\label{tutorial-2}}
\addcontentsline{toc}{section}{Tutorial}

\hypertarget{scene-director-mode}{%
\subsection*{Scene Director Mode}\label{scene-director-mode}}
\addcontentsline{toc}{subsection}{Scene Director Mode}

\hypertarget{content-replication-assignment-2}{%
\section*{Content Replication Assignment}\label{content-replication-assignment-2}}
\addcontentsline{toc}{section}{Content Replication Assignment}

\hypertarget{nature-documentary}{%
\chapter{Nature Documentary}\label{nature-documentary}}

//general introduction about the creation of a synthetic forms of nature, ecological issues, creation of virtual sublime, flora and fauna that are usually props that become the focus of the player's explorations, ``virtual world naturalism''\ldots{}

\hypertarget{san-andreas-streaming-deer-cam-by-brent-watanabe}{%
\section*{\texorpdfstring{\emph{San Andreas Streaming Deer Cam} by Brent Watanabe}{San Andreas Streaming Deer Cam by Brent Watanabe}}\label{san-andreas-streaming-deer-cam-by-brent-watanabe}}
\addcontentsline{toc}{section}{\emph{San Andreas Streaming Deer Cam} by Brent Watanabe}

artwork text

\href{https://bwatanabe.com/GTA_V_WanderingDeer.html}{More about \emph{Deercam}}

\hypertarget{getting-there-5}{%
\subsection*{Getting There}\label{getting-there-5}}
\addcontentsline{toc}{subsection}{Getting There}

\href{https://grandtheftdata.com/landmarks/\#86.534,6158.577,4,atlas,name=mount_chiliad,Mount_Chiliad}{Mount Chiliad} is located in the Chiliad Mountain State Wilderness, and it is the tallest mountain in the game at 798m above sea level. The state park is home to lots of wildlife such as deer and mountain lions.

\hypertarget{virtual-botany-cyanotype-by-alan-butler}{%
\section*{\texorpdfstring{\emph{Virtual Botany Cyanotype} by Alan Butler}{Virtual Botany Cyanotype by Alan Butler}}\label{virtual-botany-cyanotype-by-alan-butler}}
\addcontentsline{toc}{section}{\emph{Virtual Botany Cyanotype} by Alan Butler}

//selection of flora from GTA V

artwork text

\href{http://www.alanbutler.info/virtual-botany-cyanotypes}{More about \emph{Virtual Botany Cyanotype}}

\hypertarget{getting-there-6}{%
\subsection*{Getting There}\label{getting-there-6}}
\addcontentsline{toc}{subsection}{Getting There}

\hypertarget{readings-3}{%
\section*{Readings}\label{readings-3}}
\addcontentsline{toc}{section}{Readings}

\hypertarget{tutorial-3}{%
\section*{Tutorial}\label{tutorial-3}}
\addcontentsline{toc}{section}{Tutorial}

\hypertarget{scripting-introduction}{%
\subsection*{Scripting Introduction}\label{scripting-introduction}}
\addcontentsline{toc}{subsection}{Scripting Introduction}

\hypertarget{preparation-and-setup}{%
\subsection*{Preparation and Setup}\label{preparation-and-setup}}
\addcontentsline{toc}{subsection}{Preparation and Setup}

\begin{itemize}
\item
  Install Windows 11
\item
  Download and install \href{https://store.steampowered.com/about/}{Steam} (with a copy of GTA V or buy the game if you do not have it. GTA V is 100+ GB so it will take a few hours depending on your internet connections)
\item
  Download \href{https://www.gta5-mods.com/tools/script-hook-v}{Script Hook V}, go to the bin folder and copy \texttt{dinput8.dll} and \texttt{ScriptHookV.dll} files into your GTA V directory \texttt{C:\textbackslash{}Program\ Files\ (x86)\textbackslash{}Steam\textbackslash{}steamapps\textbackslash{}common\textbackslash{}Grand\ Theft\ Auto\ V}
\item
  Download \href{https://github.com/crosire/scripthookvdotnet/releases}{Script Hook V dot net}, copy the \texttt{ScriptHookVDotNet.asi} file, \texttt{ScriptHookVDotNet2.dll} and \texttt{ScriptHookVDotNet3.dll} files into your GTA V directory \texttt{C:\textbackslash{}Program\ Files\ (x86)\textbackslash{}Steam\textbackslash{}steamapps\textbackslash{}common\textbackslash{}Grand\ Theft\ Auto\ V}
\item
  Create a new folder in GTA V directory and call it ``scripts''.
\item
  Download and install \href{https://visualstudio.microsoft.com/vs/community/}{Visual Studio Community} (free version of VS). Open Visual Studio and check the .NET desktop development package and install it
\item
  Run GTA V and test if Script Hook V is working by pressing \texttt{F4}. This should toggle the console view. Try to type Help() and press ``ÈNTER``` to get a list of available commands.
\item
  It's also recommended to use a completed \href{https://www.gta5-mods.com/misc/100-save-game}{game save file} to skip the story mode and compulsory introductory mission. Extract the content of the file and copy both files \texttt{SGTAXXXXX} and \texttt{SGTAXXXXX.bak} in \texttt{Documents/GTA\ V/Profiles/YYYYYYYY/}. Once the files are there, you will be able to load the game state by running GTA V and navigating to GAME \textgreater{} Load Game and select the 100\% game file.
\end{itemize}

\hypertarget{creating-a-mod-file}{%
\subsubsection*{Creating a Mod File}\label{creating-a-mod-file}}
\addcontentsline{toc}{subsubsection}{Creating a Mod File}

\begin{itemize}
\item
  Open Visual Studio
\item
  Select File \textgreater{} New \textgreater{} Project
\item
  Select Visual C\# and Class Library (.NET Framework)
\item
  Give a custom file name (e.g.~moddingTutorial)
\item
  Rename public class Class1 as ``moddingTutorial'' in the right panel Solution Explorer
\item
  In the same panel go to References and click add References\ldots{} \textgreater{} Browse \textgreater{} browse to Downloads
\item
  Select ScriptHookedVDotNet \textgreater{} \texttt{ScriptHookVDotNet2.dll} and \texttt{ScriptHookVDotNet3.dll} and add them
\item
  Also add \texttt{System.Windows.forms}
\item
  Also add \texttt{System.Drawing}
\item
  In your code file add the following lines on top:
\end{itemize}

\begin{Shaded}
\begin{Highlighting}[]
\KeywordTok{using}\NormalTok{ GTA;}
\KeywordTok{using}\NormalTok{ GTA.}\FunctionTok{Math}\NormalTok{;}
\KeywordTok{using}\NormalTok{ System.}\FunctionTok{Windows}\NormalTok{.}\FunctionTok{Forms}\NormalTok{;}
\KeywordTok{using}\NormalTok{ System.}\FunctionTok{Drawing}\NormalTok{;}
\KeywordTok{using}\NormalTok{ GTA.}\FunctionTok{Native}\NormalTok{;}
\end{Highlighting}
\end{Shaded}

\begin{itemize}
\tightlist
\item
  Modify class moddingTutorial to the following:
\end{itemize}

\begin{Shaded}
\begin{Highlighting}[]
\KeywordTok{namespace}\NormalTok{ moddingTutorial}
\NormalTok{\{}
  \KeywordTok{public} \KeywordTok{class}\NormalTok{ moddingTutorial : Script}
\NormalTok{  \{}
      \KeywordTok{public} \FunctionTok{moddingTutorial}\NormalTok{()}
\NormalTok{      \{}
          \KeywordTok{this}\NormalTok{.}\FunctionTok{Tick}\NormalTok{ += onTick;}
      \KeywordTok{this}\NormalTok{.}\FunctionTok{KeyUp}\NormalTok{ += onKeyUp;}
      \KeywordTok{this}\NormalTok{.}\FunctionTok{KeyDown}\NormalTok{ += onKeyDown;}
\NormalTok{      \}}
    
      \KeywordTok{private} \DataTypeTok{void} \FunctionTok{onTick}\NormalTok{(}\DataTypeTok{object}\NormalTok{ sender, EventArgs e)}
\NormalTok{      \{}
\NormalTok{      \}}
    
      \KeywordTok{private} \DataTypeTok{void} \FunctionTok{onKeyUp}\NormalTok{(}\DataTypeTok{object}\NormalTok{ sender, KeyEventArgs e)}
\NormalTok{      \{}
\NormalTok{      \}}
 
      \KeywordTok{private} \DataTypeTok{void} \FunctionTok{onKeyDown}\NormalTok{(}\DataTypeTok{object}\NormalTok{ sender, KeyEventArgs e)}
\NormalTok{      \{}
        \KeywordTok{if}\NormalTok{ (e.}\FunctionTok{KeyCode}\NormalTok{ == Keys.}\FunctionTok{H}\NormalTok{)}
\NormalTok{        \{}
\NormalTok{              Game.}\FunctionTok{Player}\NormalTok{.}\FunctionTok{ChangeModel}\NormalTok{(PedHash.}\FunctionTok{Cat}\NormalTok{); }
\NormalTok{          \}}
\NormalTok{      \} }
\NormalTok{  \}}
\NormalTok{\}}
\end{Highlighting}
\end{Shaded}

\begin{itemize}
\item
  Save file
\item
  Go to Documents \textgreater{} Visual Studio \textgreater{} Project \textgreater{} moddingTutorial \textgreater{} moddingTutorial \textgreater{} \texttt{moddingTutorial.cs}
\item
  Copy the .cs file in the GTA V directory inside the scripts folder
\item
  Open GTA V, run the game in Story Mode (mods are only allowed in single player mode, not in GTA Online) and press `H' to see if the game turns your avatar into a cat
\item
  Note: every time you make changes to your .cs file in the scripts folder you can hit \texttt{F4} to open the console, type \texttt{Reload()} in the console for the program to reload the script and test again the changes.
\end{itemize}

\hypertarget{ontick-onkeyup-and-onkeydown}{%
\subsubsection*{onTick, onKeyUp and onKeyDown}\label{ontick-onkeyup-and-onkeydown}}
\addcontentsline{toc}{subsubsection}{onTick, onKeyUp and onKeyDown}

The main events of Script Hook V Dot Net are onTick, onKeyUp and onKeyDown. Script Hook V Dot Net will invoke your functions whenever an event is called.

The code within the onTick brackets is executed every interval milliseconds (which is by default 0), meaning that the event will be executed at every frame, for as long as the game is running.

\begin{Shaded}
\begin{Highlighting}[]
 \KeywordTok{private} \DataTypeTok{void} \FunctionTok{onTick}\NormalTok{(}\DataTypeTok{object}\NormalTok{ sender, EventArgs e)}
\NormalTok{ \{}
        \CommentTok{//code here will be executed every frame (or per usef defined interval)}
\NormalTok{ \}}
\end{Highlighting}
\end{Shaded}

If your function is written inside onKeyDown (withiin the curly brackets following onKeyUp(object sender, KeyEventArgs e)\{\}), your code will be executed every time a key is pressed. If your function is written inside onKeyUp, your code will be executed every time a key is released.

\begin{Shaded}
\begin{Highlighting}[]
\KeywordTok{private} \DataTypeTok{void} \FunctionTok{onKeyUp}\NormalTok{(}\DataTypeTok{object}\NormalTok{ sender, KeyEventArgs e)}
\NormalTok{\{}
      \CommentTok{//code here will be executed whenever a key is released}
\NormalTok{\}}

\KeywordTok{private} \DataTypeTok{void} \FunctionTok{onKeyDown}\NormalTok{(}\DataTypeTok{object}\NormalTok{ sender, KeyEventArgs e)}
\NormalTok{\{}
      \CommentTok{//code here will be executed whenever a key is pressed}
\NormalTok{\} }
\end{Highlighting}
\end{Shaded}

We can specify which code is executed based on what keys are pressed/released

\begin{Shaded}
\begin{Highlighting}[]
\KeywordTok{private} \DataTypeTok{void} \FunctionTok{onKeyDown}\NormalTok{(}\DataTypeTok{object}\NormalTok{ sender, KeyEventArgs e)}
\NormalTok{\{}
    \KeywordTok{if}\NormalTok{ (e.}\FunctionTok{KeyCode}\NormalTok{ == Keys.}\FunctionTok{H}\NormalTok{)}
\NormalTok{    \{}
        \CommentTok{//code here will be executed whenever the key 'H' is pressed }
\NormalTok{    \}}
\NormalTok{\} }
\end{Highlighting}
\end{Shaded}

\hypertarget{change-player-model}{%
\subsection*{Change Player Model}\label{change-player-model}}
\addcontentsline{toc}{subsection}{Change Player Model}

The player character is controlled as Game.Player. Game.Player can perform different functions, including changing the avatar model, and performing tasks.

Change the 3D model of your character by using the \texttt{ChangeModel} function.
The function needs a model ID, in order to load the model file of our game character.
You can browse through this list of models to find the one you want to try (note: not all models seem to load properly): \url{https://wiki.gtanet.work/index.php/Peds}

These models are all PedHashes, basically ID numbers within the PedHash group. Copy the name of the model below the image and add it to PedHash.
For example if you choose the model Poodle, you'll need to write \texttt{PedHash.Poodle}.

To change the model of your player character into a poodle you can write the following function:

\begin{Shaded}
\begin{Highlighting}[]
\NormalTok{Game.}\FunctionTok{Player}\NormalTok{.}\FunctionTok{ChangeModel}\NormalTok{(PedHash.}\FunctionTok{Poodle}\NormalTok{);}
\end{Highlighting}
\end{Shaded}

add it in your .cs file in the onKeyDown event, triggered by the pressing of the `h' key:

Example code

\begin{Shaded}
\begin{Highlighting}[]
\KeywordTok{using}\NormalTok{ System;}
\KeywordTok{using}\NormalTok{ System.}\FunctionTok{Collections}\NormalTok{.}\FunctionTok{Generic}\NormalTok{;}
\KeywordTok{using}\NormalTok{ System.}\FunctionTok{Linq}\NormalTok{;}
\KeywordTok{using}\NormalTok{ System.}\FunctionTok{Text}\NormalTok{;}
\KeywordTok{using}\NormalTok{ System.}\FunctionTok{Threading}\NormalTok{.}\FunctionTok{Tasks}\NormalTok{;}
 
\KeywordTok{using}\NormalTok{ GTA;}
\KeywordTok{using}\NormalTok{ GTA.}\FunctionTok{Math}\NormalTok{;}
\KeywordTok{using}\NormalTok{ System.}\FunctionTok{Windows}\NormalTok{.}\FunctionTok{Forms}\NormalTok{;}
\KeywordTok{using}\NormalTok{ System.}\FunctionTok{Drawing}\NormalTok{;}
\KeywordTok{using}\NormalTok{ GTA.}\FunctionTok{Native}\NormalTok{;}
 
 
\KeywordTok{namespace}\NormalTok{ moddingTutorial}
\NormalTok{\{}
    \KeywordTok{public} \KeywordTok{class}\NormalTok{ moddingTutorial : Script}
\NormalTok{    \{}
        \KeywordTok{public} \FunctionTok{moddingTutorial}\NormalTok{()}
\NormalTok{        \{}
            \KeywordTok{this}\NormalTok{.}\FunctionTok{Tick}\NormalTok{ += onTick;}
            \KeywordTok{this}\NormalTok{.}\FunctionTok{KeyUp}\NormalTok{ += onKeyUp;}
            \KeywordTok{this}\NormalTok{.}\FunctionTok{KeyDown}\NormalTok{ += onKeyDown;}
\NormalTok{        \}}
 
        \KeywordTok{private} \DataTypeTok{void} \FunctionTok{onTick}\NormalTok{(}\DataTypeTok{object}\NormalTok{ sender, EventArgs e) }\CommentTok{//this function gets executed continuously }
\NormalTok{        \{}
\NormalTok{        \}}
 
        \KeywordTok{private} \DataTypeTok{void} \FunctionTok{onKeyUp}\NormalTok{(}\DataTypeTok{object}\NormalTok{ sender, KeyEventArgs e)}\CommentTok{//everything inside here is executed only when we release a key}
\NormalTok{        \{}
\NormalTok{        \}}
 
        \KeywordTok{private} \DataTypeTok{void} \FunctionTok{onKeyDown}\NormalTok{(}\DataTypeTok{object}\NormalTok{ sender, KeyEventArgs e) }\CommentTok{//everything inside here is executed only when we press a key}
\NormalTok{        \{}
            \CommentTok{//when pressing 'H'}
            \KeywordTok{if}\NormalTok{(e.}\FunctionTok{KeyCode}\NormalTok{ == Keys.}\FunctionTok{H}\NormalTok{)}
\NormalTok{            \{}
                \CommentTok{//change player char into a different model}
\NormalTok{                Game.}\FunctionTok{Player}\NormalTok{.}\FunctionTok{ChangeModel}\NormalTok{(PedHash.}\FunctionTok{Poodle}\NormalTok{); }
\NormalTok{            \}}
\NormalTok{        \}}
\NormalTok{    \}}
\NormalTok{\}}
\end{Highlighting}
\end{Shaded}

Try to select different models and assign them to different keys to change the model of your character. Use keys that are not already implemented in the game controls to avoid clashes with built in operations.

\hypertarget{tasks}{%
\subsection*{Tasks}\label{tasks}}
\addcontentsline{toc}{subsection}{Tasks}

Our character can be controlled by our script, and given actions that override manual control of the player. These actions are called \emph{Tasks} and in order to assign tasks to our characters we have to define our \texttt{Game.Player} as \texttt{Game.Player.Character}. The \texttt{Game.Player.Character} code gets the specific model the player is controlling.

Now we can give tasks to the character by adding the \texttt{Task} function: \texttt{Game.Player.Character.Task}.

Finally we can specify what task to give the character by choosing a task from \href{https://nitanmarcel.github.io/shvdn-docs.github.io/class_g_t_a_1_1_task_invoker.html}{TaskInvoker list} of possible actions.

Jump:

\begin{Shaded}
\begin{Highlighting}[]
\NormalTok{Game.}\FunctionTok{Player}\NormalTok{.}\FunctionTok{Character}\NormalTok{.}\FunctionTok{Task}\NormalTok{.}\FunctionTok{Jump}\NormalTok{();}
\end{Highlighting}
\end{Shaded}

Wander around:

\begin{Shaded}
\begin{Highlighting}[]
\NormalTok{Game.}\FunctionTok{Player}\NormalTok{.}\FunctionTok{Character}\NormalTok{.}\FunctionTok{Task}\NormalTok{.}\FunctionTok{WanderAround}\NormalTok{();}
\end{Highlighting}
\end{Shaded}

Hands up for 3000 milliseconds:

\begin{Shaded}
\begin{Highlighting}[]
\NormalTok{Game.}\FunctionTok{Player}\NormalTok{.}\FunctionTok{Character}\NormalTok{.}\FunctionTok{Task}\NormalTok{.}\FunctionTok{HandsUp}\NormalTok{(}\DecValTok{3000}\NormalTok{);}
\end{Highlighting}
\end{Shaded}

Turn towards the camera:

\begin{Shaded}
\begin{Highlighting}[]
\NormalTok{Game.}\FunctionTok{Player}\NormalTok{.}\FunctionTok{Character}\NormalTok{.}\FunctionTok{Task}\NormalTok{.}\FunctionTok{TurnTo}\NormalTok{(GameplayCamera.}\FunctionTok{Position}\NormalTok{);}
\end{Highlighting}
\end{Shaded}

Some of the tasks are temporary and accept a time parameter (in milliseconds). Others are persistent, meaning they will keep being executed until the task is actively stopped. To stop a task you can use the \texttt{ClearAllImmediately();} command:

\begin{Shaded}
\begin{Highlighting}[]
\NormalTok{Game.}\FunctionTok{Player}\NormalTok{.}\FunctionTok{Task}\NormalTok{.}\FunctionTok{ClearAllImmediately}\NormalTok{();}
\end{Highlighting}
\end{Shaded}

\hypertarget{task-sequences}{%
\subsection*{Task Sequences}\label{task-sequences}}
\addcontentsline{toc}{subsection}{Task Sequences}

You can create sequence of multiple tasks by using \texttt{TaskSequence} and the \texttt{PerformSequence} function.
Create a new \texttt{TaskSequence} with a custom name, add tasks to it with \texttt{AddTask}, close the sequence with \texttt{Close} and then call \texttt{Task.PerformSequence} to perform the sequence.

\begin{Shaded}
\begin{Highlighting}[]
\NormalTok{TaskSequence mySeq = }\KeywordTok{new} \FunctionTok{TaskSequence}\NormalTok{();}
\NormalTok{mySeq.}\FunctionTok{AddTask}\NormalTok{.}\FunctionTok{Jump}\NormalTok{();}
\NormalTok{mySeq.}\FunctionTok{AddTask}\NormalTok{.}\FunctionTok{HandsUp}\NormalTok{(}\DecValTok{3000}\NormalTok{);}
\NormalTok{mySeq.}\FunctionTok{Close}\NormalTok{();}
                
\NormalTok{Game.}\FunctionTok{Player}\NormalTok{.}\FunctionTok{Character}\NormalTok{.}\FunctionTok{Task}\NormalTok{.}\FunctionTok{PerformSequence}\NormalTok{(mySeq);}
\end{Highlighting}
\end{Shaded}

\hypertarget{random}{%
\subsection*{Random}\label{random}}
\addcontentsline{toc}{subsection}{Random}

We can add randomness by using a randomly generated number, which makes things outside of thepredefined programme controlled by us and introduces more autonomous behaviours. We use the \texttt{Random}function to create a randomly generated number between our minimum and maximum parameter (if only one parameter is inserted, the minimum is 0).

\begin{Shaded}
\begin{Highlighting}[]
\NormalTok{Random rnd = }\KeywordTok{new} \FunctionTok{Random}\NormalTok{(); }
\DataTypeTok{int}\NormalTok{ month = rnd.}\FunctionTok{Next}\NormalTok{(}\DecValTok{1}\NormalTok{, }\DecValTok{13}\NormalTok{); }\CommentTok{// creates a number between 1 and 12 }
\DataTypeTok{int}\NormalTok{ dice = rnd.}\FunctionTok{Next}\NormalTok{(}\DecValTok{1}\NormalTok{, }\DecValTok{7}\NormalTok{); }\CommentTok{// creates a number between 1 and 6 }
\DataTypeTok{int}\NormalTok{ card = rnd.}\FunctionTok{Next}\NormalTok{(}\DecValTok{52}\NormalTok{); }\CommentTok{// creates a number between 0 and 51}
\end{Highlighting}
\end{Shaded}

Let's create a number to generate a random duration between 1 and 6 seconds, for the \texttt{HandsUp} task.

\begin{Shaded}
\begin{Highlighting}[]
\NormalTok{Random rnd = }\KeywordTok{new} \FunctionTok{Random}\NormalTok{(); }
\DataTypeTok{int}\NormalTok{ waitingTime = rnd.}\FunctionTok{Next}\NormalTok{(}\DecValTok{1}\NormalTok{, }\DecValTok{7}\NormalTok{);}
\NormalTok{Game.}\FunctionTok{Player}\NormalTok{.}\FunctionTok{Character}\NormalTok{.}\FunctionTok{Task}\NormalTok{.}\FunctionTok{HandsUp}\NormalTok{(waitingTime * }\DecValTok{1000}\NormalTok{);}
\end{Highlighting}
\end{Shaded}

\hypertarget{subtitles-and-notifications}{%
\subsection*{Subtitles and Notifications}\label{subtitles-and-notifications}}
\addcontentsline{toc}{subsection}{Subtitles and Notifications}

Generate subtitles with a custom text string and duration (in milliseconds):

\begin{Shaded}
\begin{Highlighting}[]
\NormalTok{UI.}\FunctionTok{ShowSubtitle}\NormalTok{(}\StringTok{"Hello World"}\NormalTok{, }\DecValTok{3000}\NormalTok{);}
\end{Highlighting}
\end{Shaded}

Generate a notification with a custom text string:

\begin{Shaded}
\begin{Highlighting}[]
\NormalTok{UI.}\FunctionTok{Notify}\NormalTok{(}\StringTok{"Hello World"}\NormalTok{);}
\end{Highlighting}
\end{Shaded}

\hypertarget{content-replication-assignment-3}{%
\section*{Content Replication Assignment}\label{content-replication-assignment-3}}
\addcontentsline{toc}{section}{Content Replication Assignment}

\hypertarget{deercam-reenactment}{%
\subsection*{Deercam reenactment}\label{deercam-reenactment}}
\addcontentsline{toc}{subsection}{Deercam reenactment}

Write a mod script to change your game character into a deer by pressing a key, and make it autonomously wander around Los Santos by pressing another key.

\end{document}
