% Options for packages loaded elsewhere
\PassOptionsToPackage{unicode}{hyperref}
\PassOptionsToPackage{hyphens}{url}
%
\documentclass[
  openany]{book}
\usepackage{lmodern}
\usepackage{amssymb,amsmath}
\usepackage{ifxetex,ifluatex}
\ifnum 0\ifxetex 1\fi\ifluatex 1\fi=0 % if pdftex
  \usepackage[T1]{fontenc}
  \usepackage[utf8]{inputenc}
  \usepackage{textcomp} % provide euro and other symbols
\else % if luatex or xetex
  \usepackage{unicode-math}
  \defaultfontfeatures{Scale=MatchLowercase}
  \defaultfontfeatures[\rmfamily]{Ligatures=TeX,Scale=1}
\fi
% Use upquote if available, for straight quotes in verbatim environments
\IfFileExists{upquote.sty}{\usepackage{upquote}}{}
\IfFileExists{microtype.sty}{% use microtype if available
  \usepackage[]{microtype}
  \UseMicrotypeSet[protrusion]{basicmath} % disable protrusion for tt fonts
}{}
\makeatletter
\@ifundefined{KOMAClassName}{% if non-KOMA class
  \IfFileExists{parskip.sty}{%
    \usepackage{parskip}
  }{% else
    \setlength{\parindent}{0pt}
    \setlength{\parskip}{6pt plus 2pt minus 1pt}}
}{% if KOMA class
  \KOMAoptions{parskip=half}}
\makeatother
\usepackage{xcolor}
\IfFileExists{xurl.sty}{\usepackage{xurl}}{} % add URL line breaks if available
\IfFileExists{bookmark.sty}{\usepackage{bookmark}}{\usepackage{hyperref}}
\hypersetup{
  pdftitle={The Photographer's Guide to Los Santos},
  hidelinks,
  pdfcreator={LaTeX via pandoc}}
\urlstyle{same} % disable monospaced font for URLs
\usepackage{color}
\usepackage{fancyvrb}
\newcommand{\VerbBar}{|}
\newcommand{\VERB}{\Verb[commandchars=\\\{\}]}
\DefineVerbatimEnvironment{Highlighting}{Verbatim}{commandchars=\\\{\}}
% Add ',fontsize=\small' for more characters per line
\usepackage{framed}
\definecolor{shadecolor}{RGB}{248,248,248}
\newenvironment{Shaded}{\begin{snugshade}}{\end{snugshade}}
\newcommand{\AlertTok}[1]{\textcolor[rgb]{0.94,0.16,0.16}{#1}}
\newcommand{\AnnotationTok}[1]{\textcolor[rgb]{0.56,0.35,0.01}{\textbf{\textit{#1}}}}
\newcommand{\AttributeTok}[1]{\textcolor[rgb]{0.77,0.63,0.00}{#1}}
\newcommand{\BaseNTok}[1]{\textcolor[rgb]{0.00,0.00,0.81}{#1}}
\newcommand{\BuiltInTok}[1]{#1}
\newcommand{\CharTok}[1]{\textcolor[rgb]{0.31,0.60,0.02}{#1}}
\newcommand{\CommentTok}[1]{\textcolor[rgb]{0.56,0.35,0.01}{\textit{#1}}}
\newcommand{\CommentVarTok}[1]{\textcolor[rgb]{0.56,0.35,0.01}{\textbf{\textit{#1}}}}
\newcommand{\ConstantTok}[1]{\textcolor[rgb]{0.00,0.00,0.00}{#1}}
\newcommand{\ControlFlowTok}[1]{\textcolor[rgb]{0.13,0.29,0.53}{\textbf{#1}}}
\newcommand{\DataTypeTok}[1]{\textcolor[rgb]{0.13,0.29,0.53}{#1}}
\newcommand{\DecValTok}[1]{\textcolor[rgb]{0.00,0.00,0.81}{#1}}
\newcommand{\DocumentationTok}[1]{\textcolor[rgb]{0.56,0.35,0.01}{\textbf{\textit{#1}}}}
\newcommand{\ErrorTok}[1]{\textcolor[rgb]{0.64,0.00,0.00}{\textbf{#1}}}
\newcommand{\ExtensionTok}[1]{#1}
\newcommand{\FloatTok}[1]{\textcolor[rgb]{0.00,0.00,0.81}{#1}}
\newcommand{\FunctionTok}[1]{\textcolor[rgb]{0.00,0.00,0.00}{#1}}
\newcommand{\ImportTok}[1]{#1}
\newcommand{\InformationTok}[1]{\textcolor[rgb]{0.56,0.35,0.01}{\textbf{\textit{#1}}}}
\newcommand{\KeywordTok}[1]{\textcolor[rgb]{0.13,0.29,0.53}{\textbf{#1}}}
\newcommand{\NormalTok}[1]{#1}
\newcommand{\OperatorTok}[1]{\textcolor[rgb]{0.81,0.36,0.00}{\textbf{#1}}}
\newcommand{\OtherTok}[1]{\textcolor[rgb]{0.56,0.35,0.01}{#1}}
\newcommand{\PreprocessorTok}[1]{\textcolor[rgb]{0.56,0.35,0.01}{\textit{#1}}}
\newcommand{\RegionMarkerTok}[1]{#1}
\newcommand{\SpecialCharTok}[1]{\textcolor[rgb]{0.00,0.00,0.00}{#1}}
\newcommand{\SpecialStringTok}[1]{\textcolor[rgb]{0.31,0.60,0.02}{#1}}
\newcommand{\StringTok}[1]{\textcolor[rgb]{0.31,0.60,0.02}{#1}}
\newcommand{\VariableTok}[1]{\textcolor[rgb]{0.00,0.00,0.00}{#1}}
\newcommand{\VerbatimStringTok}[1]{\textcolor[rgb]{0.31,0.60,0.02}{#1}}
\newcommand{\WarningTok}[1]{\textcolor[rgb]{0.56,0.35,0.01}{\textbf{\textit{#1}}}}
\usepackage{longtable,booktabs}
% Correct order of tables after \paragraph or \subparagraph
\usepackage{etoolbox}
\makeatletter
\patchcmd\longtable{\par}{\if@noskipsec\mbox{}\fi\par}{}{}
\makeatother
% Allow footnotes in longtable head/foot
\IfFileExists{footnotehyper.sty}{\usepackage{footnotehyper}}{\usepackage{footnote}}
\makesavenoteenv{longtable}
\usepackage{graphicx,grffile}
\makeatletter
\def\maxwidth{\ifdim\Gin@nat@width>\linewidth\linewidth\else\Gin@nat@width\fi}
\def\maxheight{\ifdim\Gin@nat@height>\textheight\textheight\else\Gin@nat@height\fi}
\makeatother
% Scale images if necessary, so that they will not overflow the page
% margins by default, and it is still possible to overwrite the defaults
% using explicit options in \includegraphics[width, height, ...]{}
\setkeys{Gin}{width=\maxwidth,height=\maxheight,keepaspectratio}
% Set default figure placement to htbp
\makeatletter
\def\fps@figure{htbp}
\makeatother
\setlength{\emergencystretch}{3em} % prevent overfull lines
\providecommand{\tightlist}{%
  \setlength{\itemsep}{0pt}\setlength{\parskip}{0pt}}
\setcounter{secnumdepth}{5}

\title{The Photographer's Guide to Los Santos}
\author{}
\date{\vspace{-2.5em}}

\begin{document}
\maketitle

{
\setcounter{tocdepth}{1}
\tableofcontents
}
\hypertarget{preface}{%
\chapter*{Preface}\label{preface}}
\addcontentsline{toc}{chapter}{Preface}

\begin{quote}
``Los Santos. The city of shitheads. Where else would he be?''

--- Trevor Philips
\end{quote}

\hypertarget{introduction}{%
\chapter{Introduction}\label{introduction}}

\hypertarget{about-the-photographers-guide-to-los-santos}{%
\section*{About The Photographer's Guide to Los Santos}\label{about-the-photographers-guide-to-los-santos}}
\addcontentsline{toc}{section}{About The Photographer's Guide to Los Santos}

The Photographer's Guide to Los Santos sits between a touristic guide and a photography manual, and between an exhibition catalogue and a peak behind the scenes of artwork creation.

The Photographer's Guide to Los Santos is an ongoing project that builds on top of a research on artistic practices within spaces of computer games, with a particular focus on in-game photography, machinima and digital visual arts. It follows some themes and ideas previously explored in the exhibition \href{https://www.howtowinat.photography/}{\emph{How to Win at Photography}}, while focusing more specifically on the relationship between computer games and photographic activities inside the world of Grand Theft Auto V.

The idea of a guide refers to in-game photography as a form of `virtual tourism' (\href{https://papers.ssrn.com/sol3/papers.cfm?abstract_id=538182}{Book, 2003}), which was also the premise of an actual tourist guide published by Rough Guides in their 2019 \href{https://www.roughguides.com/articles/introduction-to-the-rough-guide-to-xbox/}{\emph{Rough Guide to XBOX}}. Yet this guide project also understands the game world as a site for image production and artistic creation, turning the game into a destination for a `game art tourist'. The Photographer's Guide to Los Santos presents the game environment of Grand Theft Auto V both as a space to explore and in which to create images, as well as a place to navigate and learn about some of the most important artworks that it has enabled to create.

The project also brings together several experiences from teaching in-game photography as an artistic practice in different educational settings and institutions, compiling materials and tools for students and artists interested in engaging with the field. The tourist guide of the game world doubles as a photography manual for the in-game photography age, featuring tutorials ranging from game screenshotting to computer programming for creative modding. Through the practical exercises, the project invites to rethink the game object as a space for creative, subversive and critical endeavours, which can be played differently, documented, reclaimed or modified through an artistic approach.

Finally, the project draws inspiration from the works of artists who have explored the `metaplay' of photographing game words instead of following the game rules and attempt to reach the goal of winning. The Photographer's Guide to Los Santos is indebted to all the artists it features, but was particularly inspired by Gareth Damian Martin's live streamed workshop \href{https://www.twitch.tv/videos/591840067}{\emph{Photography Tour of No Man's Sky}} (realized for \href{https://www.somersethouse.org.uk/whats-on/now-play-this-2020}{Now Play This Festival 2020}), Total Refusal \& Ismaël Joffroy Chandoutis's 2021 in-game lecture performance and guided tour \href{https://vimeo.com/506064357}{\emph{Everyday Daylight}} (realized for the \href{https://ccsparis.com/en/events/total-refusal-digital-disarmament-movement-a-la-gaite-lyrique/}{CCS Paris}), and Alan Butler's epic 2020 live endurance performance \href{https://www.youtube.com/watch?v=R4Q2G6tOQ_Q}{\emph{Witness to a Changing West}} (realized for \href{https://screenwalks.com/}{Screen Walks}) and his `Content Replication Assignments'.

\hypertarget{grand-theft-auto-v-studies}{%
\section*{Grand Theft Auto V Studies}\label{grand-theft-auto-v-studies}}
\addcontentsline{toc}{section}{Grand Theft Auto V Studies}

Los Santos is the Grand Theft Auto V's fictional, parodic version of real-life Los Angeles. Just like Los Angeles is the global centre of film and commercial media production, Los Santos is the epicentre of in-game photography and machinima creation. While it may seem reductive to only focus on a single game to address the larger phenomenon of in-game photography, GTA V is the biggest source of creative outputs to date, with its extended open world and one of the largest community of active modders.

Launched in 2013, the game contains a world map of more than 80 square kilometers of total area, which includes the urban area of the city of Los Santos and the rural area of Blaine County. This incredibly vast environment features a large desert region, dense forest, several mountains, beachside towns, on top of the large metropolis of Los Santos. The game simulates the everyday life of hundreds of individual NPCs (while it allegedly counts a population of over 4 million) as well as counting 28 animal species, and \href{https://grandtheftdata.com/landmarks/\#0,0,2,satellite}{more than 800 buildings in GTA V are based on real-life landmarks}.

The size of the photorealistic simulation is only matched by the complexity of the game engine and its code, which - thanks to the effort of GTA V's modding community - allows players to use the game world as a powerful tool to create new scenes, take controls of its algorithmic entities, modify cameras and reshaping the game into a movie set or a photo studio.

\hypertarget{grand-theft-auto-v-tourism}{%
\section*{Grand Theft Auto V Tourism}\label{grand-theft-auto-v-tourism}}
\addcontentsline{toc}{section}{Grand Theft Auto V Tourism}

The project can be seen as something between a playful travel guide of Los Santos and one of the star maps offered to tourist in Hollywood, pointing to the homes of movie actors and hollywood celebrities. This guide allows players to explore the game environment following some of the most interesting artworks that have been created with(in) it. It's divided in thematic chapters that follow different artistic practices, taking place in different locations of the game environment, followed by different tutorials and exercises connected with the works and the space analyzed.

The themes explore different approaches and practices connected to established artistic and photographic currents, with a general introduction text that gives an overview of the ideas, contexts and issues connected to the specific topic. A selection of artworks for each themes is presented by a curatorial statement, introducing the work and its artistic relevance. The work is accompanied by information on the in-game location in which it was produced, inviting the readers to reach the destination in Grand Theft Auto V through maps and indications.

The game environment thus becomes the space for possible `art tours', getting insights into the artworks made in GTA V. This form of game tourism allows the player to see the behind the scenes, and experience the making of the works in its place of origin. While the complex algorithms of GTA V produce unexpected interactions and scenes, Los Santos is also stuck in the same time forever. Gas stations, shops, palm trees remain in the same state and location forever, allowing the tourist to witness the exact scene that was first encountered by the artists.

\hypertarget{grand-theft-auto-v-art-education}{%
\section*{Grand Theft Auto V Art Education}\label{grand-theft-auto-v-art-education}}
\addcontentsline{toc}{section}{Grand Theft Auto V Art Education}

This project is also an attempt to introduce a video game as a space for artistic intervention, and an invitation to use its mechanics, its code and its environment as a creative tool itself. The game can be played, documented and captured through a form of artistic play, that differs from normative gameplay and does not focus on advancinf and winning but rather engages with the game object critically. Furthermore, the game software can also be manipulated, modified and used as an apparatus to create new images and interactions. The goal of this guide is to combine a curatorial approach that leads the viewer to discover the artworks made in GTA V with a hands on approach that teaches the player the tools for possible artistic interventions in this space.

Games are often seen as producing specific cultures and shaping identities through through forms of play that follow the intentions of the developer. Here we understand games as objects to be reclaimed and tools to be deconstructed and rebuilt, both conceptually and literally. Consequently, players are not just passive actors that push buttons in the sequence that they are taught by the machine and its softwares, but open up the black box of the game and become critical thinkers and makers that actively play with the game, or even against it.

The Photographer's Guide to Los Santos can be employed as a resource to accompany workshops for students and artists approaching computer games and interested in learning how to engage with it. Each thematic chapter features a tutorial section that introduces different techniques and strategies to capture images within Grand Theft Auto V, connected to the examples and locations of each section. The chapters are thought to be experienced in order, as the tutorials at times rely on knowledge that is built on top of previous lessons. Each tutorial is accompanied by content replication assignments, in which the readers is invited to use the skills learnd from each chapter to recreate a work presented in that section. Tutorials are intended for anyone who might be playing GTA V for the first time in their life and do not assume any previous experience, although some basic idea of programming is helpful when dealing with scripting and modding the game.

\hypertarget{architecture-photography}{%
\chapter{Architecture Photography}\label{architecture-photography}}

//general intro on architecture in game spaces, the contruction of virtual cities, architecture photography and how it relates to the game environment, the player as a photographer documenting urban spaces\ldots{}

\hypertarget{the-continuous-city-by-gareth-damian-martin}{%
\section*{\texorpdfstring{\emph{The Continuous City}, by Gareth Damian Martin}{The Continuous City, by Gareth Damian Martin}}\label{the-continuous-city-by-gareth-damian-martin}}
\addcontentsline{toc}{section}{\emph{The Continuous City}, by Gareth Damian Martin}

Gareth Damian Martin, \emph{Outskirts}, from \emph{The Continuous City},

Gareth Damian Martin, \emph{Pathways}, from \emph{The Continuous City},

artwork text

\href{https://socks-studio.com/2019/10/13/gareth-damian-martin-postcards-from-the-continuous-city-2018/}{More about \emph{The Continuous City}}

\href{https://www.gamescenes.org/2018/04/interview-gareth-damian-martin-the-aesthetics-of-analogue-game-photography.html}{Interview with Gareth Damian Martin}

\hypertarget{getting-there}{%
\subsection*{Getting there}\label{getting-there}}
\addcontentsline{toc}{subsection}{Getting there}

\begin{itemize}
\tightlist
\item
  \href{https://grandtheftdata.com/landmarks/\#951.507,-1144.265,4,atlas,name=trainyard_warehouse,Trainyard_Warehouse,_East_Los_Santos}{The intersection of Interstate 4 and Interstate 5} manifests the architecture of traffic of the megalopolis.
\end{itemize}

\hypertarget{readings}{%
\section*{Readings}\label{readings}}
\addcontentsline{toc}{section}{Readings}

\href{https://www.heterotopiaszine.com/}{Heterotopias}

Mark D Teo, The Urban Architecture of Los Angeles and Grand Theft Auto, 2015. \url{https://www.academia.edu/18173221/The_Urban_Architecture_of_Los_Angeles_and_Grand_Theft_Auto}

\hypertarget{tutorial}{%
\section*{Tutorial}\label{tutorial}}
\addcontentsline{toc}{section}{Tutorial}

\hypertarget{photographing-the-game-screen}{%
\subsection*{Photographing the Game Screen}\label{photographing-the-game-screen}}
\addcontentsline{toc}{subsection}{Photographing the Game Screen}

\hypertarget{analogue-game-photography}{%
\subsubsection*{Analogue Game Photography}\label{analogue-game-photography}}
\addcontentsline{toc}{subsubsection}{Analogue Game Photography}

\hypertarget{screenshotting}{%
\subsubsection*{Screenshotting}\label{screenshotting}}
\addcontentsline{toc}{subsubsection}{Screenshotting}

\hypertarget{content-replication-assignment}{%
\section*{Content Replication Assignment}\label{content-replication-assignment}}
\addcontentsline{toc}{section}{Content Replication Assignment}

\hypertarget{social-documentary}{%
\chapter{Social Documentary}\label{social-documentary}}

//general intro on simulating society, the creation of NPCs, documentary and street photography traditions connected to politics of visibility and representation, and how they relate to the politics of simulation, how the player-photographer documents the creation of complex social spaces and reveals the process of simulating people and issues of class, gender, race in the game space\ldots{}

\hypertarget{down-and-out-in-los-santos-by-alan-butler}{%
\section*{\texorpdfstring{\emph{Down and Out in Los Santos} by Alan Butler}{Down and Out in Los Santos by Alan Butler}}\label{down-and-out-in-los-santos-by-alan-butler}}
\addcontentsline{toc}{section}{\emph{Down and Out in Los Santos} by Alan Butler}

artwork text

\href{http://www.alanbutler.info/down-and-out-in-los-santos-2016}{More about \emph{Down and Out in Los Santos}}

\hypertarget{getting-there-1}{%
\subsection*{Getting There}\label{getting-there-1}}
\addcontentsline{toc}{subsection}{Getting There}

The homeless camp in Los Santos is under the \href{https://grandtheftdata.com/landmarks/\#252.086,-1208.975,5,hybrid,name,Strawberry_Subway_Station,_Downtown}{Olympic Freeway in Strawberry}.

Dignity Village is a tent city established by homeless people near Procopio Beach, east of \href{https://grandtheftdata.com/landmarks/\#-171.667,6208.676,5,hybrid,name=paleto_bay,Belinda_May's_Beauty_Salon,_Paleto_Blvd,_Paleto_Bay}{Paleto Bay}.

\hypertarget{fear-and-loathing-in-gta-v-by-morten-rockford-ravn}{%
\section*{\texorpdfstring{\emph{Fear and Loathing in GTA V} by Morten Rockford Ravn}{Fear and Loathing in GTA V by Morten Rockford Ravn}}\label{fear-and-loathing-in-gta-v-by-morten-rockford-ravn}}
\addcontentsline{toc}{section}{\emph{Fear and Loathing in GTA V} by Morten Rockford Ravn}

artwork text

\href{https://fearandloathingingtav.tumblr.com/}{More about \emph{Fear and Loathing in GTA V}}

\hypertarget{getting-there-2}{%
\subsection*{Getting There}\label{getting-there-2}}
\addcontentsline{toc}{subsection}{Getting There}

\hypertarget{readings-1}{%
\section*{Readings}\label{readings-1}}
\addcontentsline{toc}{section}{Readings}

\hypertarget{tutorial-1}{%
\section*{Tutorial}\label{tutorial-1}}
\addcontentsline{toc}{section}{Tutorial}

\hypertarget{in-game-smartphone-camera}{%
\subsection*{In-game Smartphone Camera}\label{in-game-smartphone-camera}}
\addcontentsline{toc}{subsection}{In-game Smartphone Camera}

Snapmatic is the photo app on your simulated mobile phone in GTA V.

\begin{itemize}
\item
  Press \texttt{UP} on the d-pad to bring up your phone.
\item
  Select the Snapmatic app - it's on the bottom left of the homescreen.
\item
  You can shuffle through filters with \texttt{DOWN} on the d-pad or borders with \texttt{UP} on the d-pad.
\item
  Move the camera with the \texttt{RIGHT\ STICK} and zoom in and out with the \texttt{LEFT\ STICK}.
\item
  You're also able to concentrate focus and depth of field.
\item
  To take selfie press the \texttt{R3\ STICK} to turn the camera on yourself.
\item
  \texttt{L3} will let you pull different facial expressions.
\item
  \texttt{LEFT} on the d-pad will let you strike more of a pose (this changes depending on the character).
\item
  Once you're happy, take the photo with \texttt{X} on the PS4 and \texttt{A} on the Xbox One and save it to the Gallery
\end{itemize}

\hypertarget{content-replication-assignment-1}{%
\section*{Content Replication Assignment}\label{content-replication-assignment-1}}
\addcontentsline{toc}{section}{Content Replication Assignment}

\hypertarget{re-enactment-photography}{%
\chapter{Re-enactment Photography}\label{re-enactment-photography}}

//general introduction on the development of photorealism in games, the relationship between photography and CGI, the remediation of photographic images and the analog apparatus, the player as photographer situated in the tradition of conceptual photographers like Sherrie Levine and Sturtevant, the copy as a conceptual approach that create new meaning through a similar image but a different context\ldots{}

\hypertarget{gasoline-stations-in-gta-v-by-lorna-ruth-galloway}{%
\section*{\texorpdfstring{\emph{26 Gasoline stations in GTA V} by Lorna Ruth Galloway}{26 Gasoline stations in GTA V by Lorna Ruth Galloway}}\label{gasoline-stations-in-gta-v-by-lorna-ruth-galloway}}
\addcontentsline{toc}{section}{\emph{26 Gasoline stations in GTA V} by Lorna Ruth Galloway}

artwork text

\href{https://www.lornaruthgalloway.com/charcoal-halftone}{More about \emph{26 Gasoline stations in GTA V}}

\hypertarget{getting-there-3}{%
\subsection*{Getting There}\label{getting-there-3}}
\addcontentsline{toc}{subsection}{Getting There}

\begin{itemize}
\tightlist
\item
  \href{https://grandtheftdata.com/landmarks/\#513.325,-1374.453,4,atlas,name=gas_station,Globe_Oil_Gas_Station,_Innocence_Blvd_\&_Alta_St,_South_Los_Santos}{Globe Oil Gas Station, Innocence Blvd \& Alta St, South Los Santos}
\item
  \href{https://grandtheftdata.com/landmarks/\#513.325,-1374.453,4,atlas,name=gas_station,LTD_Gas_Station,_Davis_Ave_\&_Grove_St,_South_Los_Santos}{LTD Gas Station, Davis Ave \& Grove St, South Los Santos}
\item
  \href{https://grandtheftdata.com/landmarks/\#378.149,-934.191,4,atlas,name=gas_station,LTD_Gas_Station,_Mirror_Park_Blvd_\&_W_Mirror_Dr,_Mirror_Park}{LTD Gas Station, Mirror Park Blvd \& W Mirror Dr, Mirror Park}
\item
  \href{https://grandtheftdata.com/landmarks/\#613.567,192.271,6,atlas,name=gas_station,Globe_Oil_Gas_Station,_Clinton_Ave_\&_Fenwell_Pl,_Vinewood_Hills}{Globe Oil Gas Station, Clinton Ave \& Fenwell Pl, Vinewood Hills}
\item
  \href{https://grandtheftdata.com/landmarks/\#513.325,-1374.453,4,atlas,name=gas_station,Xero_Gas_Station,_Strawberry_Ave_\&_Capital_Blvd,_South_Los_Santos}{Xero Gas Station, Strawberry Ave \& Capital Blvd, South Los Santos}
\item
  \href{https://grandtheftdata.com/landmarks/\#513.325,-1374.453,4,atlas,name=gas_station,RON_Gas_Station,_Davis_Ave_\&_Macdonald_St,_South_Los_Santos}{RON Gas Station, Davis Ave \& Macdonald St, South Los Santos}
\item
  \href{https://grandtheftdata.com/landmarks/\#513.325,-1374.453,4,atlas,name=gas_station,Xero_Gas_Station,_Calais_Ave_\&_Innocence_Blvd,_Little_Seoul}{Xero Gas Station, Calais Ave \& Innocence Blvd, Little Seoul}
\item
  \href{https://grandtheftdata.com/landmarks/\#513.325,-1374.453,4,atlas,name=gas_station,LTD_Gas_Station,_Lindsay_Circus_\&_Ginger_St,_Little_Seoul}{LTD Gas Station, Lindsay Circus \& Ginger St, Little Seoul}
\item
  \href{https://grandtheftdata.com/landmarks/\#-1543.17,-676.107,4,atlas,name=gas_station,RON_Gas_Station,_N_Rockford_Dr_\&_Perth_St,_Morningwood}{RON Gas Station, N Rockford Dr \& Perth St, Morningwood}
\item
  \href{https://grandtheftdata.com/landmarks/\#-1543.17,-676.107,4,atlas,name=gas_station,Xero_Gas_Station,_Great_Ocean_Hwy,_Pacific_Bluffs}{Xero Gas Station, Great Ocean Hwy, Pacific Bluffs}
\end{itemize}

\hypertarget{a-study-on-perspective-by-roc-herms}{%
\section*{\texorpdfstring{\emph{A Study on Perspective} by Roc Herms}{A Study on Perspective by Roc Herms}}\label{a-study-on-perspective-by-roc-herms}}
\addcontentsline{toc}{section}{\emph{A Study on Perspective} by Roc Herms}

artwork text

\href{https://www.rocherms.com/projects/study-of-perspective/}{More about \emph{A Study on Perspective}}

\hypertarget{getting-there-4}{%
\subsection*{Getting There}\label{getting-there-4}}
\addcontentsline{toc}{subsection}{Getting There}

\href{https://grandtheftdata.com/landmarks/\#1016.057,312.205,4,atlas,name=vinewood,Vinewood_Sign,_Vinewood_Hills}{Vinewood Sign, Vinewood Hills}

\hypertarget{further-references}{%
\section*{Further references}\label{further-references}}
\addcontentsline{toc}{section}{Further references}

\hypertarget{little-books-of-los-santos-by-luke-caspar-pearson}{%
\subsection*{\texorpdfstring{\emph{Little Books of Los Santos} by Luke Caspar Pearson}{Little Books of Los Santos by Luke Caspar Pearson}}\label{little-books-of-los-santos-by-luke-caspar-pearson}}
\addcontentsline{toc}{subsection}{\emph{Little Books of Los Santos} by Luke Caspar Pearson}

artwork text

\href{https://www.alephograph.com/little-books-of-los-santos}{More about \emph{Little Books of Los Santos}}

\hypertarget{gasoline-stations-in-gta-v-by-m.-earl-williams}{%
\subsection*{\texorpdfstring{\emph{26 Gasoline stations in GTA V} by M. Earl Williams}{26 Gasoline stations in GTA V by M. Earl Williams}}\label{gasoline-stations-in-gta-v-by-m.-earl-williams}}
\addcontentsline{toc}{subsection}{\emph{26 Gasoline stations in GTA V} by M. Earl Williams}

artwork text

\href{https://www.mearlwilliams.com/gasoline_stations\#1}{More about \emph{26 Gasoline stations in GTA V}}

\hypertarget{readings-2}{%
\section*{Readings}\label{readings-2}}
\addcontentsline{toc}{section}{Readings}

\hypertarget{tutorial-2}{%
\section*{Tutorial}\label{tutorial-2}}
\addcontentsline{toc}{section}{Tutorial}

\hypertarget{scene-director-mode}{%
\subsection*{Scene Director Mode}\label{scene-director-mode}}
\addcontentsline{toc}{subsection}{Scene Director Mode}

\hypertarget{content-replication-assignment-2}{%
\section*{Content Replication Assignment}\label{content-replication-assignment-2}}
\addcontentsline{toc}{section}{Content Replication Assignment}

\hypertarget{nature-documentary}{%
\chapter{Nature Documentary}\label{nature-documentary}}

//general introduction about the creation of a synthetic forms of nature, ecological issues, creation of virtual sublime, flora and fauna that are usually props that become the focus of the player's explorations, ``virtual world naturalism''\ldots{}

\hypertarget{san-andreas-streaming-deer-cam-by-brent-watanabe}{%
\section*{\texorpdfstring{\emph{San Andreas Streaming Deer Cam} by Brent Watanabe}{San Andreas Streaming Deer Cam by Brent Watanabe}}\label{san-andreas-streaming-deer-cam-by-brent-watanabe}}
\addcontentsline{toc}{section}{\emph{San Andreas Streaming Deer Cam} by Brent Watanabe}

artwork text

\href{https://bwatanabe.com/GTA_V_WanderingDeer.html}{More about \emph{Deercam}}

\hypertarget{getting-there-5}{%
\subsection*{Getting There}\label{getting-there-5}}
\addcontentsline{toc}{subsection}{Getting There}

\href{https://grandtheftdata.com/landmarks/\#86.534,6158.577,4,atlas,name=mount_chiliad,Mount_Chiliad}{Mount Chiliad} is located in the Chiliad Mountain State Wilderness, and it is the tallest mountain in the game at 798m above sea level. The state park is home to lots of wildlife such as deer and mountain lions.

\hypertarget{virtual-botany-cyanotype-by-alan-butler}{%
\section*{\texorpdfstring{\emph{Virtual Botany Cyanotype} by Alan Butler}{Virtual Botany Cyanotype by Alan Butler}}\label{virtual-botany-cyanotype-by-alan-butler}}
\addcontentsline{toc}{section}{\emph{Virtual Botany Cyanotype} by Alan Butler}

//selectioon of flora from GTA V

artwork text

\href{http://www.alanbutler.info/virtual-botany-cyanotypes}{More about \emph{Virtual Botany Cyanotype}}

\hypertarget{getting-there-6}{%
\subsection*{Getting There}\label{getting-there-6}}
\addcontentsline{toc}{subsection}{Getting There}

\hypertarget{readings-3}{%
\section*{Readings}\label{readings-3}}
\addcontentsline{toc}{section}{Readings}

\hypertarget{tutorial-3}{%
\section*{Tutorial}\label{tutorial-3}}
\addcontentsline{toc}{section}{Tutorial}

\hypertarget{scripting-introduction}{%
\subsection*{Scripting Introduction}\label{scripting-introduction}}
\addcontentsline{toc}{subsection}{Scripting Introduction}

\hypertarget{preparation-and-setup}{%
\subsection*{Preparation and Setup}\label{preparation-and-setup}}
\addcontentsline{toc}{subsection}{Preparation and Setup}

\begin{itemize}
\item
  Install Windows 11
\item
  Download and install \href{https://store.steampowered.com/about/}{Steam} (with a copy of GTA V or buy the game if you do not have it. GTA V is 100+ GB so it will take a few hours depending on your internet connections)
\item
  Download \href{https://store.steampowered.com/about/}{Script Hook V}, go to the bin folder and copy \texttt{dinput8.dll} and \texttt{ScriptHookV.dll} files into your GTA V directory \texttt{C:\textbackslash{}Program\ Files\ (x86)\textbackslash{}Steam\textbackslash{}steamapps\textbackslash{}common\textbackslash{}Grand\ Theft\ Auto\ V}
\item
  Download \href{https://store.steampowered.com/about/}{Script Hook V dot net}, copy the \texttt{ScriptHookVDotNet.asi} file, \texttt{ScriptHookVDotNet2.dll} and \texttt{ScriptHookVDotNet3.dll} files into your GTA V directory \texttt{C:\textbackslash{}Program\ Files\ (x86)\textbackslash{}Steam\textbackslash{}steamapps\textbackslash{}common\textbackslash{}Grand\ Theft\ Auto\ V}
\item
  Create a new folder in GTA V directory and call it ``scripts''.
\item
  Download and install \href{https://store.steampowered.com/about/}{Visual Studio Community} (free version of VS). Open Visual Studio and check the .NET desktop development package and install it
\item
  Run GTA V and test if Script Hook V is working by pressing \texttt{F4}. This should toggle the console view.
\end{itemize}

\hypertarget{creating-a-mod-file}{%
\subsubsection*{Creating a Mod File}\label{creating-a-mod-file}}
\addcontentsline{toc}{subsubsection}{Creating a Mod File}

\begin{itemize}
\item
  Open Visual Studio
\item
  Select File \textgreater{} New \textgreater{} Project
\item
  Select Visual C\# and Class Library (.NET Framework)
\item
  Give a custom file name (e.g.~moddingTutorial)
\item
  Rename public class Class1 as ``moddingTutorial'' in the right panel Solution Explorer
\item
  In the same panel go to References and click add References\ldots{} \textgreater{} Browse \textgreater{} browse to Downloads
\item
  Select ScriptHookedVDotNet \textgreater{} \texttt{ScriptHookVDotNet2.dll} and \texttt{ScriptHookVDotNet3.dll} and add them
\item
  Also add \texttt{System.Windows.forms}
\item
  Also add \texttt{System.Drawing}
\item
  In your code file add the following lines on top:
\end{itemize}

\begin{Shaded}
\begin{Highlighting}[]
\KeywordTok{using}\NormalTok{ GTA;}
\KeywordTok{using}\NormalTok{ GTA.}\FunctionTok{Math}\NormalTok{;}
\KeywordTok{using}\NormalTok{ System.}\FunctionTok{Windows}\NormalTok{.}\FunctionTok{Forms}\NormalTok{;}
\KeywordTok{using}\NormalTok{ System.}\FunctionTok{Drawing}\NormalTok{;}
\KeywordTok{using}\NormalTok{ GTA.}\FunctionTok{Native}\NormalTok{;}
\end{Highlighting}
\end{Shaded}

\begin{itemize}
\tightlist
\item
  Modify class moddingTutorial to the following:
\end{itemize}

\begin{Shaded}
\begin{Highlighting}[]
\KeywordTok{namespace}\NormalTok{ moddingTutorial}
\NormalTok{\{}
  \KeywordTok{public} \KeywordTok{class}\NormalTok{ moddingTutorial : Script}
\NormalTok{  \{}
      \KeywordTok{public} \FunctionTok{moddingTutorial}\NormalTok{()}
\NormalTok{      \{}
          \KeywordTok{this}\NormalTok{.}\FunctionTok{Tick}\NormalTok{ += onTick;}
      \KeywordTok{this}\NormalTok{.}\FunctionTok{KeyUp}\NormalTok{ += onKeyUp;}
      \KeywordTok{this}\NormalTok{.}\FunctionTok{KeyDown}\NormalTok{ += onKeyDown;}
\NormalTok{      \}}
    
      \KeywordTok{private} \DataTypeTok{void} \FunctionTok{onTick}\NormalTok{(}\DataTypeTok{object}\NormalTok{ sender, EventArgs e)}
\NormalTok{      \{}
\NormalTok{      \}}
    
      \KeywordTok{private} \DataTypeTok{void} \FunctionTok{onKeyUp}\NormalTok{(}\DataTypeTok{object}\NormalTok{ sender, KeyEventArgs e)}
\NormalTok{      \{}
\NormalTok{      \}}
 
      \KeywordTok{private} \DataTypeTok{void} \FunctionTok{onKeyDown}\NormalTok{(}\DataTypeTok{object}\NormalTok{ sender, KeyEventArgs e)}
\NormalTok{      \{}
        \KeywordTok{if}\NormalTok{ (e.}\FunctionTok{KeyCode}\NormalTok{ == Keys.}\FunctionTok{H}\NormalTok{)}
\NormalTok{        \{}
\NormalTok{              Game.}\FunctionTok{Player}\NormalTok{.}\FunctionTok{ChangeModel}\NormalTok{(PedHash.}\FunctionTok{Cat}\NormalTok{); }
\NormalTok{          \}}
\NormalTok{      \} }
\NormalTok{  \}}
\NormalTok{\}}
\end{Highlighting}
\end{Shaded}

\begin{itemize}
\item
  Save file
\item
  Go to Documents \textgreater{} Visual Studio \textgreater{} Project \textgreater{} moddingTutorial \textgreater{} moddingTutorial \textgreater{} \texttt{moddingTutorial.cs}
\item
  Copy the .cs file in the GTA V directory inside the scripts folder
\item
  Open GTA V, run the game in Story Mode (mods are only allowed in single player mode, not in GTA Online) and press `H' to see if the game turns your avatar into a cat
\item
  Note: every time you make changes to your .cs file in the scripts folder you can hit \texttt{F4} to open the console, type \texttt{Reload()} in the console for the program to reload the script and test again the changes.
\end{itemize}

\hypertarget{ontick-onkeyup-and-onkeydown}{%
\subsubsection*{onTick, onKeyUp and onKeyDown}\label{ontick-onkeyup-and-onkeydown}}
\addcontentsline{toc}{subsubsection}{onTick, onKeyUp and onKeyDown}

The main events of Script Hook V Dot Net are onTick, onKeyUp and onKeyDown. Script Hook V Dot Net will invoke your functions whenever an event is called.

The code within the onTick brackets is executed every interval milliseconds (which is by default 0), meaning that the event will be executed at every frame, for as long as the game is running.

\begin{Shaded}
\begin{Highlighting}[]
 \KeywordTok{private} \DataTypeTok{void} \FunctionTok{onTick}\NormalTok{(}\DataTypeTok{object}\NormalTok{ sender, EventArgs e)}
\NormalTok{ \{}
        \CommentTok{//code here will be executed every frame (or per usef defined interval)}
\NormalTok{ \}}
\end{Highlighting}
\end{Shaded}

If your function is written inside onKeyDown (withiin the curly brackets following onKeyUp(object sender, KeyEventArgs e)\{\}), your code will be executed every time a key is pressed. If your function is written inside onKeyUp, your code will be executed every time a key is released.

\begin{Shaded}
\begin{Highlighting}[]
\KeywordTok{private} \DataTypeTok{void} \FunctionTok{onKeyUp}\NormalTok{(}\DataTypeTok{object}\NormalTok{ sender, KeyEventArgs e)}
\NormalTok{\{}
      \CommentTok{//code here will be executed whenever a key is released}
\NormalTok{\}}

\KeywordTok{private} \DataTypeTok{void} \FunctionTok{onKeyDown}\NormalTok{(}\DataTypeTok{object}\NormalTok{ sender, KeyEventArgs e)}
\NormalTok{\{}
      \CommentTok{//code here will be executed whenever a key is pressed}
\NormalTok{\} }
\end{Highlighting}
\end{Shaded}

We can specify which code is executed based on what keys are pressed/released

\begin{Shaded}
\begin{Highlighting}[]
\KeywordTok{private} \DataTypeTok{void} \FunctionTok{onKeyDown}\NormalTok{(}\DataTypeTok{object}\NormalTok{ sender, KeyEventArgs e)}
\NormalTok{\{}
    \KeywordTok{if}\NormalTok{ (e.}\FunctionTok{KeyCode}\NormalTok{ == Keys.}\FunctionTok{H}\NormalTok{)}
\NormalTok{    \{}
        \CommentTok{//code here will be executed whenever the key 'H' is pressed }
\NormalTok{    \}}
\NormalTok{\} }
\end{Highlighting}
\end{Shaded}

\hypertarget{change-player-model}{%
\subsection*{Change Player Model}\label{change-player-model}}
\addcontentsline{toc}{subsection}{Change Player Model}

The player character is controlled as Game.Player. Game.Player can perform different functions, including changing the avatar model, and performing tasks.

Change the 3D model of your character by using the \texttt{ChangeModel} function.
The function needs a model ID, in order to load the model file of our game character.
You can browse through this list of models to find the one you want to try: \url{https://wiki.gtanet.work/index.php/Peds}

These models are all PedHashes, basically ID numbers within the PedHash group. Copy the name of the model below the image and add it to PedHash.
For example if you choose the model Cat, you'll need to write \texttt{PedHash.Cat}.

To change the model of your player character into a cat you can write the following function:

\begin{Shaded}
\begin{Highlighting}[]
\NormalTok{Game.}\FunctionTok{Player}\NormalTok{.}\FunctionTok{ChangeModel}\NormalTok{(PedHash.}\FunctionTok{Cat}\NormalTok{);}
\end{Highlighting}
\end{Shaded}

add it in your .cs file in the onKeyDown event, triggered by the pressing of the `h' key:

Example code

\begin{Shaded}
\begin{Highlighting}[]
\KeywordTok{using}\NormalTok{ System;}
\KeywordTok{using}\NormalTok{ System.}\FunctionTok{Collections}\NormalTok{.}\FunctionTok{Generic}\NormalTok{;}
\KeywordTok{using}\NormalTok{ System.}\FunctionTok{Linq}\NormalTok{;}
\KeywordTok{using}\NormalTok{ System.}\FunctionTok{Text}\NormalTok{;}
\KeywordTok{using}\NormalTok{ System.}\FunctionTok{Threading}\NormalTok{.}\FunctionTok{Tasks}\NormalTok{;}
 
\KeywordTok{using}\NormalTok{ GTA;}
\KeywordTok{using}\NormalTok{ GTA.}\FunctionTok{Math}\NormalTok{;}
\KeywordTok{using}\NormalTok{ System.}\FunctionTok{Windows}\NormalTok{.}\FunctionTok{Forms}\NormalTok{;}
\KeywordTok{using}\NormalTok{ System.}\FunctionTok{Drawing}\NormalTok{;}
\KeywordTok{using}\NormalTok{ GTA.}\FunctionTok{Native}\NormalTok{;}
 
 
\KeywordTok{namespace}\NormalTok{ moddingTutorial}
\NormalTok{\{}
    \KeywordTok{public} \KeywordTok{class}\NormalTok{ moddingTutorial : Script}
\NormalTok{    \{}
        \KeywordTok{public} \FunctionTok{moddingTutorial}\NormalTok{()}
\NormalTok{        \{}
            \KeywordTok{this}\NormalTok{.}\FunctionTok{Tick}\NormalTok{ += onTick;}
            \KeywordTok{this}\NormalTok{.}\FunctionTok{KeyUp}\NormalTok{ += onKeyUp;}
            \KeywordTok{this}\NormalTok{.}\FunctionTok{KeyDown}\NormalTok{ += onKeyDown;}
\NormalTok{        \}}
 
        \KeywordTok{private} \DataTypeTok{void} \FunctionTok{onTick}\NormalTok{(}\DataTypeTok{object}\NormalTok{ sender, EventArgs e) }\CommentTok{//this function gets executed continuously }
\NormalTok{        \{}
\NormalTok{        \}}
 
        \KeywordTok{private} \DataTypeTok{void} \FunctionTok{onKeyUp}\NormalTok{(}\DataTypeTok{object}\NormalTok{ sender, KeyEventArgs e)}\CommentTok{//everything inside here is executed only when we release a key}
\NormalTok{        \{}
\NormalTok{        \}}
 
        \KeywordTok{private} \DataTypeTok{void} \FunctionTok{onKeyDown}\NormalTok{(}\DataTypeTok{object}\NormalTok{ sender, KeyEventArgs e) }\CommentTok{//everything inside here is executed only when we press a key}
\NormalTok{        \{}
            \CommentTok{//when pressing 'H'}
            \KeywordTok{if}\NormalTok{(e.}\FunctionTok{KeyCode}\NormalTok{ == Keys.}\FunctionTok{H}\NormalTok{)}
\NormalTok{            \{}
                \CommentTok{//change player char into a different model}
\NormalTok{                Game.}\FunctionTok{Player}\NormalTok{.}\FunctionTok{ChangeModel}\NormalTok{(PedHash.}\FunctionTok{Cat}\NormalTok{); }
\NormalTok{            \}}
\NormalTok{        \}}
\NormalTok{    \}}
\NormalTok{\}}
\end{Highlighting}
\end{Shaded}

\hypertarget{tasks}{%
\subsection*{Tasks}\label{tasks}}
\addcontentsline{toc}{subsection}{Tasks}

Our character can be controlled by our script, and given actions that override manual control of the player. These actions are called \emph{Tasks} and in order to assign tasks to our characters we have to define our \texttt{Game.Player} as \texttt{Game.Player.Character}. The \texttt{Game.Player.Character} code gets the specific model the player is controlling.

Now we can give tasks to the character by adding the \texttt{Task} function: \texttt{Game.Player.Character.Task}.

Finally we can specify what task to give the character by choosing a task from \href{https://nitanmarcel.github.io/shvdn-docs.github.io/class_g_t_a_1_1_task_invoker.html}{TaskInvoker list} of possible actions.

Jump:

\begin{Shaded}
\begin{Highlighting}[]
\NormalTok{Game.}\FunctionTok{Player}\NormalTok{.}\FunctionTok{Character}\NormalTok{.}\FunctionTok{Task}\NormalTok{.}\FunctionTok{Jump}\NormalTok{();}
\end{Highlighting}
\end{Shaded}

Wander around:

\begin{Shaded}
\begin{Highlighting}[]
\NormalTok{Game.}\FunctionTok{Player}\NormalTok{.}\FunctionTok{Character}\NormalTok{.}\FunctionTok{Task}\NormalTok{.}\FunctionTok{WanderAround}\NormalTok{();}
\end{Highlighting}
\end{Shaded}

Hands up for 3000 milliseconds:

\begin{Shaded}
\begin{Highlighting}[]
\NormalTok{Game.}\FunctionTok{Player}\NormalTok{.}\FunctionTok{Character}\NormalTok{.}\FunctionTok{Task}\NormalTok{.}\FunctionTok{HandsUp}\NormalTok{(}\DecValTok{3000}\NormalTok{);}
\end{Highlighting}
\end{Shaded}

\hypertarget{task-sequences}{%
\subsection*{Task Sequences}\label{task-sequences}}
\addcontentsline{toc}{subsection}{Task Sequences}

You can create sequence of multiple tasks by using \texttt{TaskSequence} and the \texttt{PerformSequence} function.
Create a new \texttt{TaskSequence} with a custom name, add tasks to it with \texttt{AddTask}, close the sequence with \texttt{Close} and then call \texttt{Task.PerformSequence} to perform the sequence.

\begin{Shaded}
\begin{Highlighting}[]
\NormalTok{TaskSequence mySeq = }\KeywordTok{new} \FunctionTok{TaskSequence}\NormalTok{();}
\NormalTok{mySeq.}\FunctionTok{AddTask}\NormalTok{.}\FunctionTok{Jump}\NormalTok{();}
\NormalTok{mySeq.}\FunctionTok{AddTask}\NormalTok{.}\FunctionTok{HandsUp}\NormalTok{(}\DecValTok{3000}\NormalTok{);}
\NormalTok{mySeq.}\FunctionTok{Close}\NormalTok{();}
                
\NormalTok{Game.}\FunctionTok{Player}\NormalTok{.}\FunctionTok{Character}\NormalTok{.}\FunctionTok{Task}\NormalTok{.}\FunctionTok{PerformSequence}\NormalTok{(mySeq);}
\end{Highlighting}
\end{Shaded}

\hypertarget{random}{%
\subsection*{Random}\label{random}}
\addcontentsline{toc}{subsection}{Random}

We can add randomness by using a randomly generated number, which makes things outside of thepredefined programme controlled by us and introduces more autonomous behaviours. We use the \texttt{Random}function to create a randomly generated number between our minimum and maximum parameter (if only one parameter is inserted, the minimum is 0).

\begin{Shaded}
\begin{Highlighting}[]
\NormalTok{Random rnd = }\KeywordTok{new} \FunctionTok{Random}\NormalTok{(); }
\DataTypeTok{int}\NormalTok{ month = rnd.}\FunctionTok{Next}\NormalTok{(}\DecValTok{1}\NormalTok{, }\DecValTok{13}\NormalTok{); }\CommentTok{// creates a number between 1 and 12 }
\DataTypeTok{int}\NormalTok{ dice = rnd.}\FunctionTok{Next}\NormalTok{(}\DecValTok{1}\NormalTok{, }\DecValTok{7}\NormalTok{); }\CommentTok{// creates a number between 1 and 6 }
\DataTypeTok{int}\NormalTok{ card = rnd.}\FunctionTok{Next}\NormalTok{(}\DecValTok{52}\NormalTok{); }\CommentTok{// creates a number between 0 and 51}
\end{Highlighting}
\end{Shaded}

Let's create a number to generate a random duration between 1 and 6 seconds, for the \texttt{HandsUp} task.

\begin{Shaded}
\begin{Highlighting}[]
\NormalTok{Random rnd = }\KeywordTok{new} \FunctionTok{Random}\NormalTok{(); }
\DataTypeTok{int}\NormalTok{ waitingTime = rnd.}\FunctionTok{Next}\NormalTok{(}\DecValTok{1}\NormalTok{, }\DecValTok{7}\NormalTok{);}
\NormalTok{Game.}\FunctionTok{Player}\NormalTok{.}\FunctionTok{Character}\NormalTok{.}\FunctionTok{Task}\NormalTok{.}\FunctionTok{HandsUp}\NormalTok{(waitingTime * }\DecValTok{1000}\NormalTok{);}
\end{Highlighting}
\end{Shaded}

\hypertarget{subtitles-and-notifications}{%
\subsection*{Subtitles and Notifications}\label{subtitles-and-notifications}}
\addcontentsline{toc}{subsection}{Subtitles and Notifications}

Generate subtitles with a custom text string and duration (in milliseconds):

\begin{Shaded}
\begin{Highlighting}[]
\NormalTok{UI.}\FunctionTok{ShowSubtitle}\NormalTok{(}\StringTok{"Hello World"}\NormalTok{, }\DecValTok{3000}\NormalTok{);}
\end{Highlighting}
\end{Shaded}

Generate a notification with a custom text string:

\begin{Shaded}
\begin{Highlighting}[]
\NormalTok{UI.}\FunctionTok{Notify}\NormalTok{(}\StringTok{"Hello World"}\NormalTok{);}
\end{Highlighting}
\end{Shaded}

\hypertarget{content-replication-assignment-3}{%
\section*{Content Replication Assignment}\label{content-replication-assignment-3}}
\addcontentsline{toc}{section}{Content Replication Assignment}

\hypertarget{deercam-reenactment}{%
\subsection*{Deercam reenactment}\label{deercam-reenactment}}
\addcontentsline{toc}{subsection}{Deercam reenactment}

Write a mod script to change your game character into a deer by pressing a key, and make it autonomously wander around Los Santos by pressing another key.

\hypertarget{surrealist-photography}{%
\chapter{Surrealist Photography}\label{surrealist-photography}}

//general introduction about avant garde traditions of distancing from reality and exploring the possibilities of CGI decoupled from realism adn life-like simulation, the game as an engine that can be used to create oniric scenes, which in turn reveal the untapped possibilities hidden within the game code, the player as a modder which can generate worlds within the world\ldots{}

\hypertarget{alexey-andrienko-aka-happ-v2}{%
\section*{Alexey Andrienko aka HAPP v2}\label{alexey-andrienko-aka-happ-v2}}
\addcontentsline{toc}{section}{Alexey Andrienko aka HAPP v2}

artwork text

\href{https://www.gamescenes.org/2018/02/game-art-happ-v2s-in-game-photography.html}{More about Happ v2}

\hypertarget{getting-there-7}{%
\subsection*{Getting There}\label{getting-there-7}}
\addcontentsline{toc}{subsection}{Getting There}

\href{https://grandtheftdata.com/landmarks/\#-2629.132,626.461,4,atlas,name=beach,Chumash_Beach,_Chumash}{Chumash Beach}

\hypertarget{readings-4}{%
\section*{Readings}\label{readings-4}}
\addcontentsline{toc}{section}{Readings}

\hypertarget{tutorial-4}{%
\section*{Tutorial}\label{tutorial-4}}
\addcontentsline{toc}{section}{Tutorial}

\hypertarget{scripting-characters}{%
\subsection*{Scripting Characters}\label{scripting-characters}}
\addcontentsline{toc}{subsection}{Scripting Characters}

\hypertarget{npcs}{%
\subsection*{NPCs}\label{npcs}}
\addcontentsline{toc}{subsection}{NPCs}

NPCs are non playable characters and in GTA V scripting they are called \texttt{Peds}. Peds are an entity like Props or Vehicles and can be created, assigned different model textures, equipped with weapons and controlled through different tasks.

\hypertarget{spawn-a-new-npc}{%
\subsection*{Spawn a new NPC}\label{spawn-a-new-npc}}
\addcontentsline{toc}{subsection}{Spawn a new NPC}

A GTA V Ped can be created by the \texttt{World.CreatePed} function. This takes two parameters: an ID to assign the 3D model and textures, and the location where the Ped is created.

The model IDs are the same we used in the previous tutorial, when we changed our character's appearance to a cat. A list of all available models can be found \href{https://wiki.gtanet.work/index.php/Peds}{here}. \texttt{PedHash.Cat}, \texttt{PedHash.Deer}, \texttt{PedHash.AviSchwartzman}are all possible IDs we can assign to the NPC we want to create.
We can create a new model variable, which we will name `myPedModel' and assign it a model ID:

\begin{Shaded}
\begin{Highlighting}[]
\NormalTok{Models myPedModel = }\StringTok{"PedHash.AviSchwartzman"}\NormalTok{;}
\end{Highlighting}
\end{Shaded}

The location where the NPC is created through a vector3 data type, which represents a vector in 3D space. This basically means a point that contains X, Y and Z coordinates. We can give absolute coordinates, making the Ped appear at a specific location in the game, but we can also use a location relative to our position in the game. In order not to risk making a Ped appear somewhere completely outside of our view -- on some mountain or in the sea -- let's look at a vector3 that points to a position in front of the player.

We want to establish the player with\texttt{Game.Player.Character}, followed by a function that retireve the player position within the game world. That's called by using \texttt{GetOffsetInWorldCoords}, which takes a vector3. The values of the X, Y and Z of the vector 3 offset the location based on the origin point represented by the player. Therefore, we can move the place where we want the Ped to appear by adding values to the X axis (left or right of player), Y axis (ahead or behind the player), and Z axis (above or below the player).
To make a Ped appear in front of the player we can create a vector3 data type with 0 for X, 5 for Y and 0 for Z: \texttt{new\ Vector3(0,\ 5,\ 0)}. Let's make a vector3 variable, which we will name `myPedSpawnPosition', assign it the values above for X, Y and Z coordinates from the player position.

\begin{Shaded}
\begin{Highlighting}[]
\NormalTok{Vector3 myPedSpawnPosition = Game.}\FunctionTok{Player}\NormalTok{.}\FunctionTok{Character}\NormalTok{.}\FunctionTok{GetOffsetInWorldCoords}\NormalTok{(}\KeywordTok{new} \FunctionTok{Vector3}\NormalTok{(}\DecValTok{0}\NormalTok{, }\DecValTok{5}\NormalTok{, }\DecValTok{0}\NormalTok{));}
\end{Highlighting}
\end{Shaded}

Now we can use the model and the position variables to spawn the NPC in front of the player. We'll create a Ped named `myPed1' and use the \texttt{World.CreatePed} function with the two variables as parameters:

\begin{Shaded}
\begin{Highlighting}[]
\DataTypeTok{var}\NormalTok{ myPed1 = World.}\FunctionTok{CreatePed}\NormalTok{(myPedModel, myPedSpawnPosition); }
\end{Highlighting}
\end{Shaded}

Example code

\begin{Shaded}
\begin{Highlighting}[]
\KeywordTok{using}\NormalTok{ System;}
\KeywordTok{using}\NormalTok{ System.}\FunctionTok{Collections}\NormalTok{.}\FunctionTok{Generic}\NormalTok{;}
\KeywordTok{using}\NormalTok{ System.}\FunctionTok{Linq}\NormalTok{;}
\KeywordTok{using}\NormalTok{ System.}\FunctionTok{Text}\NormalTok{;}
\KeywordTok{using}\NormalTok{ System.}\FunctionTok{Threading}\NormalTok{.}\FunctionTok{Tasks}\NormalTok{;}
 
\KeywordTok{using}\NormalTok{ GTA;}
\KeywordTok{using}\NormalTok{ GTA.}\FunctionTok{Math}\NormalTok{;}
\KeywordTok{using}\NormalTok{ System.}\FunctionTok{Windows}\NormalTok{.}\FunctionTok{Forms}\NormalTok{;}
\KeywordTok{using}\NormalTok{ System.}\FunctionTok{Drawing}\NormalTok{;}
\KeywordTok{using}\NormalTok{ GTA.}\FunctionTok{Native}\NormalTok{;}
 
 
\KeywordTok{namespace}\NormalTok{ moddingTutorial}
\NormalTok{\{}
    \KeywordTok{public} \KeywordTok{class}\NormalTok{ moddingTutorial : Script}
\NormalTok{    \{}
        \KeywordTok{public} \FunctionTok{moddingTutorial}\NormalTok{()}
\NormalTok{        \{}
            \KeywordTok{this}\NormalTok{.}\FunctionTok{Tick}\NormalTok{ += onTick;}
            \KeywordTok{this}\NormalTok{.}\FunctionTok{KeyUp}\NormalTok{ += onKeyUp;}
            \KeywordTok{this}\NormalTok{.}\FunctionTok{KeyDown}\NormalTok{ += onKeyDown;}
\NormalTok{        \}}
 
        \KeywordTok{private} \DataTypeTok{void} \FunctionTok{onTick}\NormalTok{(}\DataTypeTok{object}\NormalTok{ sender, EventArgs e) }\CommentTok{//this function gets executed continuously }
\NormalTok{        \{}
\NormalTok{        \}}
 
        \KeywordTok{private} \DataTypeTok{void} \FunctionTok{onKeyUp}\NormalTok{(}\DataTypeTok{object}\NormalTok{ sender, KeyEventArgs e)}\CommentTok{//everything inside here is executed only when we release a key}
\NormalTok{        \{}
\NormalTok{        \}}
 
        \KeywordTok{private} \DataTypeTok{void} \FunctionTok{onKeyDown}\NormalTok{(}\DataTypeTok{object}\NormalTok{ sender, KeyEventArgs e) }\CommentTok{//everything inside here is executed only when we press a key}
\NormalTok{        \{}
            \CommentTok{//when pressing 'K'}
            \KeywordTok{if}\NormalTok{(e.}\FunctionTok{KeyCode}\NormalTok{ == Keys.}\FunctionTok{K}\NormalTok{)}
\NormalTok{            \{}
                \CommentTok{//select a model and store it in a variable}
\NormalTok{                Models myPedModel = }\StringTok{"PedHash.AviSchwartzman"}\NormalTok{;}
                    \CommentTok{//create a position relative to the player}
\NormalTok{                    Vector3 myPedSpawnPosition = Game.}\FunctionTok{Player}\NormalTok{.}\FunctionTok{Character}\NormalTok{.}\FunctionTok{GetOffsetInWorldCoords}\NormalTok{(}\KeywordTok{new} \FunctionTok{Vector3}\NormalTok{(}\DecValTok{0}\NormalTok{, }\DecValTok{5}\NormalTok{, }\DecValTok{0}\NormalTok{));}
                    \CommentTok{//create a Ped with the chosen model, spawning at the chosen position}
                    \DataTypeTok{var}\NormalTok{ myPed1 = World.}\FunctionTok{CreatePed}\NormalTok{(myPedModel, myPedSpawnPosition); }
\NormalTok{            \}}
\NormalTok{        \}}
\NormalTok{    \}}
\NormalTok{\}}
\end{Highlighting}
\end{Shaded}

\hypertarget{control-multiple-npcs}{%
\subsection*{Control Multiple NPCs}\label{control-multiple-npcs}}
\addcontentsline{toc}{subsection}{Control Multiple NPCs}

You can create multiple NPCs and give them custom names. Let's create a human NPC and a cat NPC and call them Jim and MannyTheCat respectively:

\begin{Shaded}
\begin{Highlighting}[]
\DataTypeTok{var}\NormalTok{ Jim = World.}\FunctionTok{CreatePed}\NormalTok{(PedHash.}\FunctionTok{AviSchwartzman}\NormalTok{, Game.}\FunctionTok{Player}\NormalTok{.}\FunctionTok{Character}\NormalTok{.}\FunctionTok{GetOffsetInWorldCoords}\NormalTok{(}\KeywordTok{new} \FunctionTok{Vector3}\NormalTok{(}\DecValTok{0}\NormalTok{, }\DecValTok{5}\NormalTok{, }\DecValTok{0}\NormalTok{)));}
\DataTypeTok{var}\NormalTok{ MannyTheCat = World.}\FunctionTok{CreatePed}\NormalTok{(PedHash.}\FunctionTok{Cat}\NormalTok{, Game.}\FunctionTok{Player}\NormalTok{.}\FunctionTok{Character}\NormalTok{.}\FunctionTok{GetOffsetInWorldCoords}\NormalTok{(}\KeywordTok{new} \FunctionTok{Vector3}\NormalTok{(}\DecValTok{0}\NormalTok{, }\DecValTok{3}\NormalTok{, }\DecValTok{0}\NormalTok{)));}
\end{Highlighting}
\end{Shaded}

Try to kill one of the \texttt{Ped} NPCs you created by using the \texttt{Kill()}.

\begin{Shaded}
\begin{Highlighting}[]
\NormalTok{Jim.}\FunctionTok{Kill}\NormalTok{();}
\end{Highlighting}
\end{Shaded}

Note that when you kill your \texttt{Ped} `Jim', it falls on the floor and it won't actually respond to any call or task you will give it, but it's not removed from the game. To remove a specific \texttt{Ped} you not use the \texttt{Delete}function, which will remove that instance (and will make the NPC disappear).

\begin{Shaded}
\begin{Highlighting}[]
\NormalTok{Jim.}\FunctionTok{Delete}\NormalTok{();}
\end{Highlighting}
\end{Shaded}

To handle groups of NPCs we can use the \texttt{List} class. A \texttt{List} is a collection of objects, and a \texttt{List}of \texttt{Peds} allows us to store our NPCs. We can use an index to retrieve and control specific \texttt{Peds} in the group. ou can see the \href{https://learn.microsoft.com/en-us/dotnet/api/system.collections.generic.list-1?view=net-7.0}{reference} for more detailed information.

Create a \texttt{List} of \texttt{Peds} named myPeds as a global variable in the public class \texttt{public\ class\ moddingTutorial\ :\ Script}.

\begin{Shaded}
\begin{Highlighting}[]
\NormalTok{List<Ped> myPeds = }\KeywordTok{new}\NormalTok{ List<Ped>();}
\end{Highlighting}
\end{Shaded}

In the onKeyDown function \texttt{private\ void\ onKeyDown(object\ sender,\ KeyEventArgs\ e)} create 5 new Peds with a \href{https://www.w3schools.com/cs/cs_for_loop.php}{For Loop}

\begin{Shaded}
\begin{Highlighting}[]
\KeywordTok{for}\NormalTok{ (}\DataTypeTok{int}\NormalTok{ i = }\DecValTok{0}\NormalTok{; i < }\DecValTok{5}\NormalTok{; i++)}
\NormalTok{\{}
    \CommentTok{//spawn a new Ped called newPed}
    \DataTypeTok{var}\NormalTok{ newPed = World.}\FunctionTok{CreatePed}\NormalTok{(PedHash.}\FunctionTok{AviSchwartzman}\NormalTok{, Game.}\FunctionTok{Player}\NormalTok{.}\FunctionTok{Character}\NormalTok{.}\FunctionTok{GetOffsetInWorldCoords}\NormalTok{(}\KeywordTok{new} \FunctionTok{Vector3}\NormalTok{(i, }\DecValTok{3}\NormalTok{, }\DecValTok{0}\NormalTok{)));}
    \CommentTok{//add the new Ped to my list of Peds myPeds}
\NormalTok{    myPeds.}\FunctionTok{Add}\NormalTok{(newPed);}
\NormalTok{\}}
\end{Highlighting}
\end{Shaded}

Now all the 5 \texttt{Peds} are part of the myPeds{[}{]} \texttt{List}. You can control each Ped individually by calling their individual number ID in the group. The first spawn \texttt{Ped} is myPed{[}0{]}, the last one is myPeds{[}4{]}.

Kill the 3rd spawned NPC:

\begin{Shaded}
\begin{Highlighting}[]
\NormalTok{myPeds[}\DecValTok{2}\NormalTok{].}\FunctionTok{Kill}\NormalTok{();}
\end{Highlighting}
\end{Shaded}

\hypertarget{nearby-npcs}{%
\subsection*{Nearby NPCs}\label{nearby-npcs}}
\addcontentsline{toc}{subsection}{Nearby NPCs}

Script Hook V DOt Net provides a function \texttt{GetNearbyPeds}which groups all the \texttt{Peds} within a nearby radius from a character.

Create a new group that adds \texttt{Peds} which are closer than 20 meters from the player:

\begin{Shaded}
\begin{Highlighting}[]
\NormalTok{Ped[] NearbyPeds = World.}\FunctionTok{GetNearbyPeds}\NormalTok{(Game.}\FunctionTok{Player}\NormalTok{.}\FunctionTok{Character}\NormalTok{, 20f);}
\end{Highlighting}
\end{Shaded}

Use a \href{https://www.simplilearn.com/tutorials/asp-dot-net-tutorial/csharp-foreach\#:~:text=The\%20foreach\%20loop\%20in\%20C\%23,readable\%20alternative\%20to\%20for\%20loop.}{Foreach Loop} to get every \texttt{Ped} in the group and kill them:

\begin{Shaded}
\begin{Highlighting}[]
\KeywordTok{foreach}\NormalTok{ (Ped p }\KeywordTok{in}\NormalTok{ NearbyPeds)}
\NormalTok{\{}
\NormalTok{    p.}\FunctionTok{Kill}\NormalTok{();}
\NormalTok{\}}
\end{Highlighting}
\end{Shaded}

\texttt{GetNearbyPeds} does not sort out individual \texttt{Peds}in the group based on distance, so we have to do a bit of manual filtering to get the nearest NPC within the chosen radius from the player character.

Define the global variables in the public class \texttt{public\ class\ moddingTutorial\ :\ Script}:

\begin{Shaded}
\begin{Highlighting}[]
\DataTypeTok{float}\NormalTok{ lastDistance;}
\NormalTok{Ped Squad1Leader = }\KeywordTok{null}\NormalTok{;}
\NormalTok{Ped oldSquad1Leader = }\KeywordTok{null}\NormalTok{;}
\end{Highlighting}
\end{Shaded}

Get and parse the nearby NPCs in the OnTick function \texttt{private\ void\ onTick(object\ sender,\ EventArgs\ e)}:

\begin{Shaded}
\begin{Highlighting}[]
\CommentTok{//set radius}
\DataTypeTok{float}\NormalTok{ maxDistance = 25f;}
\CommentTok{//get nearest peds      }
\NormalTok{Ped[] pedsGroup = World.}\FunctionTok{GetNearbyPeds}\NormalTok{(Game.}\FunctionTok{Player}\NormalTok{.}\FunctionTok{Character}\NormalTok{, maxDistance);}
 
\DataTypeTok{float}\NormalTok{ lastDistance = maxDistance;}
\KeywordTok{foreach}\NormalTok{ (Ped ped }\KeywordTok{in}\NormalTok{ pedsGroup)}
\NormalTok{\{}
    \DataTypeTok{float}\NormalTok{ distance = ped.}\FunctionTok{Position}\NormalTok{.}\FunctionTok{DistanceTo}\NormalTok{(Game.}\FunctionTok{Player}\NormalTok{.}\FunctionTok{Character}\NormalTok{.}\FunctionTok{Position}\NormalTok{);}
    \KeywordTok{if}\NormalTok{ (distance < lastDistance)}
\NormalTok{    \{}
\NormalTok{        Squad1Leader = ped;}
\NormalTok{        lastDistance = distance;}
\NormalTok{    \}}
\NormalTok{\}}

    \KeywordTok{if}\NormalTok{ (Squad1Leader != }\KeywordTok{null}\NormalTok{ && oldSquad1Leader != Squad1Leader)}
\NormalTok{    \{}
\NormalTok{        Squad1Leader.}\FunctionTok{Kill}\NormalTok{();}
\NormalTok{    \}}
\NormalTok{oldSquad1Leader = Squad1Leader;}
\end{Highlighting}
\end{Shaded}

\hypertarget{give-tasks-to-npcs}{%
\subsection*{Give Tasks to NPCs}\label{give-tasks-to-npcs}}
\addcontentsline{toc}{subsection}{Give Tasks to NPCs}

A Ped can be given a task using the \texttt{Task} function, just like we did in the previous tutorial for the player character.

\begin{Shaded}
\begin{Highlighting}[]
\NormalTok{myPed1.}\FunctionTok{Task}\NormalTok{.}\FunctionTok{WanderAround}\NormalTok{();}
\end{Highlighting}
\end{Shaded}

Some tasks involve interacting with other characters (Peds or Game.Player.Character) or take different parameters like positions (vector3), duration (in milliseconds), and other data types.
We can give our NPC the task to fight against the player by using the \texttt{FightAgainst} function, which requires a Ped parameter -- which in the case of the player is expressed as \texttt{Game.Player.Character}.

\begin{Shaded}
\begin{Highlighting}[]
\NormalTok{myPed1.}\FunctionTok{Task}\NormalTok{.}\FunctionTok{FightAgainst}\NormalTok{(Game.}\FunctionTok{Player}\NormalTok{.}\FunctionTok{Character}\NormalTok{); }\CommentTok{//give npc task to fight against player}
\end{Highlighting}
\end{Shaded}

Try to replace the task to ``fight against'' with ``flee from (player)'' , ``hands up'', ``jump''\ldots{} or some of the other available tasks.

See the \href{https://nitanmarcel.github.io/shvdn-docs.github.io/class_g_t_a_1_1_task_invoker.html}{TaskInvoker list} for possible tasks, or click on the list of available tasks below.

List of Available Tasks

\begin{Shaded}
\begin{Highlighting}[]
\DataTypeTok{void} \FunctionTok{AchieveHeading}\NormalTok{ (}\DataTypeTok{float}\NormalTok{ heading, }\DataTypeTok{int}\NormalTok{ timeout=}\DecValTok{0}\NormalTok{)}
 
\DataTypeTok{void} \FunctionTok{AimAt}\NormalTok{ (Entity target, }\DataTypeTok{int}\NormalTok{ duration)}
 
\DataTypeTok{void} \FunctionTok{AimAt}\NormalTok{ (Vector3 target, }\DataTypeTok{int}\NormalTok{ duration)}
 
\DataTypeTok{void} \FunctionTok{Arrest}\NormalTok{ (Ped ped)}
 
\DataTypeTok{void} \FunctionTok{ChatTo}\NormalTok{ (Ped ped)}
 
\DataTypeTok{void} \FunctionTok{Jump}\NormalTok{ ()}
 
\DataTypeTok{void} \FunctionTok{Climb}\NormalTok{ ()}
 
\DataTypeTok{void} \FunctionTok{ClimbLadder}\NormalTok{ ()}
 
\DataTypeTok{void} \FunctionTok{Cower}\NormalTok{ (}\DataTypeTok{int}\NormalTok{ duration)}
 
\DataTypeTok{void} \FunctionTok{ChaseWithGroundVehicle}\NormalTok{ (Ped target)}
 
\DataTypeTok{void} \FunctionTok{ChaseWithHelicopter}\NormalTok{ (Ped target, Vector3 offset)}
 
\DataTypeTok{void} \FunctionTok{ChaseWithPlane}\NormalTok{ (Ped target, Vector3 offset)}
 
\DataTypeTok{void} \FunctionTok{CruiseWithVehicle}\NormalTok{ (Vehicle vehicle, }\DataTypeTok{float}\NormalTok{ speed, DrivingStyle style=DrivingStyle.}\FunctionTok{Normal}\NormalTok{)}
 
\DataTypeTok{void} \FunctionTok{DriveTo}\NormalTok{ (Vehicle vehicle, Vector3 target, }\DataTypeTok{float}\NormalTok{ radius, }\DataTypeTok{float}\NormalTok{ speed, DrivingStyle style=DrivingStyle.}\FunctionTok{Normal}\NormalTok{)}
 
\DataTypeTok{void} \FunctionTok{EnterAnyVehicle}\NormalTok{ (VehicleSeat seat=VehicleSeat.}\FunctionTok{Any}\NormalTok{, }\DataTypeTok{int}\NormalTok{ timeout=-}\DecValTok{1}\NormalTok{, }\DataTypeTok{float}\NormalTok{ speed=1f, EnterVehicleFlags flag=EnterVehicleFlags.}\FunctionTok{None}\NormalTok{)}
 
\DataTypeTok{void} \FunctionTok{EnterVehicle}\NormalTok{ (Vehicle vehicle, VehicleSeat seat=VehicleSeat.}\FunctionTok{Any}\NormalTok{, }\DataTypeTok{int}\NormalTok{ timeout=-}\DecValTok{1}\NormalTok{, }\DataTypeTok{float}\NormalTok{ speed=1f, EnterVehicleFlags flag=EnterVehicleFlags.}\FunctionTok{None}\NormalTok{)}
 
\DataTypeTok{void} \FunctionTok{FightAgainst}\NormalTok{ (Ped target)}
 
\DataTypeTok{void} \FunctionTok{FightAgainst}\NormalTok{ (Ped target, }\DataTypeTok{int}\NormalTok{ duration)}
 
\DataTypeTok{void} \FunctionTok{FightAgainstHatedTargets}\NormalTok{ (}\DataTypeTok{float}\NormalTok{ radius)}
 
\DataTypeTok{void} \FunctionTok{FightAgainstHatedTargets}\NormalTok{ (}\DataTypeTok{float}\NormalTok{ radius, }\DataTypeTok{int}\NormalTok{ duration)}
 
\DataTypeTok{void} \FunctionTok{FleeFrom}\NormalTok{ (Ped ped, }\DataTypeTok{int}\NormalTok{ duration=-}\DecValTok{1}\NormalTok{)}
 
\DataTypeTok{void} \FunctionTok{FleeFrom}\NormalTok{ (Vector3 position, }\DataTypeTok{int}\NormalTok{ duration=-}\DecValTok{1}\NormalTok{)}
 
\DataTypeTok{void} \FunctionTok{FollowPointRoute}\NormalTok{ (}\KeywordTok{params}\NormalTok{ Vector3[] points)}
 
\DataTypeTok{void} \FunctionTok{FollowPointRoute}\NormalTok{ (}\DataTypeTok{float}\NormalTok{ movementSpeed, }\KeywordTok{params}\NormalTok{ Vector3[] points)}
 
\DataTypeTok{void} \FunctionTok{FollowToOffsetFromEntity}\NormalTok{ (Entity target, Vector3 offset, }\DataTypeTok{float}\NormalTok{ movementSpeed, }\DataTypeTok{int}\NormalTok{ timeout=-}\DecValTok{1}\NormalTok{, }\DataTypeTok{float}\NormalTok{ distanceToFollow=10f, }\DataTypeTok{bool}\NormalTok{ persistFollowing=}\KeywordTok{true}\NormalTok{)}
 
\DataTypeTok{void} \FunctionTok{GoTo}\NormalTok{ (Entity target, Vector3 offset=}\KeywordTok{default}\NormalTok{(Vector3), }\DataTypeTok{int}\NormalTok{ timeout=-}\DecValTok{1}\NormalTok{)}
 
\DataTypeTok{void} \FunctionTok{GoTo}\NormalTok{ (Vector3 position, }\DataTypeTok{int}\NormalTok{ timeout=-}\DecValTok{1}\NormalTok{)}
 
\DataTypeTok{void} \FunctionTok{GoStraightTo}\NormalTok{ (Vector3 position, }\DataTypeTok{int}\NormalTok{ timeout=-}\DecValTok{1}\NormalTok{, }\DataTypeTok{float}\NormalTok{ targetHeading=0f, }\DataTypeTok{float}\NormalTok{ distanceToSlide=0f)}
 
\DataTypeTok{void} \FunctionTok{GuardCurrentPosition}\NormalTok{ ()}
 
\DataTypeTok{void} \FunctionTok{HandsUp}\NormalTok{ (}\DataTypeTok{int}\NormalTok{ duration)}
 
\DataTypeTok{void} \FunctionTok{LandPlane}\NormalTok{ (Vector3 startPosition, Vector3 touchdownPosition, Vehicle plane=}\KeywordTok{null}\NormalTok{)}
 
\DataTypeTok{void} \FunctionTok{LeaveVehicle}\NormalTok{ (LeaveVehicleFlags flags=LeaveVehicleFlags.}\FunctionTok{None}\NormalTok{)}
 
\DataTypeTok{void} \FunctionTok{LeaveVehicle}\NormalTok{ (Vehicle vehicle, }\DataTypeTok{bool}\NormalTok{ closeDoor)}
 
\DataTypeTok{void} \FunctionTok{LeaveVehicle}\NormalTok{ (Vehicle vehicle, LeaveVehicleFlags flags)}
 
\DataTypeTok{void} \FunctionTok{LookAt}\NormalTok{ (Entity target, }\DataTypeTok{int}\NormalTok{ duration=-}\DecValTok{1}\NormalTok{)}

\DataTypeTok{void} \FunctionTok{LookAt}\NormalTok{ (Vector3 position, }\DataTypeTok{int}\NormalTok{ duration=-}\DecValTok{1}\NormalTok{)}
  
\DataTypeTok{void} \FunctionTok{ParachuteTo}\NormalTok{ (Vector3 position)}
 
\DataTypeTok{void} \FunctionTok{ParkVehicle}\NormalTok{ (Vehicle vehicle, Vector3 position, }\DataTypeTok{float}\NormalTok{ heading, }\DataTypeTok{float}\NormalTok{ radius=}\FloatTok{20.0f}\NormalTok{, }\DataTypeTok{bool}\NormalTok{ keepEngineOn=}\KeywordTok{false}\NormalTok{)}
 
\DataTypeTok{void} \FunctionTok{PerformSequence}\NormalTok{ (TaskSequence sequence)}
 
\DataTypeTok{void} \FunctionTok{PlayAnimation}\NormalTok{ (}\DataTypeTok{string}\NormalTok{ animDict, }\DataTypeTok{string}\NormalTok{ animName)}
 
\DataTypeTok{void} \FunctionTok{PlayAnimation}\NormalTok{ (}\DataTypeTok{string}\NormalTok{ animDict, }\DataTypeTok{string}\NormalTok{ animName, }\DataTypeTok{float}\NormalTok{ speed, }\DataTypeTok{int}\NormalTok{ duration, }\DataTypeTok{float}\NormalTok{ playbackRate)}
 
\DataTypeTok{void} \FunctionTok{PlayAnimation}\NormalTok{ (}\DataTypeTok{string}\NormalTok{ animDict, }\DataTypeTok{string}\NormalTok{ animName, }\DataTypeTok{float}\NormalTok{ blendInSpeed, }\DataTypeTok{int}\NormalTok{ duration, AnimationFlags flags)}
 
\DataTypeTok{void} \FunctionTok{PlayAnimation}\NormalTok{ (}\DataTypeTok{string}\NormalTok{ animDict, }\DataTypeTok{string}\NormalTok{ animName, }\DataTypeTok{float}\NormalTok{ blendInSpeed, }\DataTypeTok{float}\NormalTok{ blendOutSpeed, }\DataTypeTok{int}\NormalTok{ duration, AnimationFlags flags, }\DataTypeTok{float}\NormalTok{ playbackRate)}
 
\DataTypeTok{void} \FunctionTok{RappelFromHelicopter}\NormalTok{ ()}
 
\DataTypeTok{void} \FunctionTok{ReactAndFlee}\NormalTok{ (Ped ped)}
 
\DataTypeTok{void} \FunctionTok{ReloadWeapon}\NormalTok{ ()}
 
\DataTypeTok{void} \FunctionTok{RunTo}\NormalTok{ (Vector3 position, }\DataTypeTok{bool}\NormalTok{ ignorePaths=}\KeywordTok{false}\NormalTok{, }\DataTypeTok{int}\NormalTok{ timeout=-}\DecValTok{1}\NormalTok{)}
 
\DataTypeTok{void} \FunctionTok{ShootAt}\NormalTok{ (Ped target, }\DataTypeTok{int}\NormalTok{ duration=-}\DecValTok{1}\NormalTok{, FiringPattern pattern=FiringPattern.}\FunctionTok{Default}\NormalTok{)}
 
\DataTypeTok{void} \FunctionTok{ShootAt}\NormalTok{ (Vector3 position, }\DataTypeTok{int}\NormalTok{ duration=-}\DecValTok{1}\NormalTok{, FiringPattern pattern=FiringPattern.}\FunctionTok{Default}\NormalTok{)}
 
\DataTypeTok{void} \FunctionTok{ShuffleToNextVehicleSeat}\NormalTok{ (Vehicle vehicle=}\KeywordTok{null}\NormalTok{)}
 
\DataTypeTok{void} \FunctionTok{Skydive}\NormalTok{ ()}
 
\DataTypeTok{void} \FunctionTok{SlideTo}\NormalTok{ (Vector3 position, }\DataTypeTok{float}\NormalTok{ heading)}
 
\DataTypeTok{void} \FunctionTok{StandStill}\NormalTok{ (}\DataTypeTok{int}\NormalTok{ duration)}
 
\DataTypeTok{void} \FunctionTok{StartScenario}\NormalTok{ (}\DataTypeTok{string}\NormalTok{ name, }\DataTypeTok{float}\NormalTok{ heading)}
 
\DataTypeTok{void} \FunctionTok{StartScenario}\NormalTok{ (}\DataTypeTok{string}\NormalTok{ name, Vector3 position, }\DataTypeTok{float}\NormalTok{ heading)}
 
\DataTypeTok{void} \FunctionTok{SwapWeapon}\NormalTok{ ()}
 
\DataTypeTok{void} \FunctionTok{TurnTo}\NormalTok{ (Entity target, }\DataTypeTok{int}\NormalTok{ duration=-}\DecValTok{1}\NormalTok{)}
 
\DataTypeTok{void} \FunctionTok{TurnTo}\NormalTok{ (Vector3 position, }\DataTypeTok{int}\NormalTok{ duration=-}\DecValTok{1}\NormalTok{)}
 
\DataTypeTok{void} \FunctionTok{UseParachute}\NormalTok{ ()}
 
\DataTypeTok{void} \FunctionTok{UseMobilePhone}\NormalTok{ ()}
 
\DataTypeTok{void} \FunctionTok{UseMobilePhone}\NormalTok{ (}\DataTypeTok{int}\NormalTok{ duration)}
 
\DataTypeTok{void} \FunctionTok{PutAwayParachute}\NormalTok{ ()}
 
\DataTypeTok{void} \FunctionTok{PutAwayMobilePhone}\NormalTok{ ()}
 
\DataTypeTok{void} \FunctionTok{VehicleChase}\NormalTok{ (Ped target)}
 
\DataTypeTok{void} \FunctionTok{VehicleShootAtPed}\NormalTok{ (Ped target)}
 
\DataTypeTok{void} \FunctionTok{Wait}\NormalTok{ (}\DataTypeTok{int}\NormalTok{ duration)}
 
\DataTypeTok{void} \FunctionTok{WanderAround}\NormalTok{ ()}
 
\DataTypeTok{void} \FunctionTok{WanderAround}\NormalTok{ (Vector3 position, }\DataTypeTok{float}\NormalTok{ radius)}
 
\DataTypeTok{void} \FunctionTok{WarpIntoVehicle}\NormalTok{ (Vehicle vehicle, VehicleSeat seat)}
 
\DataTypeTok{void} \FunctionTok{WarpOutOfVehicle}\NormalTok{ (Vehicle vehicle)}
 
\DataTypeTok{void} \FunctionTok{ClearAll}\NormalTok{ ()}
 
\DataTypeTok{void} \FunctionTok{ClearAllImmediately}\NormalTok{ ()}
 
\DataTypeTok{void} \FunctionTok{ClearLookAt}\NormalTok{ ()}
 
\DataTypeTok{void} \FunctionTok{ClearSecondary}\NormalTok{ ()}
 
\DataTypeTok{void} \FunctionTok{ClearAnimation}\NormalTok{ (}\DataTypeTok{string}\NormalTok{ animSet, }\DataTypeTok{string}\NormalTok{ animName)}
\end{Highlighting}
\end{Shaded}

\hypertarget{animations}{%
\subsection*{Animations}\label{animations}}
\addcontentsline{toc}{subsection}{Animations}

//TO DO

\hypertarget{teleporting}{%
\subsection*{Teleporting}\label{teleporting}}
\addcontentsline{toc}{subsection}{Teleporting}

We can change the location of the player character or of any Ped or Vehicle entity by using the native function \texttt{SET\_ENTITY\_COORDS}. This function needs an entity and X, Y and Z coordinate to teleport to.
We need to know the exact coordinates of the locations we want to teleport to, but thankfully the modding community forums provide lists with all available coordinates we can teleport to. Let's take the XYZ coordinates of the top of Mount Chiliad (the highest point in the game) to teleport our player character to.

\begin{verbatim}
LOCATION: Top of the Mt Chilad
COORDINATES: X:450.718 Y:5566.614 Z:806.183
\end{verbatim}

To create a teleport function we will use a native function. Script Hook V Dot Net is a wrapper for the C++ ScriptHook, calling the functions in Scripthook to do things in the game. However, there are some functions that are not in Script Hook V Dot Net and in order to use these, we have to use the native calling from Script Hook.

\href{https://nitanmarcel.github.io/shvdn-docs.github.io/namespace_g_t_a_1_1_native.html\#a84977424e1cb7b6f1c2902770bf9ad2d}{Native functions} are called with \texttt{Function.Call} followed by their corresponding hash name and parameters. They use this structure:

\begin{Shaded}
\begin{Highlighting}[]
\NormalTok{Function.}\FunctionTok{Call}\NormalTok{(Hash.}\FunctionTok{HASH_NAME}\NormalTok{, input_params);}
\end{Highlighting}
\end{Shaded}

The native function for teleporting expects the hash \texttt{SET\_ENTITY\_COORDS}, the \texttt{ped} entity to teleport, and the X, Y and Z coordinates to teleport the character to. \texttt{Function.Call(Hash.SET\_ENTITY\_COORDS,\ Ped\ ped,\ X,\ Y,\ Z,\ 0,\ 0,\ 1);}

The function to teleport the player character to the top of Moutn Chiliad is:

\begin{Shaded}
\begin{Highlighting}[]
\CommentTok{//Teleport to the top of Mount Chiliad}
\NormalTok{Function.}\FunctionTok{Call}\NormalTok{(Hash.}\FunctionTok{SET_ENTITY_COORDS}\NormalTok{, Game.}\FunctionTok{Player}\NormalTok{.}\FunctionTok{Character}\NormalTok{, }\FloatTok{450.718f}\NormalTok{, }\FloatTok{5566.614f}\NormalTok{, }\FloatTok{806.183f}\NormalTok{, }\DecValTok{0}\NormalTok{, }\DecValTok{0}\NormalTok{, }\DecValTok{1}\NormalTok{);}
\end{Highlighting}
\end{Shaded}

See this \href{https://gtaforums.com/topic/792877-list-of-over-100-coordinates-more-comming/}{list of locations} to find their respective coordinates or click on the list below

List of Locations with Coordinates

\begin{verbatim}
INDOOR LOCATIONS
 
Strip Club DJ Booth X:126.135 Y:-1278.583 Z:29.270

Blaine County Savings Bank X:-109.299 Y:6464.035 Z:31.627

Police Station X:436.491 Y: -982.172 Z:30.699

Humane Labs Entrance X:3619.749 Y:2742.740 Z:28.690

Burnt FIB Building X:160.868 Y:-745.831 Z:250.063

10 Car Garage Back Room X:223.193 Y:-967.322 Z:99.000

Humane Labs Tunnel X:3525.495 Y:3705.301 Z:20.992

Ammunation Office X:12.494 Y:-1110.130 Z: 29.797

Ammunation Gun Range X: 22.153 Y:-1072.854 Z:29.797

Trevor's Meth Lab X:1391.773 Y:3608.716 Z:38.942

Pacific Standard Bank Vault X:255.851 Y: 217.030 Z:101.683

Lester's House X:1273.898 Y:-1719.304 Z:54.771

Floyd's Apartment X:-1150.703 Y:-1520.713 Z:10.633

FIB Top Floor X:135.733 Y:-749.216 Z:258.152

IAA Office X:117.220 Y:-620.938 Z:206.047

Pacific Standard Bank X:235.046 Y:216.434 Z:106.287

Fort Zancudo ATC entrance X:-2344.373 Y:3267.498 Z:32.811

Fort Zancudo ATC top floor X:-2358.132 Y:3249.754 Z:101.451

Torture Room X: 147.170 Y:-2201.804 Z:4.688

 
OUTDOOR LOCATIONS
 
Main LS Customs X:-365.425 Y:-131.809 Z:37.873

Very High Up X:-129.964 Y:8130.873 Z:6705.307

IAA Roof X:134.085 Y:-637.859 Z:262.851

FIB Roof X:150.126 Y:-754.591 Z:262.865

Maze Bank Roof X:-75.015 Y:-818.215 Z:326.176

Top of the Mt Chilad X:450.718 Y:5566.614 Z:806.183

Most Northerly Point X:24.775 Y:7644.102 Z:19.055

Vinewood Bowl Stage X:686.245 Y:577.950 Z:130.461

Sisyphus Theater Stage X:205.316 Y:1167.378 Z:227.005

Galileo Observatory Roof X:-438.804 Y:1076.097 Z:352.411

Kortz Center X:-2243.810 Y:264.048 Z:174.615

Chumash Historic Family Pier X:-3426.683 Y:967.738 Z:8.347

Paleto Bay Pier X:-275.522 Y:6635.835 Z:7.425

God's thumb X:-1006.402 Y:6272.383 Z:1.503

Calafia Train Bridge X:-517.869 Y:4425.284 Z:89.795

Altruist Cult Camp X:-1170.841 Y:4926.646 Z:224.295

Maze Bank Arena Roof X:-324.300 Y:-1968.545 Z:67.002

Marlowe Vineyards X:-1868.971 Y:2095.674 Z:139.115

Hippy Camp X:2476.712 Y:3789.645 Z:41.226

Devin Weston's House X:-2639.872 Y:1866.812 Z:160.135

Abandon Mine X:-595.342 Y: 2086.008 Z:131.412

Weed Farm X:2208.777 Y:5578.235 Z:53.735

Stab City X: 126.975 Y:3714.419 Z:46.827

Airplane Graveyard Airplane Tail X:2395.096 Y:3049.616 Z:60.053

Satellite Dish Antenna X:2034.988 Y:2953.105 Z:74.602

Satellite Dishes X: 2062.123 Y:2942.055 Z:47.431

Windmill Top X:2026.677 Y:1842.684 Z:133.313

Sandy Shores Building Site Crane X:1051.209 Y:2280.452 Z:89.727

Rebel Radio X:736.153 Y:2583.143 Z:79.634

Quarry X:2954.196 Y:2783.410 Z:41.004

Palmer-Taylor Power Station Chimney X: 2732.931 Y: 1577.540 Z:83.671

Merryweather Dock X: 486.417 Y:-3339.692 Z:6.070

Cargo Ship X:899.678 Y:-2882.191 Z:19.013

Del Perro Pier X:-1850.127 Y:-1231.751 Z:13.017

Play Boy Mansion X:-1475.234 Y:167.088Z:55.841

Jolene Cranley-Evans Ghost X:3059.620 Y:5564.246 Z:197.091

NOOSE Headquarters X:2535.243 Y:-383.799 Z:92.993

Snowman X: 971.245 Y:-1620.993 Z:30.111

Oriental Theater X:293.089 Y:180.466 Z:104.301

Beach Skatepark X:-1374.881 Y:-1398.835 Z:6.141

Underpass Skatepark X:718.341 Y:-1218.714 Z: 26.014

Casino X:925.329 Y:46.152 Z:80.908

University of San Andreas X:-1696.866 Y:142.747 Z:64.372

La Puerta Freeway Bridge X: -543.932 Y:-2225.543 Z:122.366

Land Act Dam X: 1660.369 Y:-12.013 Z:170.020

Mount Gordo X: 2877.633 Y:5911.078 Z:369.624

Little Seoul X:-889.655 Y:-853.499 Z:20.566

Epsilon Building X:-695.025 Y:82.955 Z:55.855 Z:55.855

The Richman Hotel X:-1330.911 Y:340.871 Z:64.078

Vinewood sign X:711.362 Y:1198.134 Z:348.526

Los Santos Golf Club X:-1336.715 Y:59.051 Z:55.246

Chicken X:-31.010 Y:6316.830 Z:40.083

Little Portola X:-635.463 Y:-242.402 Z:38.175

Pacific Bluffs Country Club X:-3022.222 Y:39.968 Z:13.611

Vinewood Cemetery X:-1659993 Y:-128.399 Z:59.954

Paleto Forest Sawmill Chimney X:-549.467 Y:5308.221 Z:114.146

Mirror Park X:1070.206 Y:-711.958 Z:58.483

Rocket X:1608.698 Y:6438.096 Z:37.637

El Gordo Lighthouse X:3430.155 Y:5174.196 Z:41.280
\end{verbatim}

\hypertarget{content-replication-assignment-4}{%
\section*{Content Replication Assignment}\label{content-replication-assignment-4}}
\addcontentsline{toc}{section}{Content Replication Assignment}

Teleport the player to a beach, spawn ten whales on the shore and generate an NPC wandering aroud them and take a screenshot in the style of HAPP V2.

\hypertarget{hyperrealism}{%
\chapter{Hyperrealism}\label{hyperrealism}}

//intro to the artistic current of photorealism and hyperrealism in painting and drawing, connected to the idea of photorealism and simulation in games (attempt to simulate life itself, not just photography), relationship between the game and the physical world, the way virtual spaces influence and shape society (training self driving cars in GTA V, CGI shaping architecture of buildings and object\ldots), the blurrying of the lines between virtual and physical\ldots{}

\hypertarget{k-by-aram-bartholl}{%
\subsection*{\texorpdfstring{\emph{8k} by Aram Bartholl}{8k by Aram Bartholl}}\label{k-by-aram-bartholl}}
\addcontentsline{toc}{subsection}{\emph{8k} by Aram Bartholl}

Aram Bartholl, \emph{8k}, installation view

Aram Bartholl, \emph{8k}, installation view

\href{https://arambartholl.com/8k/}{More about 8k}

\hypertarget{getting-there-8}{%
\subsection*{Getting there}\label{getting-there-8}}
\addcontentsline{toc}{subsection}{Getting there}

\begin{itemize}
\tightlist
\item
  \href{https://grandtheftdata.com/landmarks/\#1672.279,-29.78,4,atlas,name=tataviam_act_dam,Land_Act_Dam,_Tataviam_Mountains}{Land Act Dam, Tataviam Mountains}
\end{itemize}

\hypertarget{readings-5}{%
\section*{Readings}\label{readings-5}}
\addcontentsline{toc}{section}{Readings}

\hypertarget{tutorial-5}{%
\section*{Tutorial}\label{tutorial-5}}
\addcontentsline{toc}{section}{Tutorial}

\hypertarget{setting-camera-views}{%
\subsection*{Setting Camera Views}\label{setting-camera-views}}
\addcontentsline{toc}{subsection}{Setting Camera Views}

GTA V has 4 default camera views, which can be switched by pressing the \texttt{V} key on PC.
To set a specific camera view we can use the native function \texttt{SET\_FOLLOW\_PED\_CAM\_VIEW\_MOD}, followed by number 0, 1, 2 or 4 to establish the desired point of view:

\begin{itemize}
\tightlist
\item
  0 - Third Person View - Close
\item
  1 - Third Person View - Mid
\item
  2 - Third Person View - Far
\item
  4 - First Person View
\end{itemize}

Switch to first person view:

\begin{Shaded}
\begin{Highlighting}[]
\NormalTok{Function.}\FunctionTok{Call}\NormalTok{(Hash.}\FunctionTok{SET_FOLLOW_PED_CAM_VIEW_MODE}\NormalTok{, }\DecValTok{4}\NormalTok{); }
\end{Highlighting}
\end{Shaded}

\hypertarget{controlling-the-game-camera}{%
\subsection*{Controlling the Game Camera}\label{controlling-the-game-camera}}
\addcontentsline{toc}{subsection}{Controlling the Game Camera}

\hypertarget{create-multiple-cameras}{%
\subsection*{Create Multiple Cameras}\label{create-multiple-cameras}}
\addcontentsline{toc}{subsection}{Create Multiple Cameras}

\hypertarget{attach-a-camera-to-an-entity}{%
\subsection*{Attach a Camera to an Entity}\label{attach-a-camera-to-an-entity}}
\addcontentsline{toc}{subsection}{Attach a Camera to an Entity}

Example code

\begin{Shaded}
\begin{Highlighting}[]
\KeywordTok{using}\NormalTok{ System;}
\KeywordTok{using}\NormalTok{ System.}\FunctionTok{Collections}\NormalTok{.}\FunctionTok{Generic}\NormalTok{;}
\KeywordTok{using}\NormalTok{ System.}\FunctionTok{Linq}\NormalTok{;}
\KeywordTok{using}\NormalTok{ System.}\FunctionTok{Text}\NormalTok{;}
\KeywordTok{using}\NormalTok{ System.}\FunctionTok{Threading}\NormalTok{.}\FunctionTok{Tasks}\NormalTok{;}

\KeywordTok{using}\NormalTok{ GTA;}
\KeywordTok{using}\NormalTok{ GTA.}\FunctionTok{Math}\NormalTok{;}
\KeywordTok{using}\NormalTok{ System.}\FunctionTok{Windows}\NormalTok{.}\FunctionTok{Forms}\NormalTok{;}
\KeywordTok{using}\NormalTok{ System.}\FunctionTok{Drawing}\NormalTok{;}
\KeywordTok{using}\NormalTok{ GTA.}\FunctionTok{Native}\NormalTok{;}
\KeywordTok{using}\NormalTok{ System.}\FunctionTok{IO}\NormalTok{;}


\KeywordTok{namespace}\NormalTok{ moddingTutorial}
\NormalTok{\{}
    \KeywordTok{public} \KeywordTok{class}\NormalTok{ moddingTutorial : Script}
\NormalTok{    \{}
\NormalTok{        Vector3 myCamPos;}
        \DataTypeTok{int}\NormalTok{ CamSelect = }\DecValTok{0}\NormalTok{;}
\NormalTok{        Ped newPed = }\KeywordTok{null}\NormalTok{;}
\NormalTok{        Camera myCam;}

        \KeywordTok{public} \FunctionTok{moddingTutorial}\NormalTok{()}
\NormalTok{        \{}
            \KeywordTok{this}\NormalTok{.}\FunctionTok{Tick}\NormalTok{ += onTick;}
            \KeywordTok{this}\NormalTok{.}\FunctionTok{KeyUp}\NormalTok{ += onKeyUp;}
            \KeywordTok{this}\NormalTok{.}\FunctionTok{KeyDown}\NormalTok{ += onKeyDown;}
\NormalTok{        \}}

        \KeywordTok{private} \DataTypeTok{void} \FunctionTok{onTick}\NormalTok{(}\DataTypeTok{object}\NormalTok{ sender, EventArgs e) }\CommentTok{//this function gets executed continuously }
\NormalTok{        \{}
            \CommentTok{//exits from the loop if the game is loading}
            \KeywordTok{if}\NormalTok{ (Game.}\FunctionTok{IsLoading}\NormalTok{) }\KeywordTok{return}\NormalTok{;}
            
            \CommentTok{//update the cameras if the ped is spawn}
            \KeywordTok{if}\NormalTok{ (newPed != }\KeywordTok{null}\NormalTok{)}
\NormalTok{            \{           }
                \CommentTok{//create the cameras if none have been created yet.}
                \KeywordTok{if}\NormalTok{ (myCam == }\KeywordTok{null}\NormalTok{)            }
\NormalTok{                \{}
\NormalTok{                    UI.}\FunctionTok{ShowSubtitle}\NormalTok{(}\StringTok{"Set new camera"}\NormalTok{);}
\NormalTok{                    myCam = World.}\FunctionTok{CreateCamera}\NormalTok{(Vector3.}\FunctionTok{Zero}\NormalTok{, newPed.}\FunctionTok{Rotation}\NormalTok{, 50f);}
                    \CommentTok{// Set the camera position (relative pos)          }
\NormalTok{                    myCamPos = }\KeywordTok{new} \FunctionTok{Vector3}\NormalTok{(}\DecValTok{0}\NormalTok{, }\DecValTok{0}\NormalTok{, 1f);}
\NormalTok{                \}}
                \CommentTok{//attach the cameras}
\NormalTok{                myCam.}\FunctionTok{AttachTo}\NormalTok{(newPed, myCamPos);}
                \CommentTok{//sync rotation}
\NormalTok{                myCam.}\FunctionTok{Rotation}\NormalTok{ = newPed.}\FunctionTok{Rotation}\NormalTok{;}
\NormalTok{            \}}
\NormalTok{        \}}
        

        \KeywordTok{private} \DataTypeTok{void} \FunctionTok{onKeyUp}\NormalTok{(}\DataTypeTok{object}\NormalTok{ sender, KeyEventArgs e)}\CommentTok{//everything inside here is executed only when we release a key}
\NormalTok{        \{}
            \CommentTok{//press control+K to switch between gameplay default camera and the NPC camera}
            \KeywordTok{if}\NormalTok{ (e.}\FunctionTok{KeyCode}\NormalTok{ == Keys.}\FunctionTok{K}\NormalTok{ && e.}\FunctionTok{Modifiers}\NormalTok{ == Keys.}\FunctionTok{Shift}\NormalTok{ && newPed != }\KeywordTok{null}\NormalTok{)}
\NormalTok{            \{}
\NormalTok{                CamSelect = (CamSelect + }\DecValTok{1}\NormalTok{) % }\DecValTok{2}\NormalTok{;}
                \KeywordTok{switch}\NormalTok{ (CamSelect)}
\NormalTok{                \{}
                    \KeywordTok{case} \DecValTok{0}\NormalTok{: World.}\FunctionTok{RenderingCamera}\NormalTok{ = }\KeywordTok{null}\NormalTok{; }
\NormalTok{                        UI.}\FunctionTok{ShowSubtitle}\NormalTok{(}\StringTok{"Showing Gameplay Cam View"}\NormalTok{); }
                        \KeywordTok{break}\NormalTok{;}
                    \KeywordTok{case} \DecValTok{1}\NormalTok{: World.}\FunctionTok{RenderingCamera}\NormalTok{ = myCam; }
\NormalTok{                        UI.}\FunctionTok{ShowSubtitle}\NormalTok{(}\StringTok{"Showing NPC Cam View"}\NormalTok{); }
                        \KeywordTok{break}\NormalTok{;}
\NormalTok{                \}}

\NormalTok{            \}}
\NormalTok{        \}}

        \KeywordTok{private} \DataTypeTok{void} \FunctionTok{onKeyDown}\NormalTok{(}\DataTypeTok{object}\NormalTok{ sender, KeyEventArgs e) }\CommentTok{//everything inside here is executed only when we press a key}
\NormalTok{        \{}
            \KeywordTok{if}\NormalTok{(e.}\FunctionTok{KeyCode}\NormalTok{ == Keys.}\FunctionTok{G}\NormalTok{)}
\NormalTok{            \{}
                \CommentTok{//spawn new Ped}
\NormalTok{                newPed = World.}\FunctionTok{CreatePed}\NormalTok{(PedHash.}\FunctionTok{Cat}\NormalTok{, Game.}\FunctionTok{Player}\NormalTok{.}\FunctionTok{Character}\NormalTok{.}\FunctionTok{GetOffsetInWorldCoords}\NormalTok{(}\KeywordTok{new} \FunctionTok{Vector3}\NormalTok{(}\DecValTok{1}\NormalTok{, }\DecValTok{3}\NormalTok{, }\DecValTok{0}\NormalTok{)));}
\NormalTok{            \}}
        
            \KeywordTok{if}\NormalTok{ (e.}\FunctionTok{KeyCode}\NormalTok{ == Keys.}\FunctionTok{H}\NormalTok{)}
\NormalTok{            \{}
                \CommentTok{//follow player (persistent)}
\NormalTok{                Function.}\FunctionTok{Call}\NormalTok{(Hash.}\FunctionTok{TASK_FOLLOW_TO_OFFSET_OF_ENTITY}\NormalTok{, newPed.}\FunctionTok{Handle}\NormalTok{, Game.}\FunctionTok{Player}\NormalTok{.}\FunctionTok{Character}\NormalTok{.}\FunctionTok{Handle}\NormalTok{, 0f, 1f, 0f, }\FloatTok{2.0f}\NormalTok{, }\DecValTok{-1}\NormalTok{, 5f, }\KeywordTok{true}\NormalTok{);}
                \CommentTok{//look at player}
\NormalTok{                newPed.}\FunctionTok{Task}\NormalTok{.}\FunctionTok{LookAt}\NormalTok{(Game.}\FunctionTok{Player}\NormalTok{.}\FunctionTok{Character}\NormalTok{);}
\NormalTok{            \}}
            \KeywordTok{if}\NormalTok{ (e.}\FunctionTok{KeyCode}\NormalTok{ == Keys.}\FunctionTok{J}\NormalTok{)}
\NormalTok{            \{}
                \CommentTok{//stop NPC}
\NormalTok{                newPed.}\FunctionTok{Task}\NormalTok{.}\FunctionTok{ClearAll}\NormalTok{();}
\NormalTok{            \}}
            \KeywordTok{if}\NormalTok{ (e.}\FunctionTok{KeyCode}\NormalTok{ == Keys.}\FunctionTok{L}\NormalTok{)}
\NormalTok{            \{   }
                \CommentTok{//delete ped}
\NormalTok{                newPed.}\FunctionTok{Delete}\NormalTok{();}
\NormalTok{            \}}
\NormalTok{        \}}
\NormalTok{    \}}
\NormalTok{\}}
\end{Highlighting}
\end{Shaded}

\hypertarget{switching-character-through-satellite-camera-view}{%
\subsection{Switching Character through Satellite Camera View}\label{switching-character-through-satellite-camera-view}}

Example code

\begin{Shaded}
\begin{Highlighting}[]
    
\CommentTok{/*}
\CommentTok{    this was adapted from code shared by LeeC22 on gtaforums.com}
\CommentTok{    https://gtaforums.com/topic/951002-c-looking-for-player-switch-sample-solved-by-me/#comment-1071197769}
\CommentTok{*/}
    
\KeywordTok{using}\NormalTok{ System;}
\KeywordTok{using}\NormalTok{ System.}\FunctionTok{Collections}\NormalTok{.}\FunctionTok{Generic}\NormalTok{;}
\KeywordTok{using}\NormalTok{ System.}\FunctionTok{Linq}\NormalTok{;}
\KeywordTok{using}\NormalTok{ System.}\FunctionTok{Text}\NormalTok{;}
\KeywordTok{using}\NormalTok{ System.}\FunctionTok{Threading}\NormalTok{.}\FunctionTok{Tasks}\NormalTok{;}

\KeywordTok{using}\NormalTok{ GTA;}
\KeywordTok{using}\NormalTok{ GTA.}\FunctionTok{Math}\NormalTok{;}
\KeywordTok{using}\NormalTok{ System.}\FunctionTok{Windows}\NormalTok{.}\FunctionTok{Forms}\NormalTok{;}
\KeywordTok{using}\NormalTok{ System.}\FunctionTok{Drawing}\NormalTok{;}
\KeywordTok{using}\NormalTok{ GTA.}\FunctionTok{Native}\NormalTok{;}
\KeywordTok{using}\NormalTok{ System.}\FunctionTok{IO}\NormalTok{;}


\KeywordTok{namespace}\NormalTok{ moddingTutorial}
\NormalTok{\{}
    \KeywordTok{public} \KeywordTok{class}\NormalTok{ moddingTutorial : Script}
\NormalTok{    \{}
\NormalTok{        Ped newPed = }\KeywordTok{null}\NormalTok{;}
\NormalTok{        Vector3 SwitchLocation2;}
\NormalTok{        List<Vector3> switchLocations = }\KeywordTok{new}\NormalTok{ List<Vector3>();}
        \DataTypeTok{int}\NormalTok{ index = }\DecValTok{0}\NormalTok{;}
\NormalTok{        List<String> models = }\KeywordTok{new}\NormalTok{ List<String>();}
        \DataTypeTok{int}\NormalTok{ modelIndex = }\DecValTok{0}\NormalTok{;}

        \KeywordTok{public} \FunctionTok{moddingTutorial}\NormalTok{()}
\NormalTok{        \{}
            \KeywordTok{this}\NormalTok{.}\FunctionTok{Tick}\NormalTok{ += onTick;}
            \KeywordTok{this}\NormalTok{.}\FunctionTok{KeyUp}\NormalTok{ += onKeyUp;}
            \KeywordTok{this}\NormalTok{.}\FunctionTok{KeyDown}\NormalTok{ += onKeyDown;}

        \CommentTok{//add locations to the switchLocations list}
\NormalTok{            switchLocations.}\FunctionTok{Add}\NormalTok{(}\KeywordTok{new} \FunctionTok{Vector3}\NormalTok{(}\FloatTok{24.775f}\NormalTok{, }\FloatTok{7644.102f}\NormalTok{, }\FloatTok{18.055f}\NormalTok{)); }\CommentTok{//Most Northerly Point}
\NormalTok{            switchLocations.}\FunctionTok{Add}\NormalTok{(}\KeywordTok{new} \FunctionTok{Vector3}\NormalTok{(-}\FloatTok{595.342f}\NormalTok{, }\FloatTok{2086.008f}\NormalTok{, }\FloatTok{130.412f}\NormalTok{)); }\CommentTok{//Mine}
\NormalTok{            switchLocations.}\FunctionTok{Add}\NormalTok{(}\KeywordTok{new} \FunctionTok{Vector3}\NormalTok{(}\FloatTok{150.126f}\NormalTok{, }\FloatTok{-754.591f}\NormalTok{, }\FloatTok{261.865f}\NormalTok{)); }\CommentTok{//FIB Roof }
    
        \CommentTok{//add models to the models list}
\NormalTok{            models.}\FunctionTok{Add}\NormalTok{(}\StringTok{"s_m_m_doctor_01"}\NormalTok{);}
\NormalTok{            models.}\FunctionTok{Add}\NormalTok{(}\StringTok{"s_m_m_migrant_01"}\NormalTok{);}
\NormalTok{            models.}\FunctionTok{Add}\NormalTok{(}\StringTok{"a_c_cormorant"}\NormalTok{);}
\NormalTok{            models.}\FunctionTok{Add}\NormalTok{(}\StringTok{"a_c_deer"}\NormalTok{);}
\NormalTok{            models.}\FunctionTok{Add}\NormalTok{(}\StringTok{"a_c_pug"}\NormalTok{);}
\NormalTok{        \}}

        \KeywordTok{private} \DataTypeTok{void} \FunctionTok{onTick}\NormalTok{(}\DataTypeTok{object}\NormalTok{ sender, EventArgs e) }\CommentTok{//this function gets executed continuously }
\NormalTok{        \{}
            
            \CommentTok{//If the character switch is in process}
            \KeywordTok{if}\NormalTok{ (Function.}\FunctionTok{Call}\NormalTok{<}\DataTypeTok{bool}\NormalTok{>(Hash.}\FunctionTok{IS_PLAYER_SWITCH_IN_PROGRESS}\NormalTok{))}
\NormalTok{            \{}
                \CommentTok{//If Switch State is 8 – that's the point when it starts dropping to the player }
                \KeywordTok{if}\NormalTok{ (Function.}\FunctionTok{Call}\NormalTok{<}\DataTypeTok{int}\NormalTok{>(Hash.}\FunctionTok{GET_PLAYER_SWITCH_STATE}\NormalTok{) == }\DecValTok{8}\NormalTok{)}
\NormalTok{                \{}
                    \CommentTok{//Set the player to the switch location}
\NormalTok{                    Game.}\FunctionTok{Player}\NormalTok{.}\FunctionTok{Character}\NormalTok{.}\FunctionTok{Position}\NormalTok{ = switchLocations[index];}

                    \CommentTok{//Generate the hash for the chosen model}
                    \DataTypeTok{int}\NormalTok{ poshHash = Game.}\FunctionTok{GenerateHash}\NormalTok{(models[modelIndex]);}

                    \CommentTok{//Create the model}
\NormalTok{                    Model poshModel = }\KeywordTok{new} \FunctionTok{Model}\NormalTok{(poshHash);}

                    \CommentTok{//Check if it is valid}
                    \KeywordTok{if}\NormalTok{ (poshModel.}\FunctionTok{IsValid}\NormalTok{)}
\NormalTok{                    \{}
                        \CommentTok{//Wait for it to load, should be okay because it was used to create the target ped}
                        \KeywordTok{while}\NormalTok{ (!poshModel.}\FunctionTok{IsLoaded}\NormalTok{)}
\NormalTok{                        \{}
                            \FunctionTok{Wait}\NormalTok{(}\DecValTok{100}\NormalTok{);}
\NormalTok{                        \}}

                        \CommentTok{//Change the player model to the target ped model}
\NormalTok{                        Function.}\FunctionTok{Call}\NormalTok{(Hash.}\FunctionTok{SET_PLAYER_MODEL}\NormalTok{, Game.}\FunctionTok{Player}\NormalTok{, poshHash);}

                        \CommentTok{//Let the game clean up the created Model}
\NormalTok{                        poshModel.}\FunctionTok{MarkAsNoLongerNeeded}\NormalTok{();}
\NormalTok{                    \}}
                    \KeywordTok{else}
\NormalTok{                    \{}
                        \CommentTok{//Falls to here if the model valid check fails}
\NormalTok{                        Function.}\FunctionTok{Call}\NormalTok{(Hash.}\FunctionTok{SET_PLAYER_MODEL}\NormalTok{, Game.}\FunctionTok{Player}\NormalTok{, (}\DataTypeTok{int}\NormalTok{)PedHash.}\FunctionTok{Tourist01AFY}\NormalTok{);}
\NormalTok{                    \}}

                    \CommentTok{//Delete the target ped as it's no longer needed}
\NormalTok{                    newPed.}\FunctionTok{Delete}\NormalTok{();}
                    

                    \CommentTok{// Set the switch outro based on the gameplay camera position}
                    \CommentTok{// Function.Call((Hash)0xC208B673CE446B61, camPos.X, camPos.Y, camPos.Z, camRot.X, camRot.Y, camRot.Z, camFOV, camFarClip, p8);}

\NormalTok{                    Function.}\FunctionTok{Call}\NormalTok{((Hash)}\BaseNTok{0xC208B673CE446B61}\NormalTok{, GameplayCamera.}\FunctionTok{Position}\NormalTok{.}\FunctionTok{X}\NormalTok{, GameplayCamera.}\FunctionTok{Position}\NormalTok{.}\FunctionTok{Y}\NormalTok{, GameplayCamera.}\FunctionTok{Position}\NormalTok{.}\FunctionTok{Z}\NormalTok{, GameplayCamera.}\FunctionTok{Rotation}\NormalTok{.}\FunctionTok{X}\NormalTok{, GameplayCamera.}\FunctionTok{Rotation}\NormalTok{.}\FunctionTok{Y}\NormalTok{, GameplayCamera.}\FunctionTok{Rotation}\NormalTok{.}\FunctionTok{Z}\NormalTok{, GameplayCamera.}\FunctionTok{FieldOfView}\NormalTok{, }\DecValTok{500}\NormalTok{, }\DecValTok{2}\NormalTok{);}

                    \CommentTok{//Call this unknown native that seems to finish things off}
\NormalTok{                    Function.}\FunctionTok{Call}\NormalTok{(Hash._}\BaseNTok{0x74DE2E8739086740}\NormalTok{);}

            \CommentTok{//Make the character wander around autonomously}
\NormalTok{                    Game.}\FunctionTok{Player}\NormalTok{.}\FunctionTok{Character}\NormalTok{.}\FunctionTok{Task}\NormalTok{.}\FunctionTok{WanderAround}\NormalTok{();}
\NormalTok{                \}}
\NormalTok{            \}}

\NormalTok{        \}}


        \KeywordTok{private} \DataTypeTok{void} \FunctionTok{onKeyUp}\NormalTok{(}\DataTypeTok{object}\NormalTok{ sender, KeyEventArgs e)}
\NormalTok{        \{}

\NormalTok{        \}}

        \KeywordTok{private} \DataTypeTok{void} \FunctionTok{onKeyDown}\NormalTok{(}\DataTypeTok{object}\NormalTok{ sender, KeyEventArgs e)}
\NormalTok{        \{}
            \KeywordTok{if}\NormalTok{ (e.}\FunctionTok{KeyCode}\NormalTok{ == Keys.}\FunctionTok{G}\NormalTok{)}
\NormalTok{            \{}
        \CommentTok{//Stop previous tasks}
\NormalTok{                Game.}\FunctionTok{Player}\NormalTok{.}\FunctionTok{Character}\NormalTok{.}\FunctionTok{Task}\NormalTok{.}\FunctionTok{ClearAll}\NormalTok{();}

                \CommentTok{//Move the index to the next location}
\NormalTok{                index++;}
                \KeywordTok{if}\NormalTok{ (index >= switchLocations.}\FunctionTok{Count}\NormalTok{) index = }\DecValTok{0}\NormalTok{;}
                
        \CommentTok{//Move the index to the new ped model}
\NormalTok{                modelIndex++;}
                \KeywordTok{if}\NormalTok{ (modelIndex >= models.}\FunctionTok{Count}\NormalTok{) modelIndex = }\DecValTok{0}\NormalTok{;}

                \CommentTok{//Create the ped to switch to}
\NormalTok{                newPed = World.}\FunctionTok{CreatePed}\NormalTok{(models[modelIndex], switchLocations[index]);}

                \CommentTok{//Native function to initiate the switch Function.Call(Hash.START_PLAYER_SWITCH, fromPed.Handle, toPed.Handle, flags, switchType);}
\NormalTok{                Function.}\FunctionTok{Call}\NormalTok{(Hash.}\FunctionTok{START_PLAYER_SWITCH}\NormalTok{, Game.}\FunctionTok{Player}\NormalTok{.}\FunctionTok{Character}\NormalTok{.}\FunctionTok{Handle}\NormalTok{, newPed.}\FunctionTok{Handle}\NormalTok{, }\DecValTok{8}\NormalTok{, }\DecValTok{0}\NormalTok{);}

\NormalTok{            \}}
\NormalTok{        \}}
\NormalTok{    \}}
\NormalTok{\}}
\end{Highlighting}
\end{Shaded}

\hypertarget{scripting-cinematic-fade-out-in}{%
\subsection*{Scripting Cinematic Fade Out/ In}\label{scripting-cinematic-fade-out-in}}
\addcontentsline{toc}{subsection}{Scripting Cinematic Fade Out/ In}

Script Hook has native functions to create a fead to/from black. The function hash are \texttt{DO\_SCREEN\_FADE\_OUT} and \texttt{DO\_SCREEN\_FADE\_IN} and they are followed by the number of milliseconds to go from full black to showing the scene and viceversa.

Let's create a fade to black over 3 seconds when we press the letter key `O':

\begin{Shaded}
\begin{Highlighting}[]
\KeywordTok{if}\NormalTok{ (e.}\FunctionTok{KeyCode}\NormalTok{ == Keys.}\FunctionTok{O}\NormalTok{)}
\NormalTok{\{}
\NormalTok{    Function.}\FunctionTok{Call}\NormalTok{(Hash.}\FunctionTok{DO_SCREEN_FADE_OUT}\NormalTok{, }\DecValTok{3000}\NormalTok{);}
\NormalTok{\}}
\end{Highlighting}
\end{Shaded}

And a fade over 3 seconds when we press the letter key `I':

\begin{Shaded}
\begin{Highlighting}[]
\KeywordTok{if}\NormalTok{ (e.}\FunctionTok{KeyCode}\NormalTok{ == Keys.}\FunctionTok{I}\NormalTok{)}
\NormalTok{\{}
\NormalTok{    Function.}\FunctionTok{Call}\NormalTok{(Hash.}\FunctionTok{DO_SCREEN_FADE_IN}\NormalTok{, }\DecValTok{3000}\NormalTok{);}
\NormalTok{\}}
\end{Highlighting}
\end{Shaded}

We could also create an automated check in our onTick loop, which keeps seeing if the screen has been faded to black. We can use the native function \texttt{IS\_SCREEN\_FADED\_OUT} which is a boolean data type. This means it will return either true or false. If it returns true, it means the screen has been faded out.

Let's add an If statement in our onTick loop to chek if the screen has been faded to black, and if so we call a fade in over 3 second:

\begin{Shaded}
\begin{Highlighting}[]
\KeywordTok{if}\NormalTok{ (Function.}\FunctionTok{Call}\NormalTok{<}\DataTypeTok{bool}\NormalTok{>(Hash.}\FunctionTok{IS_SCREEN_FADED_OUT}\NormalTok{))}
\NormalTok{\{}
    \FunctionTok{Wait}\NormalTok{(}\DecValTok{500}\NormalTok{);}
\NormalTok{    GTA.}\FunctionTok{Native}\NormalTok{.}\FunctionTok{Function}\NormalTok{.}\FunctionTok{Call}\NormalTok{(Hash.}\FunctionTok{DO_SCREEN_FADE_IN}\NormalTok{, }\DecValTok{3000}\NormalTok{);}
\NormalTok{\}}
\end{Highlighting}
\end{Shaded}

Now if you try to hit the `O' key, the screen will fade out, and then it will automatically fade in again.

\hypertarget{natural-vision-evolved-mod}{%
\subsection*{Natural Vision Evolved Mod}\label{natural-vision-evolved-mod}}
\addcontentsline{toc}{subsection}{Natural Vision Evolved Mod}

Natural Vision Evolved is a graphic mod develped by \href{https://www.razedmods.com/}{Jamal Rashid, aka Razed}. This mod enhances GTA V's lighting, weather effects, ambient colours, world textures, building models, pushing the photo-realism and cinematic looks. While the mod contains settings for different hardware settings, it's recommended to have a relatively powerful PC with a good graphic card. \href{https://www.systemrequirementslab.com/cyri/requirements/gta-5-naturalvision-remastered/16594}{Here} you can find minimum and recommended requirements for Natural Vision Evolved mod.

Setup:

\begin{itemize}
\item
  Go to razedmods.com/gta-v and download Natural Vision Evolved (6.2 Gb).
\item
  Go to \href{https://openiv.com/}{openiv.com/} and download Open IV, Open `ovisetup' and install Open IV on your computer.
\item
  Open the Open IV app and select GTA V Windows. Choose Grand Theft Auto V folder \texttt{C:\textbackslash{}Program\ Files\ (x86)\textbackslash{}Steam\textbackslash{}steamapps\textbackslash{}common\textbackslash{}Grand\ Theft\ Auto\ V}
\item
  Once Open IV is open, go to your file window select the \texttt{Tools} menu on top of the window, and select \texttt{ASI\ Manager}. In \texttt{ASI\ Manager} install all options: ASI Loader, OpenIV.ASI and openCamera.
\item
  Select \texttt{Tools} again and click \texttt{Options}. Click on the \texttt{"mods"\ folder} tab and select \texttt{Allow\ edit\ mode\ only\ for\ archive\ inside\ "mods"\ folder}. Click \texttt{Close}.
\item
  Back in Open IV, select \texttt{Edit\ mode} at the top right of the window. Select \texttt{OK} on the pop up window.
\item
  Now you can add your mod to the mods folder.
\item
  Open the NVE mod folder and copy the subfolder containing the settings you want (low/medium/ultra graphics) to your GTA V directory folder \texttt{C:\textbackslash{}Program\ Files\ (x86)\textbackslash{}Steam\textbackslash{}steamapps\textbackslash{}common\textbackslash{}Grand\ Theft\ Auto\ V}.
\item
  Go back to your downloads and inside the NVE folder select the installation files and open number 1 and 2 with Open IV in the correct order. Always choose to select ``mod'' folder and select install.
\item
  Go back to your downloads and inside the NVE folder you can choose some optional addons. Install/Uninstall them with Open IV as above Always choose to select ``mod'' folder and select install.
\item
  Open GTA V, press \texttt{ESC} to bring up the menu and go to \texttt{SETTINGS}. Adjust the graphics quality, making sure \texttt{Shader\ Quality}, \texttt{Particle\ Quality} and \texttt{Post\ FX}are set to \texttt{\textless{}Very\ High\textgreater{}}. Restart the game to make change in effect.
\end{itemize}

\hypertarget{content-replication-assignment-5}{%
\section*{Content Replication Assignment}\label{content-replication-assignment-5}}
\addcontentsline{toc}{section}{Content Replication Assignment}

\hypertarget{meta-photography}{%
\chapter{Meta-Photography}\label{meta-photography}}

// General intro on the artistic reflection on the photographic medium itself, the tradition of conceptual photography and the connection to the simulation of the camera and the act of photographing virtual worlds

\hypertarget{gta-v-photography-bot}{%
\subsection*{GTA V Photography Bot}\label{gta-v-photography-bot}}
\addcontentsline{toc}{subsection}{GTA V Photography Bot}

\hypertarget{crossroad-of-realities-by-benoit-pailluxe9}{%
\subsection*{Crossroad of Realities by Benoit Paillé}\label{crossroad-of-realities-by-benoit-pailluxe9}}
\addcontentsline{toc}{subsection}{Crossroad of Realities by Benoit Paillé}

\hypertarget{readings-6}{%
\section*{Readings}\label{readings-6}}
\addcontentsline{toc}{section}{Readings}

\hypertarget{tutorial-6}{%
\section*{Tutorial}\label{tutorial-6}}
\addcontentsline{toc}{section}{Tutorial}

\hypertarget{virtual-keys}{%
\subsection*{Virtual Keys}\label{virtual-keys}}
\addcontentsline{toc}{subsection}{Virtual Keys}

The most reliable way to send keyboard and mouse input is the SendInput function in user32.dll.

The SendInput function takes three parameters: the number of inputs, an array of INPUT for the inputs we want to send, and the size of our INPUT struct. The INPUT struct includes an integer that indicates the type of input and a union for the inputs that will be passed.

Keys are mapped to Direct Input Keyboard hex codes. You can find a list of all hex codes for each key \href{http://www.flint.jp/misc/?q=dik\&lang=en}{here} or below.

List of DirectX key mappings

\begin{verbatim}
Value   Macro         Symbol
------------------------
0x01    DIK_ESCAPE  Esc 
0x02    DIK_1         1 
0x03    DIK_2         2 
0x04    DIK_3         3 
0x05    DIK_4         4 
0x06    DIK_5         5 
0x07    DIK_6         6 
0x08    DIK_7         7 
0x09    DIK_8         8 
0x0A    DIK_9         9 
0x0B    DIK_0         0 
0x0C    DIK_MINUS     - 
0x0D    DIK_EQUALS  =   
0x0E    DIK_BACK      Back Space    
0x0F    DIK_TAB     Tab 
0x10    DIK_Q         Q 
0x11    DIK_W         W 
0x12    DIK_E         E 
0x13    DIK_R         R 
0x14    DIK_T         T 
0x15    DIK_Y         Y 
0x16    DIK_U         U 
0x17    DIK_I         I 
0x18    DIK_O         O 
0x19    DIK_P         P 
0x1A    DIK_LBRACKET  [ 
0x1B    DIK_RBRACKET  ] 
0x1C    DIK_RETURN  Enter   
0x1D    DIK_LContol Ctrl (Left) 
0x1E    DIK_A         A 
0x1F    DIK_S         S 
0x20    DIK_D         D 
0x21    DIK_F         F 
0x22    DIK_G         G 
0x23    DIK_H         H 
0x24    DIK_J         J 
0x25    DIK_K         K 
0x26    DIK_L         L 
0x27    DIK_SEMICOLON   ;   
0x28    DIK_APOSTROPHE  '   
0x29    DIK_GRAVE     ` 
0x2A    DIK_LSHIFT  Shift (Left)    
0x2B    DIK_BACKSLASH   \   
0x2C    DIK_Z         Z 
0x2D    DIK_X         X 
0x2E    DIK_C         C 
0x2F    DIK_V         V 
0x30    DIK_B         B 
0x31    DIK_N         N 
0x32    DIK_M         M 
0x33    DIK_COMMA     , 
0x34    DIK_PERIOD  .   
0x35    DIK_SLASH     / 
0x36    DIK_RSHIFT  Shift (Right)   
0x37    DIK_MULTIPLY    * (Numpad)  
0x38    DIK_LMENU     Alt (Left)    
0x39    DIK_SPACE     Space 
0x3A    DIK_CAPITAL Caps Lock
0x3B    DIK_F1      F1  
0x3C    DIK_F2      F2  
0x3D    DIK_F3      F3  
0x3E    DIK_F4      F4  
0x3F    DIK_F5      F5  
0x40    DIK_F6      F6  
0x41    DIK_F7      F7  
0x42    DIK_F8      F8  
0x43    DIK_F9      F9  
0x44    DIK_F10     F10 
0x45    DIK_NUMLOCK Num Lock    
0x46    DIK_SCROLL  Scroll Lock 
0x47    DIK_NUMPAD7 7 (Numpad)  
0x48    DIK_NUMPAD8 8 (Numpad)  
0x49    DIK_NUMPAD9 9 (Numpad)  
0x4A    DIK_SUBTRACT  - (Numpad)    
0x4B    DIK_NUMPAD4 4 (Numpad)  
0x4C    DIK_NUMPAD5 5 (Numpad)  
0x4D    DIK_NUMPAD6 6 (Numpad)  
0x4E    DIK_ADD     + (Numpad)  
0x4F    DIK_NUMPAD1 1 (Numpad)  
0x50    DIK_NUMPAD2 2 (Numpad)  
0x51    DIK_NUMPAD3 3 (Numpad)  
0x52    DIK_NUMPAD0 0 (Numpad)  
0x53    DIK_DECIMAL . (Numpad)  
0x57    DIK_F11     F11 
0x58    DIK_F12     F12 
0x64    DIK_F13     F13 (NEC PC-98)
0x65    DIK_F14     F14 (NEC PC-98)
0x66    DIK_F15     F15 (NEC PC-98)
0x70    DIK_KANA      Kana  Japenese Keyboard
0x79    DIK_CONVERT Convert Japenese Keyboard
0x7B    DIK_NOCONVERT   No Convert  Japenese Keyboard
0x7D    DIK_YEN     ¥     Japenese Keyboard
0x8D    DIK_NUMPADEQUALS    =   NEC PC-98
0x90    DIK_CIRCUMFLEX  ^   Japenese Keyboard
0x91    DIK_AT      @   NEC PC-98
0x92    DIK_COLON     : NEC PC-98
0x93    DIK_UNDERLINE   _   NEC PC-98
0x94    DIK_KANJI     Kanji Japenese Keyboard
0x95    DIK_STOP      Stop  NEC PC-98
0x96    DIK_AX      (Japan AX)  
0x97    DIK_UNLABELED   (J3100) 
0x9C    DIK_NUMPADENTER Enter (Numpad)  
0x9D    DIK_RCONTROL    Ctrl (Right)    
0xB3    DIK_NUMPADCOMMA , (Numpad)  NEC PC-98
0xB5    DIK_DIVIDE    / (Numpad)    
0xB7    DIK_SYSRQ       Sys Rq  
0xB8    DIK_RMENU       Alt (Right) 
0xC5    DIK_PAUSE       Pause   
0xC7    DIK_HOME        Home    
0xC8    DIK_UP        ↑ (Arrow up)
0xC9    DIK_PRIOR       Page Up 
0xCB    DIK_LEFT        ←   (Arrow left)
0xCD    DIK_RIGHT       →   (Arrow right)
0xCF    DIK_END       End   
0xD0    DIK_DOWN        ↓   (Arrow down)
0xD1    DIK_NEXT        Page Down   
0xD2    DIK_INSERT    Insert    
0xD3    DIK_DELETE    Delete    
0xDB    DIK_LWIN        Windows 
0xDC    DIK_RWIN        Windows 
0xDD    DIK_APPS        Menu    
0xDE    DIK_POWER       Power   
0xDF    DIK_SLEEP       Windows
\end{verbatim}

Example code

\begin{Shaded}
\begin{Highlighting}[]
\KeywordTok{using}\NormalTok{ System;}
\KeywordTok{using}\NormalTok{ System.}\FunctionTok{Collections}\NormalTok{.}\FunctionTok{Generic}\NormalTok{;}
\KeywordTok{using}\NormalTok{ System.}\FunctionTok{Linq}\NormalTok{;}
\KeywordTok{using}\NormalTok{ System.}\FunctionTok{Text}\NormalTok{;}
\KeywordTok{using}\NormalTok{ System.}\FunctionTok{Threading}\NormalTok{.}\FunctionTok{Tasks}\NormalTok{;}

\KeywordTok{using}\NormalTok{ GTA;}
\KeywordTok{using}\NormalTok{ GTA.}\FunctionTok{Math}\NormalTok{;}
\KeywordTok{using}\NormalTok{ System.}\FunctionTok{Windows}\NormalTok{.}\FunctionTok{Forms}\NormalTok{;}
\KeywordTok{using}\NormalTok{ System.}\FunctionTok{Drawing}\NormalTok{;}
\KeywordTok{using}\NormalTok{ GTA.}\FunctionTok{Native}\NormalTok{;}
\KeywordTok{using}\NormalTok{ System.}\FunctionTok{IO}\NormalTok{;}
\KeywordTok{using}\NormalTok{ System.}\FunctionTok{Runtime}\NormalTok{.}\FunctionTok{InteropServices}\NormalTok{;}

\KeywordTok{namespace}\NormalTok{ moddingTutorial}
\NormalTok{\{}
    \KeywordTok{public} \KeywordTok{class}\NormalTok{ moddingTutorial : Script}
\NormalTok{    \{}
  
        \CommentTok{//this bunch of stuff is to control keyboard and mouse}
\NormalTok{        [}\FunctionTok{StructLayout}\NormalTok{(LayoutKind.}\FunctionTok{Sequential}\NormalTok{)]}
        \KeywordTok{public} \KeywordTok{struct}\NormalTok{ KeyboardInput}
\NormalTok{        \{}
            \KeywordTok{public} \DataTypeTok{ushort}\NormalTok{ wVk;}
            \KeywordTok{public} \DataTypeTok{ushort}\NormalTok{ wScan;}
            \KeywordTok{public} \DataTypeTok{uint}\NormalTok{ dwFlags;}
            \KeywordTok{public} \DataTypeTok{uint}\NormalTok{ time;}
            \KeywordTok{public}\NormalTok{ IntPtr dwExtraInfo;}
\NormalTok{        \}}
\NormalTok{        [}\FunctionTok{StructLayout}\NormalTok{(LayoutKind.}\FunctionTok{Sequential}\NormalTok{)]}
        \KeywordTok{public} \KeywordTok{struct}\NormalTok{ MouseInput}
\NormalTok{        \{}
            \KeywordTok{public} \DataTypeTok{int}\NormalTok{ dx;}
            \KeywordTok{public} \DataTypeTok{int}\NormalTok{ dy;}
            \KeywordTok{public} \DataTypeTok{uint}\NormalTok{ mouseData;}
            \KeywordTok{public} \DataTypeTok{uint}\NormalTok{ dwFlags;}
            \KeywordTok{public} \DataTypeTok{uint}\NormalTok{ time;}
            \KeywordTok{public}\NormalTok{ IntPtr dwExtraInfo;}
\NormalTok{        \}}
\NormalTok{        [}\FunctionTok{StructLayout}\NormalTok{(LayoutKind.}\FunctionTok{Sequential}\NormalTok{)]}
        \KeywordTok{public} \KeywordTok{struct}\NormalTok{ HardwareInput}
\NormalTok{        \{}
            \KeywordTok{public} \DataTypeTok{uint}\NormalTok{ uMsg;}
            \KeywordTok{public} \DataTypeTok{ushort}\NormalTok{ wParamL;}
            \KeywordTok{public} \DataTypeTok{ushort}\NormalTok{ wParamH;}
\NormalTok{        \}}
\NormalTok{        [}\FunctionTok{StructLayout}\NormalTok{(LayoutKind.}\FunctionTok{Explicit}\NormalTok{)]}
        \KeywordTok{public} \KeywordTok{struct}\NormalTok{ InputUnion}
\NormalTok{        \{}
\NormalTok{            [}\FunctionTok{FieldOffset}\NormalTok{(}\DecValTok{0}\NormalTok{)] }\KeywordTok{public}\NormalTok{ MouseInput mi;}
\NormalTok{            [}\FunctionTok{FieldOffset}\NormalTok{(}\DecValTok{0}\NormalTok{)] }\KeywordTok{public}\NormalTok{ KeyboardInput ki;}
\NormalTok{            [}\FunctionTok{FieldOffset}\NormalTok{(}\DecValTok{0}\NormalTok{)] }\KeywordTok{public}\NormalTok{ HardwareInput hi;}
\NormalTok{        \}}
        \KeywordTok{public} \KeywordTok{struct}\NormalTok{ Input}
\NormalTok{        \{}
            \KeywordTok{public} \DataTypeTok{int}\NormalTok{ type;}
            \KeywordTok{public}\NormalTok{ InputUnion u;}
\NormalTok{        \}}
\NormalTok{        [Flags]}
        \KeywordTok{public} \KeywordTok{enum}\NormalTok{ InputType}
\NormalTok{        \{}
\NormalTok{            Mouse = }\DecValTok{0}\NormalTok{,}
\NormalTok{            Keyboard = }\DecValTok{1}\NormalTok{,}
\NormalTok{            Hardware = }\DecValTok{2}
\NormalTok{        \}}
\NormalTok{        [Flags]}
        \KeywordTok{public} \KeywordTok{enum}\NormalTok{ KeyEventF}
\NormalTok{        \{}
\NormalTok{            KeyDown = }\BaseNTok{0x0000}\NormalTok{,}
\NormalTok{            ExtendedKey = }\BaseNTok{0x0001}\NormalTok{,}
\NormalTok{            KeyUp = }\BaseNTok{0x0002}\NormalTok{,}
\NormalTok{            Unicode = }\BaseNTok{0x0004}\NormalTok{,}
\NormalTok{            Scancode = }\BaseNTok{0x0008}
\NormalTok{        \}}
\NormalTok{        [Flags]}
        \KeywordTok{public} \KeywordTok{enum}\NormalTok{ MouseEventF}
\NormalTok{        \{}
\NormalTok{            Absolute = }\BaseNTok{0x8000}\NormalTok{,}
\NormalTok{            HWheel = }\BaseNTok{0x01000}\NormalTok{,}
\NormalTok{            Move = }\BaseNTok{0x0001}\NormalTok{,}
\NormalTok{            MoveNoCoalesce = }\BaseNTok{0x2000}\NormalTok{,}
\NormalTok{            LeftDown = }\BaseNTok{0x0002}\NormalTok{,}
\NormalTok{            LeftUp = }\BaseNTok{0x0004}\NormalTok{,}
\NormalTok{            RightDown = }\BaseNTok{0x0008}\NormalTok{,}
\NormalTok{            RightUp = }\BaseNTok{0x0010}\NormalTok{,}
\NormalTok{            MiddleDown = }\BaseNTok{0x0020}\NormalTok{,}
\NormalTok{            MiddleUp = }\BaseNTok{0x0040}\NormalTok{,}
\NormalTok{            VirtualDesk = }\BaseNTok{0x4000}\NormalTok{,}
\NormalTok{            Wheel = }\BaseNTok{0x0800}\NormalTok{,}
\NormalTok{            XDown = }\BaseNTok{0x0080}\NormalTok{,}
\NormalTok{            XUp = }\BaseNTok{0x0100}
\NormalTok{        \}}

\NormalTok{        [}\FunctionTok{DllImport}\NormalTok{(}\StringTok{"user32.dll"}\NormalTok{, SetLastError = }\KeywordTok{true}\NormalTok{)]}
        \KeywordTok{private} \KeywordTok{static} \KeywordTok{extern} \DataTypeTok{uint} \FunctionTok{SendInput}\NormalTok{(}\DataTypeTok{uint}\NormalTok{ nInputs, Input[] pInputs, }\DataTypeTok{int}\NormalTok{ cbSize);}
\NormalTok{        [}\FunctionTok{DllImport}\NormalTok{(}\StringTok{"user32.dll"}\NormalTok{)]}
        \KeywordTok{private} \KeywordTok{static} \KeywordTok{extern}\NormalTok{ IntPtr }\FunctionTok{GetMessageExtraInfo}\NormalTok{();}
        \DataTypeTok{bool}\NormalTok{ pressKey = }\KeywordTok{false}\NormalTok{;}

  
        \KeywordTok{public} \FunctionTok{moddingTutorial}\NormalTok{()}
\NormalTok{        \{}
            \KeywordTok{this}\NormalTok{.}\FunctionTok{Tick}\NormalTok{ += onTick;}
            \KeywordTok{this}\NormalTok{.}\FunctionTok{KeyUp}\NormalTok{ += onKeyUp;}
            \KeywordTok{this}\NormalTok{.}\FunctionTok{KeyDown}\NormalTok{ += onKeyDown;}
\NormalTok{        \}}

        \KeywordTok{private} \DataTypeTok{void} \FunctionTok{onTick}\NormalTok{(}\DataTypeTok{object}\NormalTok{ sender, EventArgs e) }\CommentTok{//this function gets executed continuously }
\NormalTok{        \{}
\NormalTok{        \}}

        \KeywordTok{private} \DataTypeTok{void} \FunctionTok{onKeyUp}\NormalTok{(}\DataTypeTok{object}\NormalTok{ sender, KeyEventArgs e)}
\NormalTok{        \{}
\NormalTok{        \}}

        \KeywordTok{private} \DataTypeTok{void} \FunctionTok{onKeyDown}\NormalTok{(}\DataTypeTok{object}\NormalTok{ sender, KeyEventArgs e)}
\NormalTok{        \{}
            
            \KeywordTok{if}\NormalTok{ (e.}\FunctionTok{KeyCode}\NormalTok{ == Keys.}\FunctionTok{H}\NormalTok{)}
\NormalTok{            \{   }
                \CommentTok{//define the pressing of the arrow up key}
\NormalTok{                Input[] keyUP_press = }\KeywordTok{new}\NormalTok{ Input[]}
\NormalTok{                \{}
                  \KeywordTok{new}\NormalTok{ Input}
\NormalTok{                  \{}
\NormalTok{                    type = (}\DataTypeTok{int}\NormalTok{)InputType.}\FunctionTok{Keyboard}\NormalTok{,}
\NormalTok{                    u = }\KeywordTok{new}\NormalTok{ InputUnion}
\NormalTok{                    \{}
\NormalTok{                      ki = }\KeywordTok{new}\NormalTok{ KeyboardInput}
\NormalTok{                      \{}
\NormalTok{                        wVk = }\DecValTok{0}\NormalTok{,}
\NormalTok{                        wScan = }\BaseNTok{0xC8}\NormalTok{, }\CommentTok{// key ARROW UP}
\NormalTok{                        dwFlags = (}\DataTypeTok{uint}\NormalTok{)(KeyEventF.}\FunctionTok{KeyDown}\NormalTok{ | KeyEventF.}\FunctionTok{Scancode}\NormalTok{), }\CommentTok{//key press}
\NormalTok{                        dwExtraInfo = }\FunctionTok{GetMessageExtraInfo}\NormalTok{()}
\NormalTok{                      \}}
\NormalTok{                    \}}
\NormalTok{                  \}}
\NormalTok{                \};}

                \CommentTok{//define the release of the arrow up key}
\NormalTok{                Input[] keyUP_release = }\KeywordTok{new}\NormalTok{ Input[]}
\NormalTok{                \{}
                  \KeywordTok{new}\NormalTok{ Input}
\NormalTok{                  \{}
\NormalTok{                    type = (}\DataTypeTok{int}\NormalTok{)InputType.}\FunctionTok{Keyboard}\NormalTok{,}
\NormalTok{                    u = }\KeywordTok{new}\NormalTok{ InputUnion}
\NormalTok{                    \{}
\NormalTok{                      ki = }\KeywordTok{new}\NormalTok{ KeyboardInput}
\NormalTok{                      \{}
\NormalTok{                        wVk = }\DecValTok{0}\NormalTok{,}
\NormalTok{                        wScan = }\BaseNTok{0xC8}\NormalTok{, }\CommentTok{//key ARROW UP}
\NormalTok{                        dwFlags = (}\DataTypeTok{uint}\NormalTok{)(KeyEventF.}\FunctionTok{KeyUp}\NormalTok{ | KeyEventF.}\FunctionTok{Scancode}\NormalTok{), }\CommentTok{//key release}
\NormalTok{                        dwExtraInfo = }\FunctionTok{GetMessageExtraInfo}\NormalTok{()}
\NormalTok{                      \}}
\NormalTok{                    \}}
\NormalTok{                  \}}
\NormalTok{                \};}
                
                \CommentTok{//we need to send a key press and add a delay before releasing it otherwise it's too fast for the game to register it}
                \CommentTok{//send key press of arrow up key}
                \FunctionTok{SendInput}\NormalTok{((}\DataTypeTok{uint}\NormalTok{)keyUP_press.}\FunctionTok{Length}\NormalTok{, keyUP_press, Marshal.}\FunctionTok{SizeOf}\NormalTok{(}\KeywordTok{typeof}\NormalTok{(Input)));}
                \CommentTok{//delay half a second}
                \FunctionTok{Wait}\NormalTok{(}\DecValTok{500}\NormalTok{);}
                \CommentTok{//send key release of arrow up key}
                \FunctionTok{SendInput}\NormalTok{((}\DataTypeTok{uint}\NormalTok{)keyUP_release.}\FunctionTok{Length}\NormalTok{, keyUP_release, Marshal.}\FunctionTok{SizeOf}\NormalTok{(}\KeywordTok{typeof}\NormalTok{(Input)));}
\NormalTok{            \}}
\NormalTok{        \}}
\NormalTok{    \}}
\NormalTok{\}}
\end{Highlighting}
\end{Shaded}

\hypertarget{content-replication-assignment-6}{%
\section*{Content Replication Assignment}\label{content-replication-assignment-6}}
\addcontentsline{toc}{section}{Content Replication Assignment}

\end{document}
